\documentclass[10pt,a4paper]{article}
\usepackage[utf8]{inputenc}
\usepackage{amsmath}
\usepackage{amsfonts}
\usepackage{amssymb}
\usepackage[brazilian]{babel}
\usepackage{booktabs}
\usepackage{float}
\usepackage{listings}
\usepackage{xcolor}
\usepackage[left=2cm, right=3cm, top=2cm, bottom=2cm]{geometry}

% Configuración de colores
\definecolor{mygray}{rgb}{0.5,0.5,0.5}
\definecolor{mygreen}{rgb}{0,0.6,0}
\definecolor{myblue}{rgb}{0,0,0.8}
\definecolor{myorange}{rgb}{1,0.1,0}
\definecolor{bg}{rgb}{0.99, 0.99, 0.99}
\definecolor{myred}{rgb}{0.8,0,0}
\definecolor{mygreen1}{rgb}{0,0.5,0}
\definecolor{mypurple}{rgb}{0.5,0,0.5}

% Configuración de lstlisting
\lstset{
    basicstyle=\ttfamily\small,
    breaklines=true,
    commentstyle=\color{mygray},
	commentstyle=\color{mygreen1}\itshape,
    keywordstyle=\color{myblue},
    numberstyle=\tiny\color{mygray},
    stringstyle=\color{myorange},
    frame=trbl,%single,
	framesep=0.5mm,
    frameround=tttt,  % Bordes redondeados
	rulecolor=\color{mygray},
    showstringspaces=false,
    numbers=left,
    numbersep=5pt,
    xleftmargin=15pt,
    framexleftmargin=15pt,
    framexrightmargin=5pt,
    aboveskip=15pt,
    belowskip=5pt,
    escapeinside={*@}{@*},  % Permite usar LaTeX dentro del código
    captionpos=b,
    extendedchars=true,
    inputencoding=utf8,
    language=Python,
	backgroundcolor=\color{bg},  % Fondo de color gris suave
	prebreak=\mbox{\textcolor{myred}{\scriptsize $\sqcup  $}\space }, % Símbolo tipo "u" al finalizar la línea que no cabe
	postbreak=\mbox{\textcolor{myred}{$\hookrightarrow$}\space}  % Línea roja de unión en las rupturas de línea
}

\begin{document}

\section{MODELAGEM MATEMÁTICA}

\begin{table}[H]
	\centering
	\small
	\begin{tabular}{@{}lll@{}}
		\toprule
		\textbf{Definição} & \textbf{Notação}                                                        & \textbf{Domínio}            \\ \midrule
		\multicolumn{3}{l}{\textbf{Conjuntos}}                                                                                     \\ \midrule
		I                  & Conjunto de estações de Inicio                                          &                             \\
		J                  & Conjunto de estações de Destino                                         &                             \\
		K                  & Conjunto de Classes de Control                                          &                             \\ \midrule
		\multicolumn{3}{l}{\textbf{Parâmetros}}                                                                                    \\ \midrule
		n                  & Quantidade de estações                                                  &                             \\
		Q$_k$              & Capacidade do trem para cada Classe de Control k                        & $k \in K$                   \\
		P$_{ijk}$          & Preços  dos passagem com origem i, destino j e classe de control k      & $i \in I, j \in J, k \in K$ \\
		d$_{ijk}$          & Demanda  de passagem com origem i, destino j e classe de control k      & $i \in I, j \in J, k \in K$ \\ \midrule
		\multicolumn{3}{l}{\textbf{Variáveis de decisão}}                                                                          \\ \midrule
		cap$_{ik}$         & Capacidade na estação i pra classe de control k                         & $i \in I, k \in K$          \\
		X$_{ijk}$          & Quantidade de passagem atribuídos no tramo i,j com classe de control k  & $i \in I, j \in J, k \in K$ \\
		Y$_{ijk}$          & Quantidade de passagem autorizados no tramo i,j com classe de control k & $i \in I, j \in J, k \in K$ \\ \bottomrule
	\end{tabular}
	\caption{Notação matemática}
	\label{Notacao}
\end{table}

\begin{align}
	 & Max \quad Z = \sum_{i\in I} \sum_{j\in J} \sum_{k\in K} P_{ijk} X_{ijk} \label{ecu1}                                                                & \\
	 & \text{s.a.}  \notag                                                                                                                                 & \\
	 & cap_{ik} = cap_{i-1,k} - \sum_{j\in J/j \geq i}(Y_{i-1,j,k} + X_{i-1,j,k}) + \sum_{j\in J /j<i}(Y_{jik} + X_{jik}), \quad \forall i,k  \label{ecu2} & \\
	 & \sum_{j \in J}(Y_{ijk} + X_{ijk}) \leq cap_{ik} , \quad \forall i,k/ i < n                                                \label{ecu3}              & \\
	 & Y_{ijk} \geq X_{ijk},  \quad \forall i,j,k/ i < j                                                                          \label{ecu4}             & \\
	 & Y_{ijk} \leq Y_{i,j,k+1},  \quad \forall i,j,k / i < j, k < \lVert K \rVert,  P_{ijk} \leq P_{i,j,k+1}                      \label{ecu5}            & \\
	 & X_{ijk} \leq d_{ijk},  \quad \forall i,j,k/ i < j                                                                           \label{ecu6}            & \\[15pt]
	 & X_{0,j,k} = 0,     \quad \forall j,k                                                                                        \label{ecu7}            & \\
	 & Y_{0,j,k} = 0,     \quad \forall j,k                                                                                        \label{ecu8}            & \\
	 & cap_{0,k} = Q_k,   \quad \forall k                                                                                          \label{ecu9}            & \\
	 & X_{ijk} \in \mathbb{Z}^+                                                                                                   \label{ecu10}           & \\
	 & Y_{ijk} \in \mathbb{Z}^+                                                                                                   \label{ecu11}           & \\
	 & cap_{jk} \in \mathbb{Z}^+                                                                                                   \label{ecu12}
\end{align}

Onde a restrição \ref{ecu1} representa a função objetivo, a restrição \ref{ecu2} guarda a capacidade do trem depois
de ter atribuído y autorizado os assentos das estações anteriores, a restrição \ref{ecu3} restringe que as passagem disponíveis para
saída seja menores que a capacidade disponível ate a estação i, a restrição \ref{ecu4} garanta que a quantidade de assentos atribuídos
seja menores ou iguais que os assentos autorizados, a restrição \ref{ecu6} garanta que as quantidades de assentos atribuídos seja menores
ou iguais que a demanda para cada origem i, destino j de cada classe de control k. as restrições de \ref{ecu7} até \ref{ecu9} representa
as iniciações das variáveis X, Y e cap respetivamente e pelo ultimo as restrições de \ref{ecu10} até \ref{ecu12} são o domínio das variáveis de decisão.
\\ \\


\end{document}