\chapter{Modelagem matemática}

\section{Modelagem matemática}

Dentro da pesquisa, foram realizadas duas modelagens distintas, ambas, diferentemente da literatura clássica, foram elaboradas com base nas estações de origem e destino, e não nos legs do percurso.

Para compreender essas propostas, consideremos uma versão simplificada do problema como se mostra na figura \ref{fig: fig1}, onde temos:

\begin{itemize}
	\item 4 estações pelas quais o trem deve passar em um único sentido, ou seja, o trem não tem retorno.
	\item O trem tem uma capacidade máxima de assentos.
	\item Há apenas um tipo de classe comercial.
	\item Existe apenas um período no horizonte de reserva.
	\item A variável de decisão é a quantidade de assentos que pode ser disponibilizada para venda em um trecho com origem e destino específicos.
	\item Todos os assentos disponibilizados para venda serão vendidos.
\end{itemize}

\begin{figure}[th]
	\begin{center}
		\includegraphics[scale=0.7]{img/fig1.png}
		\caption{Versão gráfica simples}
		% Fonte:~\cite{khaksar2013genetic}}
		\label{fig: fig1}
	\end{center}
\end{figure}


\section{Primeira modelagem matemática}\label{sec:modelo1}

Agora, para a primeira proposta de modelagem, temos o seguinte

\noindent $x_{ij}$: Quantidade de assentos que serão seguradas no trecho com origem em $i$ e destino em $j$, onde $j>i$ \\
\noindent $A_i$: Disponibilidade do trem na estação $i$ \\
\noindent $P_{ij}$: Preço da passagem no trajeto com origem em $i$ e destino em $j$ \\
\noindent $Q$: Capacidade do trem

Dado o exposto, a função objetivo será maximizar o lucro para cada possível venda em cada trajeto $i,j$, matematicamente seria:

$FO: max \quad x_{12}P_{12} + x_{13}P_{13} + x_{14}P_{14} + x_{23}P_{23} + x_{24}P_{24} + x_{34}P_{34}$

s.a.

Estação 1: $x_{12} + x_{13} + x_{14} \leq A_1 \quad onde \quad A_1 = Q $ \\
\indent Estação 2: $x_{23} + x_{24}  \leq  A_2 \quad onde \quad A_2 = A_1 - (x_{12} + x_{13} + x_{14}) + x_{12} $ \\
\indent Estação 3: $x_{34} \leq A_3 \quad onde \quad A_3 = A_2 - (x_{23} + x_{24}) + x_{13} + x_{23} $

Note que as restrições são aplicadas para cada uma das três primeiras estações, E1, E2 e E3, já que são as estações que têm pelo menos um destino, e a última estação, E4, é excluída, pois não possui nenhum destino.

Cada uma das restrições leva em consideração o fluxo de pessoas que sairão e entrarão no trem. Levando isso em conta, é necessário calcular a disponibilidade do trem para cada estação. Considere uma solução viável para o modelo, conforme mostrado na figura \ref{fig: fig2}, com uma capacidade total de 100 assentos para um trem.

\begin{figure}[!ht]
	\begin{center}
		\includegraphics[scale=0.4]{img/fig2.png}
		\caption{Solução factível para o problema simplificado}
		% Fonte:~\cite{khaksar2013genetic}}
		\label{fig: fig2}
	\end{center}
\end{figure}

Note que, para a restrição da estação 1, o trem está com todos os assentos vazios, ou seja, \(A_1=100\), e que a soma das variáveis seria \(x_{12} + x_{13} + x_{14} = 10 + 5 + 15 = 30\). Portanto, teríamos \(30 \leq 100\), ou seja, foram disponibilizados para venda 30 assentos dos 100 que o trem possui. Nesse sentido, no momento da partida do trem da estação 1, haveria 70 assentos vazios ou disponíveis para venda em estações posteriores.

Agora, para a estação 2, teríamos \(A_2 = 100 - 30 + 10 = 70 + 10 = 80\). Já era conhecido que havia 70 assentos disponíveis vindos da estação 1, mas também é preciso levar em conta que os assentos com destino à estação 2 também ficarão disponíveis da estação 2 em diante, para este caso \(x_{12} = 10\). Portanto, para a estação 2, teríamos 80 assentos vazios para disponibilizar, ou seja, \(60 \leq 80\). Analogamente, o mesmo raciocínio seria aplicado para a estação 3, ou seja, teríamos a soma de todos os assentos que chegaram à estação 3, \(x_{13} = 5\) e \(x_{23} = 40\), assim teríamos \(A_3 = 80 - 60 + 5 + 40 = 20 + 5 + 40 = 65\), e no final teríamos \(60 \leq 65\).

Dada a lógica anterior, vejamos agora o modelo proposto completo:

\begin{table}[h]
	\centering
	\small
	\begin{tabular}{p{2cm} p{9.5cm} p{3.2cm}}
		\toprule
		\textbf{Definição} & \textbf{Notação}                                                                                                                                                       & \textbf{Domínio}                             \\ \midrule
		\multicolumn{3}{l}{\textbf{Conjuntos}}                                                                                                                                                                                                     \\ \midrule
		$OD$               & Conjunto de Trechos com itinerario                                                                                                                                     &                                              \\
		$AD$               & Conjunto de Trechos Adjacentes com itinerario                                                                                                                          &                                              \\
		$NAD$              & Conjunto de Trechos que NÃO são Adjacentes e que tem itinerario                                                                                                        &                                              \\
		$V$                & Conjunto de Cabines do trem                                                                                                                                            &                                              \\
		$K_v$              & Conjunto de Classes de Control de cada cabine em $V$                                                                                                                   &                                              \\
		$T$                & Conjunto de Check-Points (Períodos)                                                                                                                                    &                                              \\ \midrule
		\multicolumn{3}{l}{\textbf{Parâmetros}}                                                                                                                                                                                                    \\ \midrule
		$n$                & Quantidade de Estações                                                                                                                                                 &                                              \\
		$Q$                & Capacidade do trem                                                                                                                                                     &                                              \\
		$P_{ijvk}$         & Preços  das passagem no Trecho $(i,j)$, Cabine $v$ e Classe de Control $k$                                                                                             & $(i,j) \in OD,v \in V, k \in K_v$            \\
		$D_{ijvkt}$        & Demanda  das passagem no Trecho $(i,j)$, Cabine $v$ e Classe de Control $k$                                                                                            & $(i,j) \in OD,v \in V, k \in K_v, t \in T$   \\ \midrule
		\multicolumn{3}{l}{\textbf{Variáveis de decisão}}                                                                                                                                                                                          \\ \midrule
		$A_{i}$            & Disponibilidade de passagens para vendas na estação i                                                                                                                  & $i \in I$                                    \\
		$X_{ijvkt}$        & Quantidade de passagem atribuídos no trecho $(i,j)$, cabine $v$ e com classe de control $k$ no período $t$                                                             & $(i,j) \in OD, v \in V, k \in K_v, t \in T$  \\
		$Y_{ijvkt}$        & Quantidade de passagem autorizados no trecho $(i,j)$, cabine $v$ e com classe de control $k$ no período $t$                                                            & $(i,j) \in OD, v \in V, k \in K_v, t \in T$  \\
		$BAL_{ijvt}$       & É uma variavel binaria que toma o valor de 1 quando $\sum_{k \in K_v }Y_{ijvkt} \neq 0$ e toma  valor de 0 caso contrario                                              & $(i,j) \in OD, v \in V, t \in T$             \\
		$BNA_{ijvkt}$      & É uma variavel binaria que toma o valor de 1 quando $Y_{ijvkt} \neq 0$ e toma  valor de 0 caso contrario, aplica-se apenas a trechos que não são adjacentes            & $(i,j) \in NAD, v \in V, k \in K_v, t \in T$ \\
		$Z_{ijvkt}$        & É uma variável real que assumirá três valores \{-1,0,1\}, é 1 quando a classe k da variável $Y_{ijvkt}$ é a última a assumir um valor, caso contrário pode ser -1 ou 0 & $(i,j) \in NAD, v \in V, k \in K_v, t \in T$ \\
		\bottomrule
	\end{tabular}
	\caption{Notação matemática}
	\label{tab: m1_definicao}
\end{table}

% \begin{eqnarray}
% 	&& Max \quad Z = \sum_{(i,j)\in OD} \sum_{v\in V} \sum_{k\in K_v} \sum_{t\in T} P_{ijvk} X_{ijvkt}                                                        \label{eq: m1_fo}                                                                                            \\
% 	&& \text{s.a.}  \notag                                                                                                                                                    \\
% 	&& A_{i} = A_{i-1} - \sum_{(i,j) \in OD/j \geq i} \sum_{v\in V} \sum_{k\in K_v}\sum_{t\in T}X_{i-1,j,v,k,t} + \sum_{(i,j) \in OD/j<i}\sum_{v\in V} \sum_{k\in K_v}\sum_{t\in T}X_{jivkt}, \quad \forall i \in OD  \label{eq: m1_disponi}
% \end{eqnarray}

\begin{align}
	 & Max \quad Z = \sum_{(i,j)\in OD} \sum_{v\in V} \sum_{k\in K_v} \sum_{t\in T} P_{ijvk} X_{ijvkt}                                                                                                                \label{eq: m1_fo}                          \\
	 & \text{s.a.}  \notag                                                                                                                                                                                                                                       \\
	 & A_{i} = A_{i-1} - \sum_{(i,j) \in OD/j \geq i} \sum_{v\in V} \sum_{k\in K_v}\sum_{t\in T}X_{i-1,j,v,k,t} + \sum_{(i,j) \in OD/j<i}\sum_{v\in V} \sum_{k\in K_v}\sum_{t\in T}X_{jivkt}, \quad \forall i \in OD  \label{eq: m1_disponi}                     \\
	 & \sum_{(i,j) \in OD}\sum_{v\in V} \sum_{k\in K_v}\sum_{t\in T} X_{ijvkt} \leq A_{i} , \quad \forall i \in OD /i<j, i < n                                                                                        \label{eq: m1_cap_assig}                   \\
	 & Y_{ijvkt} \geq Y_{i,j,v,k+1,t},  \quad \forall (i,j),v,k,t / i < j, k < \lVert K \rVert,  P_{ijvk} \geq P_{i,j,v,k+1}                                                                                          \label{eq: m1_jerar_class}                 \\
	 & X_{ijvkt} \leq D_{ijvkt},  \quad \forall (i,j),v,k,t/ i < j                                                                                                                                                    \label{eq: m1_assig_menor_dem}             \\
	 & \sum_{(i,j) \in OD}\sum_{v\in V}\sum_{t\in T} Y_{i,j,v,k,t} \leq Q, \quad  k = min\{K_v\}, \forall i \in OD                                                                                                    \label{eq: m1_cap_autho_1er_class}         \\
	 & Y_{i,j,v,k,t} \geq  X_{i,j,v,k,t},  \quad k = max\{K_v\}, \forall(i,j),v,t                                                                                                                                     \label{eq: m1_autho_mayor_assig_1er_class} \\
	 & Y_{i,j,v,\lVert K_v \rVert - k,t} \geq  X_{i,j,v,\lVert K_v \rVert - k,t} + Y_{i,j,v,\lVert K_v \rVert k + 1,t} , \forall(i,j),v, k \in K_v / k \leq \lVert K_v\rVert - 1, t \in T                             \label{eq: m1_autho_mayor_assig_mas_autho} \\
	 & \text{Restrições de capitalismo}  \notag                                                                                                                                                                                                                  \\
	 & BAL_{i,j,v,t} \leq \sum_{k \in K_v}Y_{i,j,v,k,t} \leq BAL_{i,j,v,t} Q, \quad  \forall (i,j)\in OD,v,t                                                                                                          \label{eq: m1_activ_bin_sum_autho}         \\
	 & BNA_{o,d,v,k,t} \leq Y_{o,d,v,k,t} \leq BNA_{o,d,v,k,t} Q, \quad  \forall (o,d)\in NAD, v, t, k                                                                                                                \label{eq: m1_activ_bin_autho}             \\
	 & BNA_{o,d,v,k,t} + BNA_{o,d,v,k-1,t} -  BNA_{o,d,v,k+1,t} -1 = Z_{o,d,v,k,t}, \quad  \forall (o,d)\in NAD, v, t, k / \lVert K_v \rVert \geq 3                                                                   \label{eq: m1_cal_betwen_autho_val}        \\
	 & BNA_{o,d,v,k,t} -  BNA_{o,d,v,k+1,t} = Z_{o,d,v,k,t}, \quad  \forall (o,d)\in NAD, v, t, k = min\{K_v\}                                                                                                        \label{eq: m1_cal_first_autho_val}         \\
	 & BNA_{o,d,v,k,t} = Z_{o,d,v,k,t}, \quad  \forall (o,d)\in NAD, v, t, k = max\{K_v\}                                                                                                                             \label{eq: m1_cal_last_autho_val}          \\
	 & Y_{i,j,v,k,t} \geq Z_{o,d,v,k,t} - (1-BAL_{i,j,v,t}), \quad  \forall (o,d)\in NAD, (i,j) \in BRI_{(o,d)}, v,t,k                                                                                                \label{eq: m1_autho_activ_noadj}           \\%[15pt]
	 & \text{Restrições de inicialização}  \notag                                                                                                                                                                                                                \\
	 & X_{0,j,v,k,t} = 0,     \quad \forall j,k,t                                                                                                                                                                     \label{eq: m1_ini_assig}                   \\
	 & A_{0} = Q                                                                                                                                                                                                      \label{eq: m1_ini_disponi}                 \\
	 & \text{Restrições de dominio}  \notag                                                                                                                                                                                                                      \\
	 & X_{ijvkt} \in \mathbb{Z}^+                                                                                                                                                                                     \label{eq: m1_dom_assig}                   \\
	 & Y_{ijvkt} \in \mathbb{Z}^+                                                                                                                                                                                     \label{eq: m1_dom_autho}                   \\
	 & A_{j} \in \mathbb{Z}^+                                                                                                                                                                                         \label{eq: m1_dom_disponi}                 \\
	 & BAL_{ijvt} \in \{0,1\}                                                                                                                                                                                    \label{eq: m1_dom_bin_all}                      \\
	 & BNA_{ijvkt} \in \{0,1\}                                                                                                                                                                                   \label{eq: m1_dom_bin_nadja}                    \\
	 & Z_{ijvkt} : Livre                                                                                                                                                                                     \label{eq: m1_dom_bin_true_last_value}
\end{align}


Na equação \ref{eq: m1_fo}, a qual representa a função objetivo, temos a soma do produto entre a quantidade de assentos atribuídos a cada trajeto de origem e destino para a classe comercial em cada período e cada vagone, multiplicada pelo preço correspondente para cada trajeto e classe, observe que queremos maximizar os ingressos em função dos asientos que estão atribuídos, que é o mais próximo que se tem da realidade em função da demanda conhecida.

A restrição \ref{eq: m1_disponi} é utilizada para calcular a disponibilidade de assentos de cada estação de origem, em cada período de tempo para cada classe em cada vagone e é a generalização do exemplo simplificado para calcular a variável de decisão $A_i$.

A restrição \ref{eq: m1_cap_assig} garante que todas as autorizações habilitadas a partir de cada estação de origem para cada período e cada classe de cada vagão não excedam a disponibilidade da sua estação de origem correspondente (a disponibilidades é calculada na restrição \ref{eq: m1_disponi}).

A restrição \ref{eq: m1_jerar_class} é uma restrição de hierarquia e garante que as quantidades de autorizações para as classes de maior preço sejam sempre maiores do que as quantidades de autorizações de menor preço em cada vagone, em cada trecho para tudos os período do horizonte de reserva.

A restrição \ref{eq: m1_assig_menor_dem} garante que a quantidade de atribuições não ultrapasse a demanda para cada trecho de cada classe em cada vagoen e em cada período no horizonte de reserva.

A restrição \ref{eq: m1_cap_autho_1er_class} garanta que a soma de autorizações da classe mais costosa de cada vagone, de tudos os periodos e de tudos os trechos não ultrapase a capacidade do trem, note que apenas estamos considerando a classe mais cara devido à natureza cumulativa das variáveis de autorização é por isso que o valor de k é o mínimo das classes de cada vagão, pois a ordem do nome das classes é crescente mas o seu valor é decrescente. Para melhor compreensão, suponhamos uma solução para um problema de dois vagones V1 e V2, 8 classes para V1 e 6 classes para V2, 10 trechos, um período e uma capacidade para 100 cadeiras, conforme mostra a figura \ref{fig: autorization}.

\begin{figure}[!ht]
	\begin{center}
		\includegraphics[scale=0.7]{img/autorization.png}
		\caption{Solução factível para a variavel de desição Autorização}
		% Fonte:~\cite{khaksar2013genetic}}
		\label{fig: autorization}
	\end{center}
\end{figure}

Observe que os nomes das classes são números ordenados em ordem crescente [1, 2, 3, ...] também o valor da classe 1 é mais caro que o valor da classe 2 e este é maior que o valor da classe classes 3 e assim por diante. Alem disso, a soma que não ultrapassará a capacidade do trem é a soma das classes 1 de cada vagone ou seja 81 + 13 que é menor a 100. Enquanto a soma total geral não teria sentido, uma vez que cada classe mais à direita conterá todas as classes à esquerda.


Até ao momento foi referido que a variável $Y$ tem um carácter cumulativo e são as restrições \ref{eq: m1_autho_mayor_assig_1er_class} e \ref{eq: m1_autho_mayor_assig_mas_autho} que controlam este comportamento. A restrição \ref{eq: m1_autho_mayor_assig_1er_class} é um caso particular da restrição \ref{eq: m1_autho_mayor_assig_mas_autho}, aplicada apenas à última classe, ou classe mais barata, segurada para cada vagão, ($k=max\{K_v\}$) e garante que a soma de todos os períodos, de cada trecho da classe mais barata da variável "autorização " é maior ou igual à variável de decisão "segurada" nas mesmas condições. Por outro lado, a restrição \ref{eq: m1_autho_mayor_assig_1er_class} garante que cada classe autorizada mais à esquerda seja sempre maior ou igual à classe autorizada imediatamente à sua direita mais a quantidade segurada da mesma classe, isto pela soma de todos os períodos, cada trecho e cada classe. Para melhor compreensão, assuma as mesmas suposições que foram feitas na restrição \ref{eq: m1_cap_autho_1er_class} com a diferença que agora as tabelas representam uma solução factivel para a soma de n períodos e não um único período.


\begin{figure}[h!]
	\centering
	\begin{subfigure}[b]{0.450\linewidth}
		\includegraphics[width=\linewidth]{img/auto_sf.png}
		\caption{Autorização [Variavel $Y$]}
		\label{fig:auto_assig_a}
	\end{subfigure}\hspace{5mm}
	\begin{subfigure}[b]{0.434\linewidth}
		\includegraphics[width=\linewidth]{img/assig_sf.png}
		\caption{Atribução [Variavel $X$]}
		\label{fig:auto_assig_b}
	\end{subfigure}
	\caption{Solução factível para as variaveis de desição Autorização e Atribução}
	\label{fig:auto_assig}
\end{figure}

Observe a linha correspondente ao trecho 1-11 do vagão V2 na tabela b, não observe que a classe segurada mais barata foi a classe 5 com valor 1 por este motivo na tabela a na mesma posição o valor deverá ser igual ou maior que 1, que neste caso é o mesmo valor; Agora observe para o mesmo trecho para o vagão v1 classe 6 em ambas as tabelas acontece a mesma coisa, esse comportamento é garantido pela restrição 6. Agora não vamos olhar para a classe mais barata, vamos olhar para qualquer outra, por exemplo, para o mesmo trecho veja a classe 4 do vagão v1 da tabela b com valor 15, se quiséssemos saber o valor correspondente na tabela a deveríamos adicionar a classe imediata a à direita da classe 4 na tabela b, neste caso seria seja a classe 5 com valor 4, e some o valor da classe 4 da tabela b, que já sabemos que é 15, assim, o valor buscado será maior ou igual a 4+15, o que resulta em 19, como visto em tabela b, lembre-se que nessa posição o valor mínimo será o calculado mas poderá assumir um valor superior. Esta última situação é controlada pela restrição 7.


As restrições de \ref{eq: m1_ini_assig} e \ref{eq: m1_ini_disponi} são usadas para inicializar a restrição \ref{eq: m1_disponi} quando \(i = 1\). E as restrições de \ref{eq: m1_dom_assig} a \ref{eq: m1_dom_disponi} representam o domínio das variáveis.

\section{Segunda modelagem matemática}\label{sec:modelo2}

Vamos considerar novamente uma versão simplificada do problema. Na verdade, para esta modelagem, serão levadas em conta as mesmas variáveis do primeiro modelo, exceto a variável de disponibilidade \(A\), conforme mostrado a seguir:

\noindent $x_{ij}$: Quantidade de assentos que será disponibilizada para venda no trecho com origem em $i$ e destino em $j$, onde $j>i$ \\
\noindent $P_{ij}$: Preço da passagem no trajeto com origem em $i$ e destino em $j$ \\
\noindent $Q$: Capacidade do trem

\noindent Assim, a função objetivo e as restrições são como segue:

$FO: max \quad x_{12}P_{12} + x_{13}P_{13} + x_{14}P_{14} + x_{23}P_{23} + x_{24}P_{24} + x_{34}P_{34}$

s.a.

\noindent{\it Restrições para estações de origem}

Estação 1: $x_{12} + x_{13} + x_{14} \leq Q $ \\
\indent Estação 2: $x_{23} + x_{24}  \leq  Q $ \\
\indent Estação 3: $x_{34} \leq Q $

\noindent{\it Restrições para estações de destino}

Estação 2: $x_{12} \leq Q $ \\
\indent Estação 3: $x_{13} + x_{23}  \leq  Q $ \\
\indent Estação 4: $x_{14} + x_{24} + x_{34} \leq Q $

Observe que esta formulação é baseada nos modelos de transporte, onde as estações de origem seriam os depósitos e estão restritas por sua capacidade (capacidade do trem), e as estações de destino seriam os destinos e estão restritas, neste caso, pela mesma capacidade do trem e não pela demanda de cada destino.

Esta formulação garante que sempre será disponibilizada, no máximo, a capacidade do trem tanto para cada saída quanto para cada chegada do trem. Imagine uma solução viável como a mostrada na figura \ref{fig: fig2}.

\begin{figure}[h]
	\begin{center}
		\includegraphics[scale=0.4]{img/fig3.png}
		\caption{Solução factível para o problema simplificado}
		% Fonte:~\cite{khaksar2013genetic}}
		\label{fig: fig3}
	\end{center}
\end{figure}

Observe que os valores das variáveis são os mesmos que foram mostrados na figura \ref{fig: fig3}. E ainda todas as restrições, tanto por linha quanto por coluna (por origens e por destinos), continuam sendo atendidas.

\begin{table}[h]
	\centering
	\small
	\begin{tabular}{p{2cm} p{9.5cm} p{3.2cm}}
		\toprule
		\textbf{Definição} & \textbf{Notação}                                                                                            & \textbf{Domínio}                            \\ \midrule
		\multicolumn{3}{l}{\textbf{Conjuntos}}                                                                                                                                         \\ \midrule
		$OD$               & Conjunto de Trechos com itinerario                                                                          &                                             \\
		$V$                & Conjunto de Cabines do trem                                                                                 &                                             \\
		$K_v$              & Conjunto de Classes de Control de cada cabine em $V$                                                        &                                             \\
		$T$                & Conjunto de Check-Points (Períodos)                                                                         &                                             \\ \midrule
		\multicolumn{3}{l}{\textbf{Parâmetros}}                                                                                                                                        \\ \midrule
		$n$                & Quantidade de Estações                                                                                      &                                             \\
		$Q$                & Capacidade do trem                                                                                          &                                             \\
		$P_{ijvk}$         & Preços  das passagem no Trecho $(i,j)$, Cabine $v$ e Classe de Control $k$                                  & $(i,j) \in OD,v \in V, k \in K_v$           \\
		$D_{ijvkt}$        & Demanda  das passagem no Trecho $(i,j)$, Cabine $v$ e Classe de Control $k$                                 & $(i,j) \in OD,v \in V, k \in K_v, t \in T$  \\ \midrule
		\multicolumn{3}{l}{\textbf{Variáveis de decisão}}                                                                                                                              \\ \midrule
		$X_{ijvkt}$        & Quantidade de passagem atribuídos no trecho $(i,j)$, cabine $v$ e com classe de control $k$ no período $t$  & $(i,j) \in OD, v \in V, k \in K_v, t \in T$ \\
		$Y_{ijvkt}$        & Quantidade de passagem autorizados no trecho $(i,j)$, cabine $v$ e com classe de control $k$ no período $t$ & $(i,j) \in OD, v \in V, k \in K_v, t \in T$ \\ \bottomrule
	\end{tabular}
	\caption{Notação matemática}
	\label{tab: m2_definicao}
\end{table}

% \renewcommand{\baselinestretch}{0.5}  
% \begin{adjustwidth}{0.5cm}{0.5cm} % Ajusta los márgenes locales
\begin{align}
	 & Max \quad Z = \sum_{(i,j)\in OD} \sum_{v\in V} \sum_{k\in K_v} \sum_{t\in T} P_{ijvk} X_{ijvkt}                                                        \label{eq: m2_fo}                                                                                                  \\
	 & \text{s.a.}  \notag                                                                                                                                                                                                                                                       \\
	 & \sum_{(i,j)\in OD}\sum_{v\in V}\sum_{k\in K_v}\sum_{t\in T}X_{ijvkt} \leq Q , \quad \forall j / j>1, i<j                                               \label{eq: m2_cap_assig_destino}                                                                                   \\
	 & \sum_{(i,j)\in OD}\sum_{v\in V}\sum_{k\in K_v}\sum_{t\in T}X_{ijvkt} \leq Q , \quad \forall i / i<n, j>i                                                \label{eq: m2_cap_assig_origem}                                                                                   \\
	 & \sum_{(i,j) \in OD}\sum_{v\in V}\sum_{t\in T} Y_{i,j,v,1,t} \leq Q                                                                                     \label{eq: m2_cap_autho_1er_class}                                                                                 \\
	 & \sum_{t\in T} Y_{i,j,v,k,t} \geq  \sum_{t\in T} X_{i,j,v,k,t},  \quad k = \lVert K_v \rVert, \forall(i,j),v                                            \label{eq: m2_autho_mayor_assig_1er_class}                                                                         \\
	 & \sum_{t\in T} Y_{i,j,v,\lVert K_v \rVert - k,t} \geq  \sum_{t\in T} X_{i,j,v,\lVert K_v \rVert - k,t} + \sum_{t\in T} Y_{i,j,v,\lVert K_v \rVert k + 1,t} , \quad \forall(i,j),v, k / k \leq \lVert K_v\rVert - 1            \label{eq: m2_autho_mayor_assig_mas_autho-1} \\
	 & Y_{ijvkt} \geq Y_{i,j,v,k+1,t},  \quad \forall (i,j),v,k,t / i < j, k < \lVert K \rVert,  P_{ijvk} \geq P_{i,j,v,k+1}                                   \label{eq: m2_jerar_class}                                                                                        \\
	 & \sum_{k \in K_v}Y_{i,j,v,k,t} \leq B_{i,j,v,t} Q, \quad  \forall (i,j)\in AD,v,t                                                                         \label{eq: m2_autho_activ_adj}                                                                                   \\
	 & Y_{i,j,v,k,t} \leq B_{o,d,v,t} Q, \quad  \forall (o,d)\in NAD, v, t, k = min\{K_v\}                                                                       \label{eq: m2_autho_activ_noadj}                                                                                \\
	 & Y_{i,j,v,k,t} \geq B_{o,d,v,t} - (1-B_{i,j,v,t}), \quad  \forall (o,d)\in NAD, (i,j) \in BRI_{(o,d)}, v,t, k = min\{K_v\}                                                                       \label{eq: m2_autho_activ_noadj}                                          \\
	 & X_{ijvkt} \leq D_{ijvkt},  \quad \forall (i,j),v,k,t/ i < j                                                                                             \label{eq: m2_assig_menor_dem}                                                                                    \\[15pt]
	 & X_{ijvkt} \in \mathbb{Z}^+                                                                                                                              \label{eq: m2_dom_assig}                                                                                          \\
	 & Y_{ijvkt} \in \mathbb{Z}^+                                                                                                                              \label{eq: m2_dom_autho}
\end{align}
% \end{adjustwidth}
Note que, na definição, não temos mais a variável de decisão de disponibilidade \(A_i\). Neste caso, a equação \ref{eq: m2_fo} representa a função objetivo que esta tentando maximizar a soma do produto entre as quantidades seguradas e os preços das mesmas, ou seja, estamos maximizando a receita em função das quantidades dos assentos que estão assegurados.
A restrição \ref{eq: m2_cap_assig_destino} garante que a quantidade total de assentos autorizados para cada destino seja a quantidade máxima de assentos do trem para todas as classes e todos os períodos.
A restrição \ref{eq: m2_cap_assig_origem} garante que a quantidade de assentos autorizados para cada origem seja no máximo a capacidade do trem para todas as classes e todos os períodos.
As restrições de \ref{eq: m2_cap_autho_1er_class} a \ref{eq: m2_dom_autho} representam o mesmo que o primeiro modelo já exposto.\\
Note que a diferenca dos dois modelos fica nas restrições de capacidade do problema.