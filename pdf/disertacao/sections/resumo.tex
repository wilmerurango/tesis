%================================= Resumo e Abstract ========================================
\chapter*{Resumo}

\vspace{-1.0cm}
\begin{quotation}
	Na atualidade, os sistemas de administração de receitas, conhecidos como { \it Revenue Management} (RM) em inglês, referem-se ao conjunto de técnicas utilizadas pela indústria para maximizar o lucro. Ele se encarrega de encontrar os produtos apropriados para os clientes certos, no momento correto e a um preço conveniente. A metodologia do RM foi desenvolvida pela primeira vez pela indústria aérea, no entanto, tem sido aplicada com sucesso em outros setores com características semelhantes, como hotelaria, transporte de passageiros, varejo, restaurantes, entre outros. O RM é uma área desafiadora e interessante que combina temas como otimização, economia, estatística inferencial e ciência do comportamento.

	Este trabalho aplica o RM na indústria de transporte ferroviário de passageiros, onde, dado o itinerário de um trem (origem, destino, hora e data de partida), é possível comprar passagens antecipadamente para a viagem. O período decorrido entre a abertura das vendas dos bilhetes até a data de partida do trem é chamado de horizonte de reserva e geralmente é discretizado em dias. Além disso, as passagens são classificadas por categoria, que chamaremos de classe comercial. Um ponto importante ao estimar a quantidade de passagens de cada classe comercial no horizonte de reserva é o modelo de demanda utilizado.

	Na literatura, alguns modelos de demanda são populares, como o modelo de demanda independente, que são modelos básicos e consistem em, dada a intenção de um cliente de pagar um valor específico por uma passagem, se não encontrar exatamente esse valor na oferta, ele prefere não fazer a compra, mesmo que o valor da passagem seja menor do que ele está disposto a pagar. É claro que essa situação é pouco usual em um ambiente real. Por outro lado, existem os modelos de demanda comportamental baseados em faixas, que oferecem aos clientes um conjunto de ofertas de acordo com uma lista de preferências. Assim, se a opção mais preferida não estiver disponível, o cliente se move na lista para escolher outra opção antes de desistir da compra.

	O objetivo desta pesquisa é desenvolver um modelo matemático linear que otimize a demanda de acordo com a lista de preferências, a fim de determinar a melhor combinação de produtos e preços que maximize o lucro. Para a validação do modelo, serão utilizados dados reais de uma empresa canadense especializada na aplicação do Revenue Management (RM) no transporte ferroviário de passageiros. Com esta proposta, espera-se superar a margem de lucro obtida até a data pela entidade que fornecerá os dados.

	\vspace{0.5cm}

	\noindent {\bf Palavras-chaves:} Demanda Comportamental, Programação Linear Inteira Mista, Transporte Ferroviário de Passageiros, Modelagem Matemática.

\end{quotation}

