%================================= Resumo e Abstract ========================================
\chapter*{Resumo}

\vspace{-1.0cm}
\begin{quotation}
	Na atualidade, os sistemas de administração de receitas, conhecidos como { \it Revenue Management} (RM) em inglês, referem-se ao conjunto de técnicas utilizadas pela indústria para maximizar o lucro. Ele se encarrega de encontrar os produtos apropriados para os clientes certos, no momento correto e a um preço conveniente. A metodologia do RM foi desenvolvida pela primeira vez pela indústria aérea, no entanto, tem sido aplicada com sucesso em outros setores com características semelhantes, como hotelaria, transporte, varejo, restaurantes, comércio eletrônico, entre outros. O RM é uma área desafiadora e interessante que combina temas como otimização, economia, estatística inferencial e ciência do comportamento.

	Este trabalho está inserido no setor de transporte, mais especificamente no transporte ferroviário de passageiros, onde o objetivo é determinar a quantidade ideal de assentos a serem reservados e disponibilizados para venda durante o período compreendido entre a abertura das vendas ao público e a partida do trem.

	O principal objetivo deste estudo foi desenvolver três modelos matemáticos inteiros mistos. O primeiro baseia-se em demandas independentes e os outros dois, em demandas comportamentais utilizando listas de preferência. A eficiência desses modelos foi avaliada em 10 instâncias reais, classificadas como grandes, médias e pequenas, fornecidas pela empresa canadense Expretio. Entre os principais achados, destaca-se que ambos os modelos atingiram resultados ótimos em tempos competitivos, em termos de segundos, além de que, em todas as instâncias, a solução ótima foi encontrada explorando, no máximo um nó.

	\vspace{0.5cm}

	\noindent {\bf Palavras-chaves:} Demanda Comportamental, Programação Inteira Mista, Transporte Ferroviário de Passageiros, Modelagem Matemática.

\end{quotation}

