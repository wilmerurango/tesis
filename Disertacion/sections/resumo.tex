%================================= Resumo e Abstract ========================================
\chapter*{Resumo}

\vspace{-1.0cm}
\begin{quotation}
	Na atualidade, os sistemas de gestão das receitas, conhecidos como { \it Revenue Management} (RM) em inglês, referem-se ao conjunto de técnicas utilizadas pela indústria para maximizar o lucro. Ele se encarrega de encontrar os produtos apropriados para os clientes certos, no momento correto e a um preço conveniente. A metodologia do RM foi desenvolvida pela primeira vez pela indústria aérea, no entanto, tem sido aplicada com sucesso em outros setores com características semelhantes, como hotelaria, transporte, varejo, restaurantes, comércio eletrônico, entre outros. O RM é uma área desafiadora e interessante que combina temas como otimização, economia, estatística inferencial e ciência do comportamento.

	Este trabalho está inserido no setor de transporte, mais especificamente no transporte ferroviário de passageiros, onde o objetivo é determinar a quantidade ideal de assentos a serem reservados e disponibilizados para venda durante o período compreendido entre a abertura das vendas ao público e a partida do trem.

	Esse problema é complexo devido à necessidade de considerar a capacidade dos trens, a estrutura hierárquica dos produtos tarifários, as reservas de bilhetes, a disponibilidade de vendas dentro do horizonte de reserva e a coerência dos preços dos bilhetes ao longo do tempo, entre outros fatores. Além disso, o comportamento dos clientes pode ser modelado de duas maneiras: como demanda independente ou como demanda comportamental. A primeira abordagem assume que os clientes compram um produto específico independentemente da oferta disponível no momento. Já a segunda considera que os clientes escolhem entre um conjunto de ofertas com base em suas preferências. Se a opção mais desejada não estiver disponível, eles passam para a próxima opção viável, desde que esta seja mais atrativa do que não realizar a compra.

	O principal objetivo deste estudo foi desenvolver dois modelos matemáticos inteiros mistos. O primeiro baseia-se em demandas independentes e o segundo é baseado em demandas comportamentais utilizando listas de preferência. A eficiência desses modelos foi avaliada em 10 instâncias reais, classificadas como grandes, médias e pequenas, fornecidas por uma empresa canadense. Entre os principais achados, destaca-se que ambos os modelos atingiram resultados ótimos em tempos competitivos, em termos de segundos, além de que, em todas as instâncias, a solução ótima foi encontrada explorando, no máximo um nó.

	\vspace{0.5cm}

	\noindent {\bf Palavras-chaves:} Gestão da Receita, Demanda Comportamental, Transporte Ferroviário de Passageiros, Programação Inteira Mista.
\end{quotation}

