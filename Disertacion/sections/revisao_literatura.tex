\chapter{Revisão da literatura }

Até o ano de 1978, a Junta de Aeronáutica Civil (CAB em inglês) limitava a concorrência entre as companhias aéreas, onde basicamente as companhias só podiam competir oferecendo serviços como refeições luxuosas e alta frequência nos horários de saída dos voos. Nesse ponto, a CAB não permitia que fosse oferecida uma tarifa menor para um voo, se esta fosse antieconômica para a indústria como um todo. Assim, mesmo que para uma companhia aérea fosse rentável colocar um valor baixo para uma passagem em comparação com outra, a CAB não permitiria, a menos que houvesse uma justificativa extremamente sólida. Quando esse tipo de situação ocorria, o restante das companhias aéreas justificava que o público seria prejudicado, pois elas teriam que aumentar o valor das passagens em outras rotas para compensar o baixo custo da nova proposta do concorrente.

Com a chegada da desregulamentação, as companhias aéreas se depararam com um mundo cheio de novas formas de concorrência, onde o preço das passagens se tornou prioritário. E foi nesse momento precioso que iniciou a verdadeira concorrência entre as transportadoras. Aqui surgiu um novo problema em função da diversidade de preços com diferentes restrições que limitam a disponibilidade de assentos a tarifas mais baixas, a presença de múltiplos voos operados por diversas companhias aéreas em diferentes rotas, e a variabilidade na demanda por assentos em função de fatores como a temporada, o dia da semana, a hora do dia e a qualidade do serviço oferecido, o que influencia a escolha dos passageiros entre diferentes opções de voo.

Nesse momento, esse problema foi denominado como problema de preços e combinação de passageiros e foi modelado como: cada passageiro em um voo representa um custo de oportunidade, já que sua ocupação de um assento impede que outro passageiro com um itinerário mais rentável ou uma classe de tarifa mais alta o utilize. Isso se traduz na possibilidade de assentos vazios em diferentes segmentos de voo, o que afeta a eficiência da rede da companhia aérea ao considerar múltiplos passageiros com diversas origens, destinos e classes de tarifas.

Houve dois possíveis resultados: 1) a otimização da combinação de passageiros permite que as companhias aéreas estruturem de maneira mais eficaz seu sistema de reservas, estabelecendo limites e prioridades adequadas para o número de passageiros com diferentes classes de tarifas em distintos voos. 2) Além disso, possibilita a avaliação de diversos cenários de preço e rota, considerando o benefício gerado a partir da melhor combinação de passageiros em relação a um cenário específico.

Ao ajustar a estrutura das classes de tarifas, as companhias aéreas buscam gerenciar o deslocamento de passageiros por meio de estratégias de preços e a aplicação de restrições como horários, duração da estadia e tempo de antecedência à saída do voo. Além disso, buscam reduzir o deslocamento controlando a capacidade, determinando a quantidade de assentos atribuídos a cada classe de tarifa em cada segmento de voo.

Por outro lado, a otimização da combinação de passageiros é formulada como: "Dada a previsão diária da demanda de passageiros nas diferentes classes de tarifas, qual combinação de passageiros e classes de tarifas em cada segmento de voo maximizará as receitas do dia?" Essa resposta ajuda a companhia aérea a determinar a alocação ideal de reservas entre as diversas classes de tarifas em cada segmento de voo \cite{article_base}. 

Essas últimas duas definições foram conhecidas como Yield Management e, posteriormente, com a chegada de novos sistemas de informação, regras de controlo e outras condições, foram generalizadas e aplicadas em outras indústrias de características semelhantes, que no futuro seriam chamadas de Revenue Management \cite{article_YM_to_RM}.

Según \cite{article_Ryzin2014}, a gestão de receitas (RM) abrange o conjunto de estratégias e táticas que as empresas utilizam para gerenciar de forma científica a demanda por seus produtos e serviços. Além disso, pode-se dizer que, Seu objetivo é vender cada unidade de ações para o cliente certo, no momento e pelo preço corretos \cite{doi:10.1080/02642069.2010.491543}.

A princípio, os problemas de gestão de RM parecem ser simples; no entanto, nada poderia estar mais longe da realidade. Esses problemas têm uma complexidade esmagadora, e este documento não seria suficiente para detalhar cada um deles, apenas para mencionar alguns, temos modelagem, análise teórica, implementação, demanda, entre outros. Como sempre acontece, trabalha-se com simplificações de fatores muito complexos e com aproximações em outros casos \cite{doi:10.1287/trsc.33.2.233}