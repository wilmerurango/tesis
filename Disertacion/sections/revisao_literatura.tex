\chapter{Revisão da literatura}


\section{Fundamentos de Revenue Management (RM)}
Nesta seção, será apresentada a definição de Revenue Management, suas principais características, um breve histórico da sua evolução e suas aplicações em diferentes setores da indústria.

\subsection{Definição e importância}
A gestão de receitas, conhecido como RM, do inglês \textit{Revenue Management}, é uma abordagem estratégica voltada para otimizar a disponibilidade e os preços dos produtos, com o objetivo de maximizar o crescimento da receita  procurando os produtos apropriados para os clientes certos, no momento adequado e por um preço conveniente \citep{Robert1997}. Essa prática envolve a previsão do comportamento do consumidor em nível de micromercado e a gestão científica da demanda por produtos e serviços \citep{4736064}.

O RM evoluiu desde suas origens na indústria aérea até se tornar uma prática amplamente adotada em diversos setores, incluindo hotelaria, energia, manufatura e trasnporte \citep{Cross1995ANIT, Cheraghi2010RevenueMI}. A implementação do RM geralmente exige a análise de dados históricos e da atividade atual de pedidos, a fim de prever a demanda com precisão. Isso permite que as empresas definam e atualizem estratégias de preço e disponibilidade de produtos nos diferentes canais de venda \citep{Cheraghi2010RevenueMI}.

O RM possibilita que cada unidade de capacidade disponível seja aproveitada ao máximo, segmentando clientes, controlando a disponibilidade de produtos e ajustando preços de forma dinâmica \citep{HEO2009446}. Além de elevar a rentabilidade, essa estratégia melhora a previsibilidade da demanda, permitindo que as empresas reajam rapidamente às oscilações do mercado. Em um ambiente altamente competitivo, o gerenciamento de receitas torna-se um diferencial crucial para manter uma posição sólida e, ao mesmo tempo, oferecer soluções mais eficientes e personalizadas para os consumidores \citep{Gallego1994}.

Quando aplicada com êxito, essa abordagem pode gerar impactos econômicos expressivos, com aumentos de receita superiores a 5\%, conforme relatado em diversas indústrias \citep{Cross1995ANIT}. À medida que a área continua a se desenvolver, vem incorporando modelos avançados de previsão e otimização da demanda, baseados em estudos das áreas de ciência da gestão e economia \citep{Cheraghi2010RevenueMI}.

\subsection{Histórico e evolução}
Até o ano de 1978, a Junta de Aeronáutica Civil (CAB em inglês) limitava a concorrência entre as companhias aéreas, onde basicamente as companhias só podiam competir oferecendo serviços como refeições luxuosas e alta frequência nos horários de saída dos voos. Nesse ponto, a CAB não permitia que fosse oferecida uma tarifa menor para um voo, se esta fosse antieconômica para a indústria como um todo. Assim, mesmo que para uma companhia aérea fosse rentável colocar um valor baixo para uma passagem em comparação com outra, a CAB não permitiria, a menos que houvesse uma justificativa extremamente sólida. Quando esse tipo de situação ocorria, o restante das companhias aéreas justificava que o público seria prejudicado, pois elas teriam que aumentar o valor das passagens em outras rotas para compensar o baixo custo da nova proposta do concorrente \citep{article_base}.

Com a chegada da desregulamentação, as companhias aéreas se depararam com um mundo cheio de novas formas de concorrência, onde o preço das passagens se tornou prioritário. E foi nesse momento que iniciou a verdadeira concorrência entre as transportadoras. Aqui surgiu um novo problema em função da diversidade de preços com diferentes restrições que limitam a disponibilidade de assentos a tarifas mais baixas, a presença de múltiplos voos operados por diversas companhias aéreas em diferentes rotas, e a variabilidade na demanda por assentos em função de fatores como a temporada, o dia da semana, a hora do dia e a qualidade do serviço oferecido, o que influencia a escolha dos passageiros entre diferentes opções de voo \citep{article_base}.

Nesse momento, esse problema foi denominado como problema de preços e combinação de passageiros e foi modelado como: cada passageiro em um voo representa um custo de oportunidade, já que sua ocupação de um assento impede que outro passageiro com um itinerário mais rentável ou uma classe de tarifa mais alta o utilize. Isso se traduz na possibilidade de assentos vazios em diferentes segmentos de voo, o que afeta a eficiência da rede da companhia aérea ao considerar múltiplos passageiros com diversas origens, destinos e classes de tarifas \citep{article_base}.

Houve dois possíveis resultados: 1) a otimização da combinação de passageiros permite que as companhias aéreas estruturem de maneira mais eficaz seu sistema de reservas, estabelecendo limites e prioridades adequadas para o número de passageiros com diferentes classes de tarifas em distintos voos. 2) Além disso, possibilita a avaliação de diversos cenários de preço e rota, considerando o benefício gerado a partir da melhor combinação de passageiros em relação a um cenário específico.

Ao ajustar a estrutura das classes de tarifas, as companhias aéreas buscam gerenciar o deslocamento de passageiros por meio de estratégias de preços e a aplicação de restrições como horários, duração da estadia e tempo de antecedência à saída do voo. Além disso, buscam reduzir o deslocamento controlando a capacidade, determinando a quantidade de assentos atribuídos a cada classe de tarifa em cada segmento de voo.

Por outro lado, a otimização da combinação de passageiros é formulada como: "Dada a previsão diária da demanda de passageiros nas diferentes classes de tarifas, qual combinação de passageiros e classes de tarifas em cada segmento de voo maximizará as receitas do dia?" Essa resposta ajuda a companhia aérea a determinar a alocação ideal de reservas entre as diversas classes de tarifas em cada segmento de voo \citep{article_base}.

Essas últimas duas definições foram conhecidas como Yield Management e, posteriormente, com a chegada de novos sistemas de informação, regras de controle e outras condições, foram generalizadas e aplicadas em outras indústrias de características semelhantes, que no futuro seriam chamadas de Revenue Management \citep{article_YM_to_RM}.

% =========

Subsequentemente, entre os anos de 2013 e 2017, o campo do RM passou por uma evolução significativa, impulsionada pelos avanços nas tecnologias analíticas, pelo acesso a dados em tempo real e pela crescente complexidade dos mercados. Esse processo de transformação foi marcado por uma mudança de perspectiva: as decisões antes centradas exclusivamente na empresa passaram a considerar com mais profundidade o comportamento do consumidor em ambientes competitivos. De acordo com \citep{Mulyani2021TheFH}, o interesse por RM cresceu de forma exponencial, refletindo sua relevância tanto na pesquisa acadêmica quanto na prática empresarial.

Nos primeiros anos do período, a atenção da literatura acadêmica ainda se concentrava nos dois pilares clássicos do RM: as decisões de precificação (pricing) e as de quantidade ou capacidade (quantity-based decisions), conforme discutido por \citep{talluri2004theory}. No entanto, uma análise de palavras-chave conduzida por \citep{Mulyani2021TheFH} mostra que, enquanto o termo resource allocation perdia importância ao longo do tempo, o conceito de pricing ganhava destaque, consolidando-se como o principal foco até o final do período analisado.

A partir de 2014, começaram a ganhar espaço linhas de pesquisa voltadas à análise do comportamento do consumidor. Temas como escolha do consumidor, percepção de preço, sensação de justiça (fairness) e dispersão de preços passaram a ser frequentemente abordados nos estudos, em resposta à necessidade de compreender melhor o processo de decisão de compra em mercados altamente competitivos. Essa mudança de foco para uma abordagem mais orientada ao cliente está em sintonia com as observações de \citep{Noone2011}, que identificaram uma transição do RM de um enfoque centrado na empresa para uma perspectiva mais sensível ao ambiente competitivo e às preferências dos usuários.

Do ponto de vista metodológico, o período foi marcado pela crescente adoção de técnicas quantitativas mais sofisticadas. Métodos tradicionais, como programação dinâmica e estocástica, modelos de otimização e heurísticas, continuaram sendo amplamente utilizados. Entretanto, a partir de 2015 — com maior intensidade entre 2016 e 2017 — abordagens como teoria dos jogos, modelos logit e análise conjunta (conjoint analysis) ganharam destaque. Essas técnicas refletem a necessidade de capturar as respostas estratégicas de consumidores e concorrentes dentro dos modelos de decisão \citep{Mulyani2021TheFH, Shen2007,STRAUSS2018375}.

Por fim, uma das transformações mais relevantes desse período foi a incorporação de tecnologias emergentes, como big data e machine learning, que começaram a ser aplicadas com mais frequência nos estudos sobre RM nos últimos anos da década. Esse movimento foi favorecido pelo avanço dos sistemas de captura e processamento de dados em tempo real, o que permitiu maior sofisticação na abordagem de problemas dinâmicos e complexos. Essa tendência, destacada por \citep{Gönsch2013}, evidencia a necessidade de métodos empíricos mais robustos e adaptáveis, que complementem os modelos teóricos tradicionais.

% De acordo com  \citep{article_Ryzin2014}, o RM abrange o conjunto de estratégias e táticas que as empresas utilizam para gerenciar de forma científica a demanda por seus produtos e serviços. Além disso, pode-se dizer que, seu objetivo é vender cada unidade de ações para o cliente certo, no momento e pelo preço corretos \citep{doi:10.1080/02642069.2010.491543}.

% A princípio, os problemas de gestão de RM parecem ser simples; no entanto, nada poderia estar mais longe da realidade. Esses problemas têm uma complexidade esmagadora, e este documento não seria suficiente para detalhar cada um deles, apenas para mencionar alguns, temos modelagem, análise teórica, implementação, previsão, vendas excessivas, controle de estoque de assentos, preços, etc. Então, como sempre acontece, trabalha-se com simplificações de fatores muito complexos e com aproximações em outros casos \citep{doi:10.1287/trsc.33.2.233}.


\section{Transporte Ferroviário de Passageiros}
O problema do transporte ferroviário de passageiros tem sido amplamente explorado na literatura científica, dada sua complexidade operacional e seu papel estratégico na promoção da mobilidade sustentável. De modo geral, esse problema busca maximizar os benefícios econômicos e sociais decorrentes do uso eficiente da infraestrutura ferroviária, levando em consideração, ao mesmo tempo, as restrições de capacidade, a qualidade do serviço prestado e as expectativas dos usuários \citep{Huisman_2005, Shan2024}. O sistema ferroviário possui características específicas que o diferenciam de outros modos de transporte: a oferta é altamente perecível — um assento não vendido em determinado trem representa uma perda definitiva —, a capacidade é limitada pelo número de composições, pela disposição dos assentos e pela configuração da rede, e a demanda apresenta flutuações e segmentações marcantes, variando conforme o horário, o dia da semana, a distância percorrida e o perfil dos usuários \citep{Guan2023, Besinovic2022}.

Além disso, as decisões operacionais no setor são altamente interdependentes: o planejamento de linhas, a alocação de assentos, a definição de políticas tarifárias e a programação dos trens se influenciam mutuamente, exigindo, portanto, uma abordagem integrada para que os resultados sejam consistentes e sustentáveis \citep{Huisman_2005, Shan2024}. Outro aspecto relevante é a conectividade entre os serviços, que requer estratégias eficientes de gestão de transferências e de atrasos, com o objetivo de minimizar os impactos ao passageiro diante de eventuais interrupções ou imprevistos operacionais \citep{Konig2020}.

\subsection{Áreas de pesquisa}

A literatura especializada apresenta diversos enfoques para tratar esse problema sob diferentes perspectivas. Um dos mais relevantes é o RM, que adapta técnicas originalmente desenvolvidas para a aviação comercial, com o intuito de maximizar a receita por meio da gestão dinâmica de preços e da alocação de assentos, considerando a segmentação do mercado e a sensibilidade dos passageiros em relação ao preço \citep{Ammirato2020, Guan2023}. Essa abordagem evoluiu significativamente, partindo de modelos estáticos para formulações dinâmicas mais sofisticadas, integrando previsão de demanda, otimização tarifária e controle de inventário, mesmo em contextos com alta incerteza \citep{HETRAKUL201468, Shan2024}.

Outro campo amplamente discutido é o da gestão de atrasos (delay management), que tem como foco a tomada de decisões operacionais orientadas a reduzir os impactos de perturbações na rede, sobretudo na coordenação entre trens com conexões interdependentes \citep{Konig2020}. A planificação de linhas e a alocação de recursos operacionais também é tema recorrente, envolvendo decisões relacionadas às rotas, paradas, frequências, capacidades e uso do material rodante, geralmente com base em modelos de programação matemática \citep{Huisman_2005, Shan2024}.

Por outro lado, pode-se encontrar o planejamento das paradas dos trens e a estratégia de preços as quais estão interligados e impactam tanto as receitas quanto a experiência dos passageiros. Em \citep{zhou2022nonlinear} propuseram um modelo de otimização não linear inteiro misto que aborda conjuntamente a estratégia de preços dos bilhetes e o planejamento das paradas. Esse modelo busca maximizar as receitas do transporte ferroviário e minimizar o tempo de viagem dos passageiros, alcançando um equilíbrio eficiente entre oferta e demanda.

A estimação da demanda tem sido tratada, em grande parte, por meio de técnicas baseadas em modelos de escolha discreta, como os modelos logit e os modelos de classes latentes, permitindo capturar a heterogeneidade nas preferências dos usuários quanto a horários, preços e atributos do serviço ofertado \citep{HETRAKUL201468, TANG2022}. Mais recentemente, observa-se uma aplicação crescente de inteligência artificial e aprendizado de máquina em diferentes estágios do sistema ferroviário — desde a previsão de demanda e falhas operacionais até a otimização em tempo real, aproveitando-se da ampla disponibilidade de dados gerados por sistemas de bilhetagem, sensores embarcados e plataformas digitais \citep{Besinovic2022, TANG2022}.

\subsection{Controle de inventário}

É importante destacar que esta pesquisa está voltada à aplicação do RM no contexto do transporte ferroviário de passageiros. Dentro desse escopo, a literatura especializada identifica dois grandes eixos de desenvolvimento. O primeiro diz respeito ao controle de inventário, reconhecido como o alicerce operacional do RM ferroviário. Essa área tem como principal objetivo determinar, ao longo do horizonte de reservas, quantos assentos devem ser protegidos para cada tarifa ou classe de controle, considerando mercados origem-destino distintos. A meta é maximizar a receita esperada ao final do ciclo de vendas. Para isso, as políticas de aceitação e rejeição de solicitações devem lidar com um dilema fundamental: rejeitar hoje uma reserva de menor valor que talvez não seja substituída, ou aceitá-la e comprometer a disponibilidade futura para passageiros dispostos a pagar tarifas mais altas.

Diversos trabalhos ao longo das últimas décadas têm aprofundado esse campo sob diferentes enfoques metodológicos. \citep{Feng2001}, por exemplo, adaptaram pela primeira vez ao transporte ferroviário de longa distância o modelo dinâmico de controle de inventário originalmente desenvolvido para companhias aéreas. Utilizando processos de decisão de Markov com chegadas Poisson independentes por classe tarifária, os autores propuseram uma política ótima baseada no tempo restante até a partida e no número de assentos disponíveis, resultando em uma regra de limiar estacionária de fácil implementação, com ganhos simulados de aproximadamente 6\% frente à política tradicional de ordem de chegada. No entanto, a independência entre classes tarifárias pressuposta pelo modelo ignora substituições entre tarifas, o que limita sua aplicabilidade prática.

Em uma linha complementar, \citep{bertsimas2002} abordaram o problema em redes com rotas sobrepostas, propondo um modelo não linear convexo com decomposição aninhada. Seus resultados indicam que proteger assentos em nível de itinerário completo pode gerar ganhos entre 3\% e 6\% em relação a políticas locais por trecho, embora o modelo trate a demanda como fixa e desconsidera interações com os preços. Posteriormente, \citep{Walczak2007} introduziram um modelo semi-Markov que incorpora o tempo de permanência em cada estado como variável influente sobre a demanda. Através de simulações de Monte Carlo, demonstraram que em cenários com aceleração da demanda próxima à data de partida, é possível obter melhorias de receita entre 5\% e 8\%, ainda que o modelo se limite a apenas dois níveis tarifários e não tenha validação empírica.

Em 2009, \citep{CHEW2009} integraram, pela primeira vez, decisões conjuntas de tarifa e proteção de inventário para produtos perecíveis, com um modelo biestágio resolvido por programação dinâmica. Comparando abordagens simultâneas e sequenciais, evidenciaram ganhos médios de receita de até 18\% em um cenário de trem de alta velocidade na Ásia, embora o estudo tenha sido restrito a um único par origem-destino e não considere efeitos de rede. Na mesma direção, \citep{Cizaire2011} estendeu o horizonte de análise para múltiplos períodos com demanda estocástica, aplicando um modelo linear inteiro misto com heurísticas de horizonte rolante. Os ganhos reportados variaram entre 10\% e 12\% em comparação com políticas fixas, mas a alta sensibilidade aos erros de previsão e os custos computacionais dificultam sua aplicação em tempo real.

No campo da escalabilidade, \citep{MENDEZDIAZ2014} propuseram uma formulação inteira mista com algoritmo branch-and-price e geração dinâmica de colunas, conseguindo gaps ótimos inferiores a 2\% em instâncias reais, dentro de tempos computacionais aceitáveis. Ainda assim, a demanda continua sendo tratada como exógena, e o impacto das tarifas dinâmicas sobre as escolhas dos passageiros não é modelado. Em uma abordagem mais integrada, \citep{HETRAKUL201468} combinaram um modelo de classes latentes com uma função log-linear de demanda, permitindo otimização simultânea de preços e proteção de inventário. Os autores demonstraram que incorporar a heterogeneidade de preferências dos passageiros pode gerar ganhos de 14\% a 25\%, reforçando a relevância de considerar o comportamento individual na modelagem da demanda. No entanto, o estudo foi limitado a um único corredor e não explorou a robustez da solução frente a perturbações operacionais.

Mais recentemente, \citep{Besinovic2022} revisaram o papel da inteligência artificial nesse domínio, propondo uma taxonomia de aplicações de aprendizado de máquina ao controle de inventário. Identificaram que algoritmos de aprendizado por reforço oferecem promissoras políticas adaptativas que dispensam pressupostos rígidos sobre a distribuição da demanda. Contudo, destacam que barreiras regulatórias e desafios de interpretabilidade ainda limitam sua adoção em contextos industriais reais.

\subsection{Modelagem da demanda}

Além do controle de inventário, outro eixo essencial da literatura é a modelagem da demanda, com destaque para a distinção entre abordagens que assumem demanda independente e aquelas que adotam uma perspectiva comportamental. Nos primeiros modelos, como os propostos por \citep{Dana1999, Feng2001}, cada classe tarifária é representada como um fluxo estocástico separado, cuja chegada não é afetada pela disponibilidade de outras tarifas. Essa simplificação facilita a modelagem dinâmica e a aplicação de processos markovianos, mas ignora os efeitos de substituição tarifária observados na prática.

Esse panorama começa a mudar a partir de 2007, com os estudos de \citep{Walczak2007}, que introduzem uma estrutura semi-Markov em que a probabilidade de chegada de cada segmento passa a depender do estado atual das reservas, capturando assim os primeiros efeitos de desvio entre tarifas. Contudo, o avanço mais significativo ocorre em 2014, quando \citep{HETRAKUL201468} propõem um modelo de escolha baseado em classes latentes que permite prever com maior precisão quando e qual tarifa será adquirida por cada passageiro, levando em conta preço, tempo de antecipação e atributos do serviço. Ao integrar esse modelo à otimização de inventário, os autores demonstram que considerar a heterogeneidade de preferências pode elevar a receita em até 25\% em comparação com estratégias baseadas em demandas independentes.

Nos últimos anos, a gestão de reservas de bilhetes no transporte ferroviário de passageiros evoluiu significativamente graças à aplicação de modelos matemáticos avançados. Esses modelos têm como objetivo otimizar a alocação de assentos, o planejamento de paradas e as estratégias de preços, buscando maximizar as receitas e melhorar a satisfação dos passageiros.

Um dos modelos destacados é o desenvolvido por \citep{zhou2023pricing}, que integra a teoria das perspectivas, o modelo logit e um modelo de transferência de fluxo de passageiros para alocar a demanda de maneira eficaz. Esse enfoque permite estabelecer preços diferenciados e distribuir os assentos de forma a maximizar as receitas, considerando as preferências e comportamentos de diferentes segmentos de passageiros.

Além disso, a demanda de passageiros está sujeita a incertezas que tornam a gestão operacional mais desafiadora. Han e Ren em 2020 desenvolveram um modelo que otimiza conjuntamente o planejamento de paradas e a alocação de bilhetes, utilizando a teoria da incerteza. Esse modelo busca maximizar a satisfação dos passageiros e a taxa média de ocupação dos assentos, oferecendo soluções robustas frente às flutuações na demanda \citep{han2020uncertainty}.

Posteriormente, em \citep{schoebel2021tariff}, investiga-se como a escolha da rota dos passageiros é influenciada pelas estruturas tarifárias e pelos preços dos bilhetes. Sua pesquisa abordou o problema de determinar a tarifa mais econômica em sistemas de transporte público, avaliando diferentes estruturas tarifárias, como aquelas baseadas em distância ou zonas, e propondo algoritmos altamente eficientes para resolver esses problemas.

A evolução da literatura mostra de forma clara que os modelos de demanda comportamental tornaram-se indispensáveis para estimar de forma realista a canibalização entre horários e níveis tarifários, desenvolver estratégias de preços personalizados por meio de canais digitais e avaliar como segmentos mais sensíveis ao preço reagem a metas de sustentabilidade ou justiça tarifária. Apesar desses avanços, revisões recentes como as de \citep{Ammirato2020} e \citep{Besinovic2022} indicam que grande parte dos estudos sobre RM ferroviário ainda operam sob suposições de demanda independente, o que evidencia a necessidade urgente de aprofundar abordagens comportamentais apoiadas por aprendizado de máquina e dados massivos de transações.

\section{Aplicação de RM ao transporte ferroviário de passageiros}
O RM aplicado ao transporte ferroviário de passageiros evoluiu de maneira desigual ao longo dos últimos vinte e cinco anos, passando de modelos conceituais inspirados na aviação para propostas integradas que combinam previsão de demanda, precificação dinâmica e alocação de capacidade com o uso de técnicas de inteligência artificial.

O marco inicial pode ser atribuído ao estudo de \cite{Dana1999}, que, embora não tenha foco específico no setor ferroviário, introduz o conceito de dispersão tarifária em mercados com capacidade limitada. Sua abordagem analítica demonstra que, mesmo diante de custos fixos, a diferenciação de preços gera equilíbrios de mercado mais lucrativos que políticas de preços uniformes. Essa conclusão antecipa a lógica de proteção de assentos por classe tarifária, mais tarde adotada amplamente no setor ferroviário. Na mesma época, não se encontram estudos ferroviários diretamente comparáveis, o que evidencia um período de transição e adoção inicial de conceitos oriundos da aviação comercial.

Com o avanço da década, \citep{Feng2001} propõem um modelo dinâmico de controle de inventário voltado à gestão de assentos perecíveis, cujo algoritmo determina políticas ótimas de aceitação e rejeição de reservas. Apesar do caso base estar focado na aviação, a estrutura matemática do modelo — um processo de decisão de Markov — revela-se diretamente aplicável ao transporte ferroviário de longa distância, estabelecendo as bases para os estudos posteriores em redes ferroviárias. Em relação ao trabalho de Dana, Feng e Xiao ampliam o horizonte de análise ao introduzir decisões periódicas no tempo e substituem a estática de preços por regras de controle que incorporam a estocasticidade nas chegadas de demanda, característica alinhada à realidade operacional dos trens.

Nos anos seguintes, a literatura ainda era fortemente marcada por abordagens conceituais e revisões setoriais. Contudo, o trabalho de \citep{Vinod2004}, traz uma importante contribuição ao introduzir a noção de integração entre o RM e os sistemas de distribuição, enfatizando a necessidade de alinhar previsões de demanda com canais de venda multicanais — uma questão que se tornaria central para o sucesso do RM no setor ferroviário de alta velocidade europeu.

Em paralelo, \citep{Netessine2005} modelam a competição tarifária entre operadores utilizando jogos de inventário. Seus resultados indicam que a coordenação horizontal entre empresas pode aumentar os lucros totais do sistema. Embora o estudo se aplique ao setor aéreo, as conclusões são diretamente comparáveis a observações do mercado ferroviário interurbano liberalizado, no qual acordos de compartilhamento de código e redes integradas atuam para mitigar guerras de preços, como já destacado por \cite{Ammirato2020}.

Nos anos seguintes, dois estudos ganham destaque. \citep{Walczak2007} propõem um modelo de precificação dinâmica com base em informação semi-Markov, otimizando a receita ao considerar a probabilidade de chegada de diferentes classes de demanda heterogêneas. Simultaneamente, \citep{FROIDH2008268} analisa a viabilidade comercial do trem de alta velocidade na Suécia, alertando que seu êxito econômico dependerá da implementação de estratégias de RM que combinem preços variáveis com ajustes finos nas frequências de operação. Enquanto o modelo de Walczak apresenta rigor matemático, carece de aplicação empírica no contexto ferroviário; por outro lado, Fröidh oferece evidência de campo, mas sem formalização teórica, evidenciando uma lacuna metodológica característica daquele momento.

No biênio seguinte, \citep{CHEW2009} formulam um problema conjunto de precificação e alocação de inventário para produtos perecíveis com dois períodos de vida útil, resolvido via programação dinâmica. Demonstram que alinhar tarifas e capacidade oferece vantagens significativas sobre políticas sequenciais. Esses achados dialogam diretamente com \citep{Gao2010}, que aplicam um algoritmo de alocação de assentos no contexto ferroviário chinês, considerando elasticidades cruzadas. Enquanto Chew comprova os ganhos em termos de receita (até 20\% em simulações), Gao valida a eficácia do modelo no mundo real, com melhorias de ocupação e equilíbrio entre trajetos curtos e longos. A comparação entre ambos os trabalhos evidencia a transição de marcos teóricos para aplicações operacionais concretas.

Prosseguindo, \citep{Guo2012}, abordam a questão da amostragem truncada da demanda — problema recorrente no setor ferroviário, onde os dados de rejeições de reserva nem sempre são observáveis. A solução proposta, baseada em algoritmos de Expectation-Maximization, aprimora as estimativas para alimentar os modelos de controle, complementando o trabalho anterior de \citep{Walczak2007} ao fornecer dados mais consistentes.

Em 2014, uma mudança de paradigma é estabelecida com a contribuição de \citep{HETRAKUL201468}. Esses autores introduzem um sistema baseado em modelos de classes latentes que integra a escolha do momento da compra com regressões log-lineares de demanda e otimização conjunta de tarifas e alocação de capacidade em uma rede real. A calibração com dados históricos demonstra aumentos de receita entre 14\% e 25\% em comparação com políticas fixas, confirmando a importância de adaptar os preços à antecipação da compra. Comparado ao estudo de \citep{Dobson2013}, que se concentra na dinâmica de fusões no setor aéreo, a pesquisa de \citep{HETRAKUL201468} destaca a microsegmentação da demanda como fator crítico, ilustrando a transição do foco em estruturas de mercado para o modelamento comportamental individualizado.

Nos anos seguintes, observa-se uma crescente convergência entre otimização de redes e políticas tarifárias. \citep{ZHENG2016} aplicam o RM ao desenho de redes multimodais com estacionamento limitado e tarifas dinâmicas, demonstrando que a coordenação entre trem e automóvel pode reduzir tempos de viagem e aumentar receitas. Em paralelo, \citep{Casey2014}, embora publicado no final do período anterior, propõe um protótipo industrial que conecta scheduling com RM em ferrovias interurbanas dos Estados Unidos, reforçando a tendência de integração entre planejamento e estratégia tarifária.

subsequentemente, \citep{TANG2022} revisam o uso da inteligência artificial em diversas subáreas do setor ferroviário, constatando que a aplicação da IA à precificação dinâmica ainda é incipiente. Ao mesmo tempo, \citep{Ammirato2020} denunciam a carência de estudos que integrem RM com planejamento estratégico. Ambos concordam na necessidade de migrar de modelos isolados para sistemas holísticos de apoio à decisão, baseados em big data.

Já nos anos mais recentes, \citep{Besinovic2022} propõem uma taxonomia detalhada sobre IA no setor ferroviário, identificando o RM como uma das áreas com maior potencial para aplicação de técnicas de machine learning, especialmente na definição de preços em tempo real e otimização de inventário. Essa análise complementa a síntese de \citep{TANG2022}, que quantifica a concentração da literatura em manutenção e evidencia a defasagem na vertente comercial. A convergência dos estudos reforça a popularização do uso de redes neurais profundas na previsão de demanda e a ausência de sua articulação com algoritmos de precificação.

Mais recentemente, \cite{Guan2023} oferecem a primeira revisão dedicada exclusivamente à precificação ferroviária sob a ótica do RM, classificando os métodos em três grupos: sistemas de tarifa base, programação matemática e abordagens orientadas por dados. Os autores observam que os modelos dinâmicos ainda dependem fortemente de suposições de elasticidade linear e que a aplicação de aprendizado de máquina na estimativa de parâmetros ainda está em estágio inicial.

Em paralelo, \citep{Shan2024} articulam uma estrutura generalizada de RM, já implementada na operação ferroviária chinesa, que integra previsão, planejamento de oferta, controle de inventário e precificação. Seus resultados reportam ganhos de utilização e rentabilidade decorrentes da segmentação baseada em big data, embora ressaltem o desafio de equilibrar lucro e inclusão social. Assim, enquanto Guan fornece uma síntese teórica robusta, Shan comprova a viabilidade prática do RM em larga escala, evidenciando a transição concreta do plano acadêmico para aplicações industriais.

\section{Outras aplicações industriais de RM}
As práticas de RM são mais comuns em setores que apresentam estoques perecíveis, capacidade fixa, altos custos fixos e uma sensibilidade variável ao preço por parte do cliente. Diversas indústrias adotaram essas técnicas para otimizar suas operações e maximizar a lucratividade.

\subsection{Aviação}
No setor aéreo, as companhias enfrentam o desafio de administrar um número limitado de assentos por voo, os quais são ofertados com preços variados conforme o perfil do cliente, a antecedência da compra ou a classe da passagem. O foco principal da gestão de receitas na aviação não é simplesmente vender todos os assentos, mas sim maximizar a receita total do voo, priorizando o ganho por assento disponível (conhecido como Revenue per Available Seat Kilometer - RASK).

Para isso, são tomadas decisões estratégicas como: quantos assentos disponibilizar em cada faixa tarifária; quando abrir ou fechar o acesso a determinadas tarifas — por exemplo, bloqueando tarifas promocionais quando há alta demanda —; e quais ações promocionais implementar para estimular a procura em períodos de baixa ocupação.

Entre os pilares desse modelo estão: \textit{A segmentação tarifária}, na qual os mesmos assentos são vendidos por preços diferentes (como tarifas promocionais versus flexíveis); \textit{A previsão de demanda}, que estima o volume de bilhetes que será vendido em cada categoria; \textit{O controle de inventário}, que busca reservar lugares para clientes com maior disposição de pagamento, geralmente mais próximos da data do voo; E por fim, \textit{A gestão de overbooking}, que autoriza vender mais bilhetes do que assentos disponíveis, considerando possíveis cancelamentos ou faltas.

A American Airlines foi pioneira na aplicação do RM na indústria aérea com o desenvolvimento do sistema DINAMO (Dynamic Inventory and Maintenance Optimizer), considerado o primeiro sistema operacional de RM. Esse sistema possibilitou o aumento da rentabilidade ao controlar a disponibilidade de assentos por classes tarifárias, antecipando o comportamento da demanda e segmentando os passageiros com base em sua disposição a pagar \citep{Smith1992}.

Nas suas fases iniciais, o RM baseava-se em modelos estáticos para a alocação ótima de capacidade entre diferentes classes tarifárias. Um dos primeiros modelos formais foi a Regra de Littlewood \citep{article_YM_to_RM}, que sugeria aceitar uma reserva de uma classe com tarifa mais baixa apenas se a receita esperada por manter o assento disponível para uma classe superior fosse inferior à tarifa oferecida. Essa regra foi posteriormente estendida por \citep{Belobaba1987}, que introduziu o modelo EMSR (Expected Marginal Seat Revenue), generalizando o princípio de Littlewood para múltiplas classes tarifárias. O EMSR tornou-se um padrão nos sistemas comerciais de RM e, até hoje, continua sendo utilizado em versões adaptadas.

Com o aumento da complexidade das redes de voos, surgiu a necessidade de otimizar não apenas cada voo individualmente, mas também a rede como um todo. Essa demanda levou ao desenvolvimento do conceito de Network Revenue Management (NRM), que considera as interdependências entre voos conectados. Pesquisadores como \citep{Gallego_Ryzin1997} propuseram modelos de controle estocástico baseados em programação dinâmica para maximizar a receita esperada em redes sob incerteza de demanda. \citep{Williamson1992AirlineNS} desenvolveu metodologias de controle de inventário em redes aéreas com múltiplas classes tarifárias.

Um componente crítico do RM no setor aéreo tem sido o gerenciamento de overbooking, ou seja, a prática de aceitar mais reservas do que a capacidade física disponível, com o objetivo de compensar cancelamentos e a ausência de passageiros. \citep{Brumelle_1993} propuseram modelos probabilísticos que determinam níveis ótimos de sobrevenda, equilibrando a receita esperada com o risco de incorrer em custos com compensações. Esses modelos têm sido essenciais para aprimorar a eficiência de ocupação sem comprometer a experiência do passageiro.

Durante a segunda metade da década de 1990, a abordagem do RM evoluiu para modelos mais dinâmicos e adaptativos. \citep{Talluri_Ryzin1998} introduziram os controles por preços sombra (bid-price controls), uma técnica de implementação eficiente para redes, que estabelece limites mínimos de aceitação com base no valor marginal esperado de um assento. Paralelamente, \citep{Gallego1994} desenvolveram modelos de precificação dinâmica ótima sob demanda estocástica e horizonte finito, integrando a definição de preços ao modelo de alocação de capacidade — um marco importante na transição para uma gestão conjunta de tarifas e recursos.

No final da década de 1990 e início dos anos 2000, a literatura passou a incorporar também o componente comportamental do consumidor. \citep{McGill_Ryzin1999} realizaram uma revisão crítica da pesquisa em Revenue Management, destacando a necessidade de compreender melhor a percepção dos consumidores diante de políticas tarifárias complexas. Nesse período, começou-se a considerar o comportamento estratégico do passageiro, no qual os consumidores ajustam suas decisões de compra antecipando possíveis alterações nos preços ou na disponibilidade futura. Esse fenômeno introduziu um novo nível de complexidade ao campo, exigindo modelos capazes de prever não apenas a demanda, mas também as reações racionais dos passageiros.

No final dos anos 2000, a pesquisa em Revenue Management passou a se diversificar para cenários com incerteza, substituição de produtos e decisões estruturais. \citep{Birbil2009} propuseram modelos robustos de alocação de assentos sob incerteza, enquanto \citep{ZHANG2009848} introduziram processos de decisão de Markov para preços em voos substituíveis, representando uma transição para estratégias de precificação sensíveis ao contexto competitivo. Ambos os estudos expandiram o arcabouço tradicional, incorporando a análise de concorrência direta e decisões de natureza estrutural.

Nos anos seguintes, \citep{HUANG2011333} avançaram no desenvolvimento de algoritmos de programação dinâmica para problemas de rede, utilizando aproximações de receitas esperadas. Paralelamente, \citep{Rusdiansyah_Dwi} aplicaram modelos de precificação dinâmica que consideram a variação de inventário e o comportamento dos concorrentes. Já \citep{ASLANI201456} exploraram a percepção de justiça tarifária com o uso de heurísticas e simulações, destacando o conflito entre equidade percebida e maximização de receita.

Entre 2015 e 2018, surgiram modelos que integravam controle de capacidade, sobrevenda e cancelamentos. \citep{OTERO2015188} apresentaram um modelo estocástico de precificação dinâmica em contextos multiclasse, ampliando a literatura ao incorporar estruturas mais realistas de demanda. \citep{WANG2018173}, por sua vez, demonstraram que a alocação dinâmica de capacidade em canais de venda diferenciados aumenta a receita e reduz custos com intermediários, reforçando a importância de decisões simultâneas em ambientes multicanal.

Mais recentemente, observou-se o uso crescente de técnicas de otimização combinatória e modelos de programação estocástica inteira. \citep{LI2022100054} desenvolveram um modelo combinado para alocação de assentos e upgrade de classe sob incerteza. \citep{FUKUSHI2022297} integraram simulação com modelos de escolha discreta, capturando o comportamento heterogêneo dos passageiros e permitindo decisões personalizadas de precificação e capacidade. Esses avanços consolidam um arcabouço integrado que une demanda, precificação e disponibilidade em ambientes simulativos.

Paralelamente, o uso de aprendizado profundo tem revolucionado a previsão de demanda. \citep{HE2023109707} aplicaram redes do tipo Long Short-Term Memory (LSTM) para prever a demanda aérea, obtendo uma melhora significativa na acurácia das previsões em relação aos métodos tradicionais - com uma redução de 45,1\% no erro absoluto médio. Esse tipo de abordagem facilita a tomada de decisão adaptativa em tempo real, fator essencial em mercados altamente voláteis.

Além disso, a aplicação de machine learning permitiu aprimorar estratégias de precificação personalizada. \citep{THIRUMURUGANATHAN2023103759} desenvolveram o modelo Price Elasticity Model (PREM), capaz de identificar segmentos com maior propensão a aceitar upgrades de classe. Utilizando uma base com mais de 64 milhões de registros, o modelo previu um aumento de 37,2\% na receita proveniente de ofertas aceitas e uma redução de 7,94\% no envio de ofertas irrelevantes, evidenciando o valor da segmentação baseada no comportamento do consumidor.

% O trabalho seminal de \citep{talluri2004theory} estabeleceu as bases metodológicas para a gestão de receitas no setor aéreo, com ênfase na segmentação da demanda por meio de modelos de controle de inventário e precificação diferenciada. No contexto brasileiro, \citep{Oliveira2003SimulatingRM} desenvolveu um modelo pioneiro utilizando simulação computacional e teoria dos jogos para analisar interações competitivas em rotas estratégicas, como Rio de Janeiro - São Paulo. Esse estudo demonstrou que a diferenciação estratégica de preços poderia aumentar a eficiência do mercado entre 18\% e 22\%, embora também tenha identificado riscos de diluição de receita devido à concorrência agressiva.

% A programação dinâmica tem sido um dos pilares da otimização de receitas no setor aéreo desde os seus primórdios. Um estudo de referência publicado em 2015 analisou sua aplicação em estratégias de precificação dinâmica sob a ótica do conceito de mental accounting (contabilidade mental) dos passageiros \citep{hu2015}. Esse trabalho evidenciou como o grau de profundidade da contabilidade mental influencia positivamente as receitas esperadas, utilizando modelos estocásticos para ajustar os preços tanto em voos únicos quanto em múltiplos trechos. Para resolver o problema, os autores aplicaram equações de Bellman, estabelecendo assim um marco importante para a personalização de tarifas com base no comportamento do consumidor \citep{hu2015}.

% No entanto, ao final dos 2018, os sistemas baseados em programação dinâmica começaram a apresentar limitações diante de novos desafios como a forte dependência de padrões históricos de demanda, a dificuldade para processar dados em tempo real em grande escala e a incapacidade de integrar variáveis não tradicionais (como preferências por serviços acessórios) \citep{Chen2022, Buyruk2022}. Essas limitações impulsionaram a integração com técnicas de machine learning, embora, até 2020, a programação dinâmica ainda fosse amplamente utilizada em sistemas legados de gestão de receitas.

% Uma tese do MIT publicada em 2020 evidenciou que a programação dinâmica continuava sendo um componente fundamental em modelos híbridos que combinavam métodos clássicos de otimização com simulações de cenários competitivos.

\subsection{Hotelaria}
Na indústria hoteleira, o RM é aplicado para maximizar a receita por quarto disponível (RevPAR), considerando a variação de preços e a ocupação dos quartos. Os hotéis utilizam técnicas de previsão de demanda para ajustar suas tarifas em tempo real, levando em conta fatores como sazonalidade, eventos locais e comportamento do consumidor. A segmentação de clientes também desempenha um papel crucial, permitindo que os hotéis ofereçam pacotes personalizados e promoções direcionadas.

Entre 2006 e 2008, os estudos sobre gestão de receitas no setor hoteleiro passaram a sistematizar os fundamentos do RM aplicados ao contexto da hospitalidade. \citep{Avinal15022006} apresentou uma análise dos sistemas de gestão de receitas (RMS) sob uma perspectiva estratégica e operacional, destacando seu principal objetivo: maximizar a receita por meio de técnicas de precificação dinâmica e controle da duração da estadia. Ressaltou-se a importância da integração entre os RMS e os sistemas de gestão de propriedades (PMS), bem como da aplicação de regras operacionais — como datas específicas de chegada e estadias mínimas — para segmentar a demanda, especialmente entre os públicos corporativo e de lazer. Esse período também se caracteriza pela identificação de fatores críticos de sucesso, como o comprometimento da liderança, a avaliação contínua do desempenho dos sistemas e a capacitação da equipe para alimentar os algoritmos com dados confiáveis e de alta qualidade.

No intervalo de 2009 a 2012, consolida-se uma abordagem mais estrutural e sistêmica da gestão de receitas na hotelaria. \citep{Ivanov2011} propuseram uma revisão crítica do RM hoteleiro, organizando-o como um sistema composto por centros geradores de receita (quartos, alimentos e bebidas, eventos, spa), ferramentas de precificação (como discriminação de preços e garantias de tarifa mínima), processos operacionais (controle de overbooking e gestão de disponibilidade) e fases sequenciais que abrangem a coleta de dados, análise, previsão, tomada de decisão, implementação e monitoramento. Além disso, os autores introduzem uma discussão inovadora sobre os aspectos éticos do RM, relacionados à percepção de justiça por parte dos clientes, e reforçam a necessidade de integração entre o RM e os sistemas de gestão de relacionamento com o cliente (CRM), estabelecendo as bases para abordagens mais centradas no consumidor em etapas posteriores.

Durante o triênio de 2013 a 2015, a atenção da literatura se volta à implementação de modelos estatísticos avançados. Observa-se a adoção sistemática de métodos como o modelo ARIMA com sazonalidade (SARIMA), modelos de suavização exponencial com dupla sazonalidade (Holt-Winters) e regressões com variáveis econômicas exógenas. Embora fora do intervalo em questão, o estudo de \citep{PEREIRA201613} é representativo dessa linha de pesquisa, ao aprofundar o uso de séries temporais de alta frequência (diárias), com múltiplos e complexos padrões de demanda. Esses métodos passam a servir de base para avaliar o desempenho de abordagens mais recentes baseadas em inteligência artificial, estabelecendo métricas comparativas para medir o valor agregado por modelos emergentes.

Entre 2016 e 2018, tem início a incorporação sistemática de modelos de aprendizado de máquina (machine learning, ML) na previsão de demanda hoteleira. \citep{henriques2024hotel} explicam que, nesse período, os estudos passaram a combinar técnicas estatísticas tradicionais com algoritmos de ML, ampliando a capacidade preditiva ao integrar grandes volumes de dados estruturados e não estruturados. Destacam-se modelos como as redes neurais artificiais (ANN), máquinas de vetores de suporte (SVM) e regressões regularizadas (como Ridge e Lasso), aplicados tanto a séries históricas quanto a reservas futuras, com destaque para o uso do modelo pickup. Esta fase também revela uma preocupação crescente com a qualidade dos dados, a seleção de variáveis relevantes e a robustez dos modelos diante de padrões não lineares e oscilações no comportamento do consumidor.

A partir de 2019 e até 2021, observa-se uma expansão significativa das técnicas de deep learning (DL) na previsão de demanda na hotelaria. \citep{dowlut2023forecasting} documentam essa evolução por meio de uma revisão sistemática que contempla 50 artigos publicados entre 2017 e 2022. O estudo identifica as redes neurais LSTM (Long Short-Term Memory) como a arquitetura mais adotada, devido à sua capacidade de capturar dependências temporais de longo alcance e lidar com séries com alta sazonalidade. Além disso, surgem modelos híbridos, como o CNN-LSTM, que combinam a extração de características espaciais das redes convolucionais com o modelamento sequencial das LSTM. Essa etapa também marca o uso de algoritmos como Random Forest, XGBoost e técnicas de ensamble dinâmico, como stacking e arbitrating. \citep{Pereira02092022}, por exemplo, compararam 22 modelos e constataram que o uso de ML pode reduzir o RMSE em até 54\% em previsões de curto prazo, quando comparado a abordagens tradicionais.

Entre 2022 e 2024, a literatura passa a adotar soluções ainda mais avançadas, como o aprendizado por reforço (Reinforcement Learning, RL) e a integração da IA a ecossistemas digitais inteligentes. \citep{Chen2023} apresentam a aplicação de um modelo de RL em uma rede hoteleira na China, totalmente integrado ao sistema ERP corporativo. Por meio de um experimento de campo controlado, o sistema obteve melhorias significativas em métricas-chave, como taxa de ocupação (OR), tarifa média diária (ADR) e RevPAR, superando de forma consistente o grupo de controle. O algoritmo demonstrou capacidade para tomada de decisão automatizada, adaptativa e em tempo real, evidenciando o potencial da IA para substituir intervenções manuais na gestão de receitas diária. Em paralelo, \citep{henriques2024hotel} desenvolvem uma revisão sobre a aplicação da IA na hotelaria, na qual destacam a consolidação de modelos híbridos que integram LSTM, SVR, Random Forest e XGBoost com dados internos (reservas, cancelamentos, tarifas) e dados externos (eventos, clima, avaliações de clientes e preços da concorrência). Além disso, propõem um enfoque de “smart hospitality”, voltado à sustentabilidade, personalização, decisões orientadas por feedback digital e à construção de sistemas autônomos e inteligentes.

Por fim, \citep{mahaboobsubani2024} inaugura uma nova linha de investigação com os chamados LangChain Models, uma arquitetura de IA encadeada que integra previsão de demanda, precificação dinâmica, otimização de inventário e gestão da percepção de justiça em um sistema único e interconectado. Esses modelos são projetados para operar em tempo real dentro de ecossistemas hoteleiros 4.0, os quais incorporam tecnologias como blockchain, robótica e reconhecimento facial. A proposta busca substituir progressivamente as decisões humanas táticas por processos automatizados, inteligentes e sustentáveis em rede.

De forma geral, essa evolução representa uma transformação profunda nas abordagens de previsão de demanda hoteleira, que passaram de metodologias manuais e estáticas para sistemas altamente automatizados, adaptativos e integrados. As contribuições analisadas não apenas aumentam a precisão preditiva, mas também redefinem o papel do gestor de receitas, que passa a atuar como um estrategista de soluções inteligentes, sustentáveis e centradas no cliente. O uso crescente da IA abre caminho para sistemas que não apenas maximizam receitas, mas também promovem uma gestão ética, personalizada e resiliente frente a ambientes competitivos e em constante mudança.


\section{Contextualização desta pesquisa}
Para contextualizar essa pesquisa dentro do estado da arte, a Tabela \ref{resumo_pesquissas} apresenta uma síntese dos estudos mais relevantes sobre o transporte ferroviário de passageiros nos últimos 25 anos.


\newcolumntype{M}[1]{>{\centering\arraybackslash}m{#1}}

\begin{center}
	\tiny %\footnotesize % <-- Aplica tamaño pequeño a todo el entorno
	\renewcommand{\arraystretch}{1} % Espaciado entre filas
	\setlength{\tabcolsep}{2pt} % Reduce el espacio entre columnas

	\begin{longtable}{|M{2cm}|M{3.7cm}|M{0.5cm}|M{1.1cm}|M{1cm}|M{2cm}|M{1.7cm}|M{1.7cm}|M{1.5cm}|}
		\caption{Resumo dos trabalhos revisados} \label{resumo_pesquissas}                                                                                                                                                                                                                                                                                                                                                                                                                                            \\
		\hline
		\textbf{Autor(es)}        & \textbf{Objetivo}                                                                                                                                                     & \setstretch{0.9} \textbf{Multi-Trem} & \textbf{Indústria}   & \setstretch{0.9} \textbf{Tipo de função objetivo}         & \textbf{Modelagem}                                                       & \setstretch{0.9} \textbf{Método de solução}                      & \setstretch{0.9} \textbf{Tipo de Demanda} & \setstretch{0.9} \textbf{Inteligência artificial}                          \\
		\hline
		\endfirsthead

		\multicolumn{9}{c}%
		{\tablename\ \thetable\ -- \textit{Continuação da página anterior}}                                                                                                                                                                                                                                                                                                                                                                                                                                           \\
		\hline
		\textbf{Autor(es)}        & \textbf{Objetivo}                                                                                                                                                     & \setstretch{0.9} \textbf{Multi-Trem} & \textbf{Indústria}   & \setstretch{0.9} \textbf{Tipo de função objetivo}         & \textbf{Modelagem}                                                       & \setstretch{0.9} \textbf{Método de solução}                      & \setstretch{0.9} \textbf{Tipo de Demanda} & \setstretch{0.9} \textbf{Inteligência artificial}                          \\
		\hline
		\endhead

		\hline \multicolumn{9}{r}{\textit{Continua na próxima página}}                                                                                                                                                                                                                                                                                                                                                                                                                                                \\
		\endfoot

		\hline
		\endlastfoot

		\cite{CLAESSENS1998474}   & \setstretch{0.9} Minimizar os custos operacionais, ao mesmo tempo que designa as rotas                                                                                & Não                 & Ferroviária          & \setstretch{0.9} Objetivo único          & \setstretch{0.9} MIP                                                     & Aproximado                                      & -                        & -                                                         \\ \hline
		\citep{CHANG200091}       & \setstretch{0.9} Minimizar os custos totais e minimizar a perda de tempo de viagem, ao mesmo tempo em que determina o plano ideal de paradas                          & Não                 & Ferroviária          & \setstretch{0.9} Multi-objetivo          & \setstretch{0.9} Matemática difusa                                       & Exata                                           & Independente             & -                                                         \\ \hline
		\citep{Dessouky01022006}  & \setstretch{0.9} Determinar os tempos ótimos de partidas dos trens                                                                                                    & Sim                 & Ferroviária          & \setstretch{0.9} -                       & \setstretch{0.9} -                                                       & \setstretch{0.9}Algoritmo de ramificação        & -                        & -                                                         \\ \hline
		\citep{Zhang2009}         & \setstretch{0.9} Maximizar a receita                                                                                                                                  & -                   & Aérea                & \setstretch{0.9} Objetivo único          & \setstretch{0.9} Programação dinâmica                                    & \setstretch{0.9}Algoritmo de geração de colunas & Comportamental           & -                                                         \\ \hline
		\citep{CORMAN201114}      & \setstretch{0.9} Reprogramar trens com diferentes classes prioritárias                                                                                                & Sim                 & Ferroviária          & \setstretch{0.9} Objetivo único          & \setstretch{0.9} Formulação gráfica                                      & Aproximado                                      & -                        & -                                                         \\ \hline
		\citep{Erdelyi2011}       & \setstretch{0.9} Maximizar a receita, ao mesmo tempo que define os preços                                                                                             & -                   & Aérea                & \setstretch{0.9} Objetivo único          & \setstretch{0.9} Programação linear determinística                       & Aproximado                                      & -                        & -                                                         \\ \hline
		\citep{HETRAKUL20131}     & \setstretch{0.9} Maximizar a receita, ao mesmo tempo que encontra a melhor distribuição da demanda                                                                    & Não                 & Ferroviária          & \setstretch{0.9} Objetivo único          & \setstretch{0.9} Modelos logit multinomial, classe latente e logit misto & Aproximado                                      & Comportamental           & -                                                         \\ \hline
		\citep{HETRAKUL201468}    & \setstretch{0.9} Maximizar as receitas, enquanto se realiza a definição de preços e a alocação de assentos                                                            & Sim                 & Ferroviária          & \setstretch{0.9} Objetivo único          & \setstretch{0.9} Estocástica (MNL e LC)                                  & Aproximado                                      & -                        & -                                                         \\ \hline
		\citep{WEN201791}         & \setstretch{0.9} Examinar a relação entre o momento da compra de passagens aéreas e diferentes características do viajante                                            & -                   & Aérea                & \setstretch{0.9} Objetivo único          & \setstretch{0.9} Programação não linear                                  & Exata                                           & -                        & -                                                         \\ \hline
		\citep{QI2018151}         & \setstretch{0.9} Minimizar o número de assentos vazios e o número total de paradas                                                                                    & Sim                 & Ferroviária          & \setstretch{0.9} Objetivo único          & \setstretch{0.9} Programação linear de volume duplo de integração mista  & Exata                                           & Independente             & -                                                         \\ \hline
		\citep{XU201882}          & \setstretch{0.9} Minimizar os custos de reserva antecipada ao escolher itinerários                                                                                    & Sim                 & Ferroviária          & \setstretch{0.9} Objetivo único          & \setstretch{0.9} Programação linear usando multiplicadores de Lagrange   & Exata                                           & Independente             & -                                                         \\ \hline
		\citep{SUN201896}         & \setstretch{0.9} Previsão das escolhas dos passageiros do trem de alta velocidade (HSR) quanto aos horários de compra de passagens, tipos de trem e classes de viagem & -                   & Ferroviária          & \setstretch{0.9} Objetivo único          & \setstretch{0.9} Modelo de aprendizado de máquina não paramétrico        & Aproximado                                      & -                        & \setstretch{0.9} Regressão vetorial (SVR); Redes neurais artificiais (ANN) \\ \hline
		\citep{ZHAO2019776}       & \setstretch{0.9} Maximizar a receita, ao mesmo tempo que calcula a reserva de assentos                                                                                & Sim                 & Ferroviária          & \setstretch{0.9} Objetivo único          & \setstretch{0.9} LP                                                      & Aproximado                                      & Comportamental           & -                                                         \\ \hline
		\citep{BARBIER20201002}   & \setstretch{0.9} Maximizar a receita, controlando o estoque de assentos homogêneos                                                                                    & -                   & Aérea                & \setstretch{0.9} Objetivo único          & \setstretch{0.9} MIP                                                     & Aproximado                                      & Comportamental           & -                                                         \\ \hline
		\citep{AGASUCCI2023}      & \setstretch{0.9} Otimizar o despacho de trens                                                                                                                         & Sim                 & Ferroviária          & \setstretch{0.9} Objetivo único          & \setstretch{0.9} Abordagem de aprendizado Q profundo                     & Aproximado                                      & -                        & \setstretch{0.9} Deep Q-Learning (Descentralizado e Centralizado)          \\ \hline
		\citep{SARHANI2024100120} & \setstretch{0.9} Melhorar a previsão de atrasos de trens                                                                                                              & -                   & Ferroviária          & \setstretch{0.9} Objetivo único          & \setstretch{0.9} Aprendizado de máquina                                  & -                                               & -                        & \setstretch{0.9} Aprendizado de máquina                                    \\ \hline
		\citep{TANG2025103900}    & \setstretch{0.9} Reduzir o tempo total de atraso do trem                                                                                                              & Sim                 & Ferroviária          & \setstretch{0.9} Objetivo único          & \setstretch{0.9} Modelo de aprendizado por reforço profundo multitarefa  & Aproximado                                      & -                        & \setstretch{0.9} Aprendizado por reforço e redes neurais                   \\ \hline
		\textbf{Esta pesquisa}    & \setstretch{0.9} \textbf{Maximizar a receita, ao mesmo tempo em que calcula as autorizações e reservas de assentos}                                                   & \textbf{Não}        & \textbf{Ferroviária} & \setstretch{0.9} \textbf{Objetivo único} & \setstretch{0.9} \textbf{MIP}                                            & \textbf{Exata}                                  & \textbf{Comportamental}  & \textbf{-}                                                \\ \hline
	\end{longtable}
\end{center}

% É importante observar que esta pesquisa está fundamentada em características clássicas semelhantes às de várias pesquisas realizadas até o momento, conforme mostrado na Tabela \ref{resumo_pesquissas}. No entanto, embora o enfoque seja similar, o design e a estrutura matemática dos modelos propostos, baseados no problema de transporte e nos problemas de fluxo, não são encontrados na literatura revisada de forma conjunta. Assim, este estudo representa uma contribuição valiosa ao avaliar a eficiência da combinação de ambos os abordagens.

É importante destacar que esta pesquisa está fundamentada em características clássicas que são comuns a diversas investigações realizadas até o momento, conforme ilustrado na Tabela  \ref{resumo_pesquissas}. Contudo, embora o enfoque metodológico adotado compartilhe semelhanças com os estudos revisados, a proposta apresentada se distingue pela aplicação simultânea de abordagens baseados em problemas de transporte de veículos e fluxos en redes. Embora essas abordagens individuais já tenham sido exploradas de forma separada na literatura, não se encontrou um modelo que integre ambas as perspectivas ao mesmo tempo. 




Dessa forma, o trabalho desenvolvido representa uma contribuição relevante ao avaliar e validar a eficiência dessa combinação de abordagens, fornecendo novas perspectivas para a resolução do problema de transporte ferroviário de passageiros do ponto de vista do RM.