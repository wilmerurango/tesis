\chapter{Fundamentação teórica}
\section{Sistemas de manufatura celular}

\cite{ham2012group} descreveram que uma célula de manufatura pode ser definida como uma agrupamento independente de máquinas funcionalmente diferentes. Tais máquinas estão localizadas juntas no chão da fábrica e são dedicadas à manufatura de uma família de produtos semelhantes.

Os sistemas de manufatura celular (CMS) produzem uma forma eficiente de reduzir custos, melhorar a qualidade dos produtos fabricados e aumentam a flexibilidade dos processos de manufatura~\citep{zhang2011modeling}. CMSs foram propostos como uma alternativa às oficinas de trabalho, uma vez que fornecem os benefícios operacionais da produção em linha de fluxo. A base dos sistemas de manufatura celular são grupos de ferramentas de máquinas e famílias de produtos. Cada agrupamento de máquinas forma uma célula de manufatura dedicada à fabricação de uma família de produtos. 

Podem ser usados vários critérios para definir o foco de manufatura de uma célula, os quais são determinados principalmente pela variedade de processos incluídos na célula (usinagem, retificação, inspeção, montagem, etc.) e/ou na variedade de produtos atribuídas à célula. Outros critérios como volume de produção, segmento de mercado e grau de automação também podem definir uma célula. Uma célula pode ser dedicada a produtos solicitados com frequência, independentemente da variedade de ferramentas, materiais, processos e requisitos de retenção. Além disso, considerando os segmentos de mercado, uma célula poderia ser dedicada a algumas empresas que encomendam os mesmos produtos~\citep{irani1999handbook}. 

\subsection{Família de produtos e tecnologia de grupo}

Na fabricação em lote, a grande variedade de produtos diferentes e pequenos em tamanhos contribuem para as ineficiências de design e fabricação, devido principalmente a algumas imprecisões no planejamento, na estimativa de custos, grandes estoques e problemas de entrega. Uma forma de solucionar essas dificuldades é a classificação dos produtos em famílias que possuem atributos de desenho semelhantes, como a forma do produto, tamanho, integridade da superfície, tipo de material, estado da matéria prima, etc. Eles também podem ser classificadas a partir dos atributos de fabricação semelhantes, como operações em sequência, tamanho do lote, máquinas e ferramentas de corte, tempos de processamento, volumes de produção, etc. Esses atributos permitem que se recuperem desenhos existentes para oferecer suporte a novos produtos. Embora dois produtos sejam idênticos em forma e tamanho, eles podem ser fabricados de maneiras diferentes e, portanto, podem ser membros de famílias diferentes~\citep{singh2012cellular}. Neste sentido, a formação de famílias de produtos aumenta e eficácia ao realizar atividades semelhantes em conjunto e padronizar tarefas semelhantes, além de armazenar e recuperar informações sobre problemas recorrentes~\citep{hyer1984group}. 

O particionamento dos produtos e máquinas em uma fábrica é alcançado aplicando os conceitos de tecnologia de grupo~\citep{choobineh1988framework}. Segundo~\cite{irani1999handbook}, a tecnologia de grupo é uma filosofia de manufatura com ampla aplicabilidade, que tem como objetivo capitalizar atividades que sejam semelhantes e recorrentes, afetando todas as áreas de uma organização. Esse conceito surgiu para reduzir configurações, tamanhos de lote e distâncias de viagem (total de deslocamentos). Vários produtos semelhantes são agrupados e designados sucessivamente em uma máquina com o objetivo de maximizar o uso de uma única configuração, ou para reduzir a configuração necessária para produzir vários produtos. De acordo com~\cite{singh2012cellular}, esse conceito foi ainda ampliado com a coleta de produtos com requisitos de usinagem semelhantes e o processamento completo em um grupo de máquinas (uma célula). 

\subsection{Formação de células}

Segundo~\cite{irani1999handbook}, o processo de identificação de células compreende as seguintes etapas principais. 

\begin{enumerate}
\item[1ª] \textit{Análise de grupo}: para identificar as células de manufatura, a análise de grupo considera as iterações de fluxo entre as máquinas devido às sequências de operações de fabricação dos produtos. As cargas são calculadas para cada família de produtos para obter os requisitos de equipamento para cada célula. Cada célula geralmente contém todo o equipamento (máquinas) necessário para satisfazer os requisitos completos de fabricação da sua família de produtos. Devido a problemas de compartilhamento de equipamentos, alguns fluxos de materiais entre células podem surgir.  
    
\item[2ª] \textit{Comparação de soluções obtidas com o análise de fluxo de produção}: a análise do fluxo de produção é um método para formar famílias de produtos e famílias de máquinas que foi introduzida por~\cite{burbidge1961new} e~\cite{burbridge1963production}. O método analisa os dados do processo de produção que devem estar listados como um roteiro de execução (planilhas de rotas) para manufatura dos produtos na fábrica. A análise de fluxo de produção agrupa os produtos que possuem sequências de operação de fabricação semelhantes. O método precisa de planilhas de rotas confiáveis e bem documentadas. A não confiabilidade delas gera uma desvantagem, porque o método assume a precisão das planilhas de rotas existentes sem considerar se esses planos de processos estão atualizados ou são ótimos em relação à combinação existente de máquinas. Para uma única fábrica, o método clássico de análise de fluxo de produção consiste em quatro estágios alinhados, onde cada estágio trata de uma parte menor de toda a fábrica: análise de fluxo de fábrica, análise de grupo, análise de linha e análise de ferramentas. Seja qual for o particionamento, algumas máquinas são sistematicamente ou frequentemente associadas com os mesmos produtos. Essas semelhanças entre as soluções são aqui identificadas e analisadas. 

\item[3ª] \textit{Processo final de formação de célula}: enquanto as etapas anteriores analisaram e reorganizaram principalmente os fluxos de produção resultantes dos roteiros existentes, nesta etapa são reduzidos os fluxos intercelulares atuando em vários parâmetros do sistema de manufatura. A Figura~\ref{fig:07} ilustra um exemplo de solução final do processo de formação de célula para um único nível, considerando três produtos (PA, PB e PC), três tipos de máquinas (M, D, L) e duas células. Note que existe três cópias da Máquina D (D1, D2, D3), duas cópias da Máquina M (M1, M1A) e duas cópias da Máquina L (L1 e L2). A figura ainda descreve o fluxo de cada um dos produtos. Por exemplo, o Produto PA inicia na Máquina L1, em seguida passa pela Máquina D2, terminando na Máquina L2. Tanto o Produto PA quanto o Produto PC são fabricados na Célula 1, já o Produto PB é produzido na Célula 2.   
\begin{figure}[t]
\begin{center}
\includegraphics[width=9cm]{Celula-formada.png}
\caption{Solução final com duas células independentes. Fonte:~\cite{irani1999handbook}}\label{fig:07}
\end{center}
\end{figure}
    
\end{enumerate}
\subsection{Design do {\it layout} para células de manufatura}

A {\it manufatura celular} envolve o agrupamento de máquinas, processos e pessoas em células responsáveis pela fabricação de produtos ou produtos semelhantes. A manufatura celular tradicional diferencia-se na forma de organizar a maquinaria. De acordo com a ilustração da Figura~\ref{fig:03}, os seguintes três formatos são as maneiras básicas para se organizar as máquinas em uma fábrica.

\begin{enumerate}
\item[a)] {\it Layout de linha}: as máquinas e outros centros de trabalho são organizados na sequência de uso para fabricar o produto. Esse arranjo é adequado para produtos de alto volume e baixa variedade.
\item[b)] {\it Layout de função}: as máquinas do mesmo tipo são agrupadas. Esse layout pode resultar em grandes quantidades de manuseio de materiais, uma grande quantidade de estoque de trabalho em processo, tempos de preparação excessivos e longos tempos de produção.
\item[c)] {\it Layout de célula}: as máquinas são organizadas como células. Cada célula realiza operações de manufatura em uma ou mais famílias de produtos. Consequentemente, a capacidade de uma célula pode ser determinada considerando apenas as famílias de partes que utilizam aquela célula. Como resultado, esse layout deve ser mais fácil de gerenciar. Essa é apenas uma das razões pelas quais um layout celular pode ser mais desejável.
\end{enumerate}

As localizações e disposições das máquinas contribuem em grande medida para a formato como uma empresa está operando. Um layout adequado é importante para uma fábrica, uma vez que os custos de manuseio de materiais compreendem de 30\% a 75\% dos custos totais de fabricação. Qualquer economia no manuseio de materiais obtida por meio de um melhor arranjo das máquinas implica em uma contribuição direta para a melhoria da eficiência geral das operações~\citep{sule2008manufacturing}. 

\begin{figure}[t]
\begin{center}
\includegraphics[width=9cm]{Celula3.png}
\caption{Três tipos de {\it Layouts} de fábrica: a) linha, b) funcionale e c) celular. Fonte:~\cite{irani1999handbook}}\label{fig:03}
\end{center}
\end{figure}


\section{Metaheurísticas}

A complexidade de alguns problemas de interesse prático impossibilita a busca de todas as soluções ou combinações possíveis delas. Com o objetivo de encontrar uma solução viável de boa qualidade e com um tempo de processamento computacional aceitável, foram criadas um conjunto de técnicas chamadas {\it metaheurísticas}.~\cite{kaveh2014advances} definiu uma metaheurística como uma técnica para produzir soluções aceitáveis para problemas complexos em um tempo computacional razonável. Para essas técnicas não há garantia de que a solução ótima possa ser encontrada. No entanto, a ideia é ter um algoritmo eficiente, prático, que funcione na maioria das vezes, e que seja capaz de produzir soluções de boa qualidade. Dentre as soluções de qualidade encontradas, espera-se que algumas delas sejam próximas da solução ótima, embora não se tenha garantia para tal otimização. Duas fases principais de qualquer algoritmo metaheurístico são: diversificação e intensificação. A diversificação significa gerar soluções diversas explorando o espaço de busca global. Enquanto que a intensificação significa focar na busca em uma região local, explorando as informações que uma boa solução atual tem nessa região. Durante essas fases também ocorre a seleção das melhores soluções. 

Os algoritmos metaheurísticos têm muitas aplicações em diferentes áreas da matemática aplicada, engenharia, medicina, economia e outras ciências. Esses métodos são amplamente utilizados na realização de projetos de diferentes sistemas em engenharia civil, mecânica, elétrica e industrial~\citep{kaveh2014advances}. Eles podem ser classificados de várias maneiras, sendo que uma forma é classificá-los baseados em população e trajetória. Por exemplo, os algoritmos genéticos são baseados em população, pois eles são constituídos por população de indivíduos que evoluem ao longo do tempo, ou seja , ao longo das iterações. A evolução ocorre pois cada indivíduo da população é um cromossomo artificial e cada cruzamento entre dois indivíduos ou uma mutação do cromossomo gera novos indivíduos para esta população. Seguindo uma regra semelhante, o algoritmo de otimização por enxame de partículas tem uma população contendo vários agentes ou partículas que evoluem ao longo das iterações. Por outro lado, o algoritmo conhecido por recozimento simulado usa um único agente ou solução que se move numa trajetória de forma inteligente através do espaço de busca em cada iteração. Em cada movimento promissor, uma solução melhor é sempre aceita, enquanto que em um movimento insatisfatório, uma nova solução (inferior às anteriores em termos de qualidade) pode ser aceita ou não dada uma certa probabilidade~\citep{yang2010nature}.  

\subsection{Método GRASP}

O método {\it greedy randomized adaptive search procedure} (GRASP) é uma técnica de busca adaptativa gananciosa com alto grau de aleatoriedade, onde cada iteração fornece uma solução para o problema a ser resolvido. O método GRASP tem sido aplicado a uma ampla gama de problemas de otimização combinatória, desde programação e roteamento até desenho e balanceamento de turbinas~\citep{festa2009annotated}. Cada uma das iterações do GRASP consiste em duas fases: uma fase de construção, na qual é criada de forma inteligente uma solução factível por meio de uma função gananciosa aleatória adaptativa; e uma fase de busca local, a qual visa otimizar/melhorar a solução construída na primeira fase. A melhor solução geral é mantida como resultado de saída. Como é apresentado na Figura~\ref{fig:06}, na fase de construção, uma solução viável é construída iterativamente. A escolha do próximo elemento (fragmento/parte de solução) a ser adicionado na solução que está sendo construída iterativamente é determinada ao ordenar todos esses fragmentos/partes em uma lista de candidatos utilizando um critério gananciosa, a qual mede o benefício de selecionar cada um desses elementos. O componente probabilístico de um GRASP é caracterizado pela escolha aleatória de um dos melhores candidatos da lista, mas não necessariamente o candidato principal.  

A lista dos melhores candidatos é denominada lista restrita de candidatos. A busca local é usada na segunda fase com o objetivo de melhorar a solução obtida na fase de construção, que também opera de forma iterativa substituindo sucessivamente a solução atual por uma solução melhor que está na vizinhança da solução atual. O processo iterativo termina quando nenhuma solução melhor é encontrada na vizinhança. A chave para o sucesso de um algoritmo de busca local consiste na escolha adequada de uma estrutura de vizinhança, no uso de técnicas eficientes de busca na vizinhança e numa boa escolha da solução inicial~\citep{feo1995greedy}.

A estrutura geral de um algoritmo GRASP pode ser resumida nas duas seguintes fases. 
\begin{enumerate}
\item[1ª] \textit{Fase de construção}: nesta fase é construída uma solução inicial factível. O primeiro passo utiliza uma função índice definida de acordo com o problema estudado para gerar uma lista de candidatos a serem incluídos na solução, cada candidato desta lista é uma parte (pedaço) para formar uma solução. Em seguinte, considera-se uma lista restrita de candidatos (LRC), formada pelos melhores candidatos a entrar na solução. Nesse passo é usado um valor $\alpha$ para determinar o nível de ganância ou aleatoriedade na construção. Iterativamente, e de forma promissora, um candidato da LRC é selecionado aleatoriamente para compor a solução~\citep{pitsoulis2002greedy}. Então, esse candidato é excluído da LRC e o processo termina quando a LRC tornar-se vazia.  
    
\item[2ª] \textit{Fase de melhoria}: se faz uma busca local a partir da solução obtida da fase de construção, tal que o processo se repete até que seja cumprido o critério de parada. A eficácia da busca local depende de vários fatores, como a estrutura da vizinhança, a função a ser minimizada e a solução inicial. Diz-se que uma solução $X$ está na região de atração do ótimo global, se a busca local a partir de $X$ leva ao ótimo global. Se a solução encontrada na fase anterior melhora a atual, então a solução é atualizada~\citep{pitsoulis2002greedy}.
\end{enumerate}

\begin{figure}[t]
\begin{center}
\includegraphics[width=5cm]{Celula6.png}
\caption{Diagrama geral do algoritmo GRASP. Fonte:~\cite{nagyoff}}\label{fig:06}
\end{center}
\end{figure}


O método GRASP tem algumas semelhanças com métodos muito conhecidos que podem ser utilizadas para entender melhor as fases do GRASP. Esses métodos são chamados de restauração inexata (RI), que foram introduzidos para resolver problemas de programação não-linear. Cada iteração dos métodos é composta por duas fases. Na primeira (restauração) procura-se encontrar um ponto intermediário com melhor medida de viabilidade, sem piorar muito a medida de otimalidade. Na segunda (minimização) tenta-se melhorar a medida de otimalidade, sem piorar muito a de viabilidade. O progresso em cada iteração é decidido ao se mensurar a solução atual usando funções de méritos ou filtros~\citep{martinez2000inexact}. A viabilidade é a primeira semelhança entre o GRASP e RI. Ela está na Fase 1 de cada método, sejam $X$ uma solução factível que pertence ao conjunto de soluções $S$, e $e$ um elemento (fragmento/parte de solução) pertencente ao conjunto de elementos $E$ que podem ser parte de uma solução $X$, o GRASP faz uma construção de uma solução viável $X$ adicionando nessa solução elementos um a um $e \in E$ até formar uma solução factível, nessa fase sempre se deve manter a viabilidade. Agora no RI, para a Fase 1 ou fase de restauração, dado um ponto $X$ que é uma solução viável do $S$, encontra-se um ponto $Y \in S$ mais viável porém sem piorar muito a otimalidade. Essa fase é realizada resolvendo inexatamente a linearização das restrições em $X$. Portanto, os dois métodos baseiam sua primeira fase em manter a viabilidade das soluções. 

Outra semelhança entre os dois métodos é a otimização na Fase 2. No GRASP, depois de obter uma solução factível na primeira fase, o método faz uma busca local na vizinhança do ponto $X$ tentando otimizar ou minimizar a solução obtida na fase de construção. No RI, constrói-se um modelo quadrático no ponto $Y$ encontrado na Fase 1, depois computa-se a direção de otimização minimizando o modelo quadrático no espaço tangente da superfície de nível das restrições em $Y$. Finalmente, a função de mérito ou filtro é usada para aceitar ou não o ponto $Y$. Note que no GRASP pode ser usado a função objetivo para aceitar uma nova solução se ela for melhor que a atual. Portanto, a segunda fase dos métodos procura otimizar (melhorar) a solução encontrada na primeira fase. 

\subsection{Método da Busca Tabu}

Busca Tabu (BT) é um método desenvolvido para resolver problemas de otimização combinatória. As aplicações incluem desde a teoria dos grafos até problemas gerais de programação inteira pura e mista. De forma geral, a Busca Tabu é um procedimento adaptativo com a capacidade de fazer uso de muitos outros métodos, como algoritmos de programação linear e heurísticas especializadas, que direciona para superar as limitações da otimização local~\citep{glover1989tabu}. O método da Busca Tabu é conhecido por obter soluções aproximadas de boa qualidade e, muitas vezes, a solução ótima, para uma grande variedade de problemas, como programação, telecomunicações, reconhecimento de caracteres, redes neurais, etc. Uma característica importante deste método é que ele pode ser sobreposto sobre outros procedimentos para evitar pontos estacionários~\citep{glover1990tabu}. 

A estrutura básica geral (dividida em fases) e a descrição de cada um dos principais parâmetros para um algoritmo de Busca Tabu pode ser resumida da seguinte forma. 

\begin{figure}[t]
\begin{center}
\includegraphics[width=10cm]{Celula4.png}
\caption{Componente de memória de curto prazo da Busca Tabu. Fonte:~\cite{glover1990tabu}}\label{fig:04}
\end{center}
\end{figure}

\begin{enumerate}
\item[1ª] \textit{Solução inicial}: solução factível usada para iniciar o processo de busca. Pode ser obtida aleatoriamente ou mediante outros métodos heurísticos e metaheurísticos~\citep{glover1990tabu}.

\item[2ª] \textit{Lista de candidatos}: é uma lista criada com cada movimento feito a partir da solução inicial ou atual, ou seja, a vizinhança. O tamanho da lista é definido pelo usuário~\citep{glover1990tabu}. 

\item[3ª] \textit{lista Tabu}: é a memoria que utiliza o procedimento de busca para não voltar as configurações visitadas anteriormente e cair em movimentos cíclicos. Quando ocorrem os movimentos, eles são marcados como tabu e salvos na lista Tabu. Esta lista tem um tamanho pré-definido que é escolhido por um número limite de iterações, em que será proibido os movimentos salvos nela. Essa regra tem uma exceção ao se cumprir o critério de aspiração~\citep{glover1990tabu}. Se o tamanho da lista é menor, o algoritmo concentra em procurar em áreas pequenas, caso contrário, se a lista é maior, o algoritmo explora mais regiões~\citep{bozorgi2015tabu}. 

\item[4ª] \textit{Critério de aspiração}: permite que um movimento seja realizado mesmo se ele estiver classificado como tabu. Isto acontece quando o movimento classificado como tabu pode melhorar a solução atual. Desta maneira, é possível restringir alguns movimentos para evitar cair em um ciclo e sair de ótimos locais para buscar pelo ótimo global~\citep{bozorgi2015tabu}.

\item[5ª] \textit{Critério de parada}: geralmente são utilizados dois critérios de parada para o algoritmo. Um deles refere-se ao número total de iterações definido pelo usuário, enquanto o outro refere-se a quantidade de iterações sem que o algoritmo encontre uma solução melhor~\citep{glover1990tabu}. 
\end{enumerate}

\begin{figure}[t]
\begin{center}
\includegraphics[width=10cm]{Celula5.png}
\caption{Seleção do melhor candidato. Fonte:~\cite{glover1990tabu}}\label{fig:05}
\end{center}
\end{figure}


O algoritmo de Busca Tabu foi definido por~\cite{glover1990tabu}, que está ilustrado na Figura~\ref{fig:04}. O seu processo inicia com a definição da solução inicial obtida aleatoriamente ou por outros métodos e, com base nessa solução, então é criada a lista de movimentos candidatos. O próximo passo, como ilustrado na Figura~\ref{fig:05}, refere-se a escolha do melhor candidato para realizar o movimento. Primeiro é avaliado cada um dos movimentos considerando a função objetivo, então, após escolher o melhor movimento, verifica-se se ele está na lista Tabu. No caso positivo, analisa-se se esse movimento melhora a solução atual, caso positivo novamente o movimento saí da lista Tabu e ele é realizado. Caso contrário, o movimento continua na lista e procura-se pelo segundo melhor movimento. Quando o melhor movimento é escolhido, a lista Tabu é atualizada. Os processos descritos são repetidos até cumprir o critério de parada, que vai depender do número de iterações definidas pelo usuário ao configurar o processo.
