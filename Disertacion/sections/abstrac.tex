%================================= Resumo e Abstract ========================================
\chapter*{Abstract}

\vspace{-1.0cm}
\begin{quotation}

Currently, Revenue Management Systems (RM), refer to a set of techniques used by industries to maximize profit. RM focuses on finding the right products for the right customers at the right time and at a convenient price. The RM methodology was first developed by the airline industry but has been successfully applied in other sectors with similar characteristics, such as hospitality, transportation, retail, restaurants, e-commerce, among others. RM is a challenging and fascinating field that combines topics such as optimization, economics, inferential statistics, and behavioral science.

This study is situated in the transportation sector, specifically passenger rail transportation, where the goal is to determine the optimal number of seats to reserve and make available for sale during the period between the opening of sales to the public and the train's departure.

The main objective of this study was to develop three mixed-integer mathematical models. The first is based on independent demands, while the other two are based on behavioral demands using preference lists. The efficiency of these models was evaluated on 10 real-world instances, classified as large, medium, and small, provided by the Canadian company Expretio. Among the main findings, it stands out that all models achieved optimal results in competitive times, measured in seconds, and in all instances, the optimal solution was obtained by exploring at most one node.

\vspace{0.5cm}

\noindent {\bf Keywords:} Behavioral Demand, Mixed Integer Programming, Passenger Rail Transportation, Mathematical Modeling.

\end{quotation}