%================================= Resumo e Abstract ========================================
\chapter*{Abstract}

\vspace{-1.0cm}
\begin{quotation}

    Currently, revenue management systems (RM), refer to the set of techniques used by industry to maximize profit. RM is responsible for finding the right products for the right customers, at the right time, and at a convenient price. The RM methodology was first developed by the airline industry; however, it has been successfully applied in other sectors with similar characteristics, such as hospitality, transportation, retail, restaurants, e-commerce, among others. RM is a challenging and fascinating field that combines topics such as optimization, economics, inferential statistics, and behavioral science.

    This work is situated within the transportation sector, more specifically in passenger rail transportation, where the objective is to determine the optimal number of seats to reserve and make available for sale during the period between the opening of public ticket sales and the departure of the train.
    
    This problem is complex due to the need to consider train capacity, the hierarchical structure of fare products, ticket reservations, sales availability within the booking horizon, and price consistency over time, among other factors. Furthermore, customer behavior can be modeled in two ways: as independent demand or as behavioral demand. The first approach assumes that customers purchase a specific product independently of the current availability. The second approach, on the other hand, considers that customers choose from a set of available offers based on their preferences. If the most desired option is not available, they move on to the next viable option, provided it is more attractive than not making a purchase.
    
    The main objective of this study was to develop two mixed-integer mathematical models. The first is based on independent demands, and the second is based on behavioral demands using preference lists. The efficiency of these models was evaluated on 10 real-world instances, classified as large, medium, and small, provided by a Canadian company. Among the main findings, it is noteworthy that both models achieved optimal results in competitive times, measured in seconds, and that, in all instances, the optimal solution was found by exploring at most one node.

\vspace{0.5cm}

\noindent {\bf Keywords:} Revenue Management, Behavioral Demand, Passenger Rail Transportation, Mixed Integer Programming.

\end{quotation}