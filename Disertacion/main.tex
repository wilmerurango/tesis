% \documentclass[12pt, openany, oneside]{book} % \documentclass[11pt,fleqn, twoside, final, openany,usenames,dvipsnames]{book}%
\documentclass[12pt, oneside]{book}
\usepackage{changepage} % Para ajustar los márgenes localmente
\usepackage[utf8]{inputenc}
\usepackage[autostyle,german=guillemets,norwegian=quotes]{csquotes}

%=============================== Preâmbulo ============================================
\usepackage{amsfonts}
\usepackage[T1]{fontenc}
\usepackage[portuguese]{babel}
% \usepackage[latin1]{inputenc}		% Inclusão de gráficos
\usepackage[table, xcdraw]{xcolor}% http://ctan.org/pkg/xcolor
\usepackage{indentfirst}
% \usepackage{subfigure}
\usepackage{amssymb,fancyhdr,fancybox,epsfig,psfrag,tabularx}
\usepackage{array}  
\usepackage{comment}
\usepackage{amsthm}
\usepackage{graphicx}
\usepackage{footnote}
\usepackage{float}
\usepackage{hyperref}
\usepackage{amsmath}
\usepackage{pslatex}
\usepackage{algpseudocode}
% \usepackage[ruled,vlined]{algorithm2e}
\usepackage{colortbl}
\definecolor{firebrick}{rgb}{.7, .13, .13}
\definecolor{lightcopper}{rgb}{.93, .76, .58}
\definecolor{blueice}{rgb}{.85, .96, .94}
% \usepackage{pdfpages}
\usepackage{subcaption}
\usepackage{caption}
\usepackage{enumitem} % Paquete para personalizar listas

\usepackage[round]{natbib}%usepackage{harvard}

\hypersetup{
    colorlinks=true,
    citecolor=blue,
    linkcolor=blue,
    filecolor=magenta,      
    urlcolor=cyan,
}

\usepackage{booktabs}
\usepackage{longtable}
\usepackage{multirow}
\usepackage{setspace}
\usepackage{tikz}
\usepackage{soul}
\usepackage{rotating}
% \usetikzlibrary{patterns}
% \usetikzlibrary{plotmarks}
\usetikzlibrary{positioning, arrows.meta, shapes.geometric}

\urlstyle{same}
\renewcommand\qedsymbol{$\blacksquare$}
%\usepackage[paperwidth=8.5in,paperheight=11in,hmargin={25mm,20mm},vmargin={20mm,20mm}]{geometry} %Papel Carta
\usepackage[a4paper,hmargin={25mm,20mm},vmargin={20mm,20mm}]{geometry} %Papel A4


\usepackage[Sonny]{fncychap} %Options: Sonny, Lenny, Glenn, Conny, Rejne, Bjarne, Bjornstrup
%\ChNameVar{\large\sf} \ChNumVar{\Large} \ChTitleVar{\Large\sf\raggedleft}
%\ChRuleWidth{0.24pt} \ChNameUpperCase

\setlength{\headheight}{15pt}
\usepackage[title,titletoc,toc]{appendix}

%\usepackage{proof}
%\newtheorem{lemma}{Lemma}
%\newtheorem{theorem}{Theorem}
%\newtheorem{proposition}{Proposition}
% O comando For, a seguir, retorna 'para #1 -- #2 até #3 faça'
%\algrenewtext{For}[3]%
%{\algorithmicfor\ #1 $\gets$ #2 \algorithmicto\ #3 \algorithmicdo}
%=============================== Cabeçalho e Rodapé ===================================
%\fancyhead{}
%\fancyfoot{}
%\fancyhead[RO]{\thepage}
%\fancyhead[RE]{\sectionmark}
%\fancyhead[LO]{\nouppercase{\rightmark}}
%\fancyhead[LE]{\thepage}

\usepackage{fancyhdr}
\fancyhf{}%limpa todos os campos
\fancyhead[R]{\thepage}
\renewcommand{\headrulewidth}{0pt}
\setlength{\headheight}{15.0pt}
\fancypagestyle{plain}{%
    \fancyhf{}
    \fancyhead[R]{\thepage}
}

\setcounter{secnumdepth}{3} % Numera até subsubsection
% \setcounter{tocdepth}{3}    % Mostra até subsubsection no sumário

\begin{document}

\pagestyle{plain}

%=============================== Capa, Resumo e Sumário ==============================
% %================================================================================================
%================================= PRIMEIRA FOLHA INTERNA  ======================================
%================================================================================================
% \thispagestyle{empty}

% \vspace*{2.0cm}
% \begin{center}
% \large Wilmer Dario Urango Narvaez
% \end{center}
% \vspace*{6.8cm}

% \begin{center}
% {\sc \Large  UM MODELO LINEAR DE DEMANDA COMPORTAMENTAL PARA MAXIMIZAR A RECEITA DO PROBLEMA DE TRANSPORTE FERROVIÁRIO DE PASSAGEIROS}
% \end{center}

% \vspace*{3.25cm}
% \null \vfill

% \begin{center}
% Limeira\\2024
% \end{center}

% \newpage
% \null \vfill

% \newpage
%================================================================================================
%====================================== FOLHA DE ROSTO ==========================================
%================================================================================================
\thispagestyle{empty}

\begin{minipage}{.2\textwidth}
	\includegraphics[width=1.5cm]{img/unicamp.pdf}
\end{minipage}
\hspace*{\fill}
\begin{minipage}{.7\textwidth}
	{\large Universidade Estadual de Campinas\\
		Faculdade de Ciências Aplicadas}
\end{minipage}
\vspace*{1.5cm}

\begin{center}
	\large Wilmer Dario Urango Narvaez
\end{center}
\vspace*{3.0cm}

\begin{center}
	{\sc \Large UM MODELO LINEAR DE DEMANDA COMPORTAMENTAL PARA MAXIMIZAR A RECEITA DO PROBLEMA DE TRANSPORTE FERROVIÁRIO DE PASSAGEIROS}
	\vspace*{1cm}
\end{center}

\begin{center}
	{\sc \Large A LINEAR BEHAVIORAL DEMAND MODEL TO MAXIMIZE REVENUE FROM THE RAIL PASSENGER TRANSPORTATION PROBLEM}
	\vspace*{1cm}
\end{center}


\vspace*{2.0cm}

\begin{flushright}
	\begin{minipage}{9.0cm}
		Dissertação apresentada à Faculdade de Ciências Aplicadas como parte dos
		requisitos exigidos para a obtenção do título de Mestrado em Engenharia de Produção e de Manufatura. Área de
		concentração: Pesquisa Operacional e Gestão de Processos.
	\end{minipage}
\end{flushright}
\vspace*{0.5cm}

\begin{flushleft}
	\begin{minipage}[c]{.5\textwidth}
		\textbf{Orientador: Prof. Diego Jacinto Fiorotto}\\
		Este exemplar corresponde à versão final da dissertação defendida pelo aluno Wilmer Dario Urango Narvaez, e orientado pelo Prof. Diego Jacinto Fiorotto.
	\end{minipage}
\end{flushleft}
\vspace*{0.5cm}

\begin{center}
	Limeira\\2024
\end{center}

%================================================================================================
%============================== Ficha (Somente na versão final) =================================
%================================================================================================
%\newpage
%\thispagestyle{empty}

%\begin{figure}[h!t]
%	\centering
%		\includegraphics[width=1.1\textwidth]{Figuras/ficha.pdf}
%\end{figure} 

%================================================================================================
%============================== Folha de aprovação (Somente na versão final) ====================
%================================================================================================
%\newpage

% Descomente as duas próximas linhas (e comente acima desde o \begin{center} até o \end{center})
%\begin{figure}[h!t]
%	\centering
%		\includegraphics[width=1\textwidth]{Figuras/declaracao.pdf}
%\end{figure} 

% \newpage
% \input{sections/capa2}
% \newpage
% \includepdf[pages=-]{Ficha-Catalografica-Protocolo-975642308}
% \newpage
% \input{Folha-aprovacao}
% \newpage

% \input{Agradecimentos} \thispagestyle{empty}
% %================================= Resumo e Abstract ========================================
\chapter*{Resumo}

\vspace{-1.0cm}
\begin{quotation}
	Na atualidade, os sistemas de administração de receitas, conhecidos como { \it Revenue Management} (RM) em inglês, referem-se ao conjunto de técnicas utilizadas pela indústria para maximizar o lucro. Ele se encarrega de encontrar os produtos apropriados para os clientes certos, no momento correto e a um preço conveniente. A metodologia do RM foi desenvolvida pela primeira vez pela indústria aérea, no entanto, tem sido aplicada com sucesso em outros setores com características semelhantes, como hotelaria, transporte, varejo, restaurantes, comércio eletrônico, entre outros. O RM é uma área desafiadora e interessante que combina temas como otimização, economia, estatística inferencial e ciência do comportamento.

	Este trabalho está inserido no setor de transporte, mais especificamente no transporte ferroviário de passageiros, onde o objetivo é determinar a quantidade ideal de assentos a serem reservados e disponibilizados para venda durante o período compreendido entre a abertura das vendas ao público e a partida do trem.

	O principal objetivo deste estudo foi desenvolver três modelos matemáticos inteiros mistos. O primeiro baseia-se em demandas independentes e os outros dois, em demandas comportamentais utilizando listas de preferência. A eficiência desses modelos foi avaliada em 10 instâncias reais, classificadas como grandes, médias e pequenas, fornecidas pela empresa canadense Expretio. Entre os principais achados, destaca-se que ambos os modelos atingiram resultados ótimos em tempos competitivos, em termos de segundos, além de que, em todas as instâncias, a solução ótima foi encontrada explorando, no máximo um nó.

	\vspace{0.5cm}

	\noindent {\bf Palavras-chaves:} Demanda Comportamental, Programação Inteira Mista, Transporte Ferroviário de Passageiros, Modelagem Matemática.

\end{quotation}

\thispagestyle{empty}
% %================================= Resumo e Abstract ========================================
\chapter*{Abstract}

\vspace{-1.0cm}
\begin{quotation}

Currently, Revenue Management Systems (RM), refer to a set of techniques used by industries to maximize profit. RM focuses on finding the right products for the right customers at the right time and at a convenient price. The RM methodology was first developed by the airline industry but has been successfully applied in other sectors with similar characteristics, such as hospitality, transportation, retail, restaurants, e-commerce, among others. RM is a challenging and fascinating field that combines topics such as optimization, economics, inferential statistics, and behavioral science.

This study is situated in the transportation sector, specifically passenger rail transportation, where the goal is to determine the optimal number of seats to reserve and make available for sale during the period between the opening of sales to the public and the train's departure.

The main objective of this study was to develop three mixed-integer mathematical models. The first is based on independent demands, while the other two are based on behavioral demands using preference lists. The efficiency of these models was evaluated on 10 real-world instances, classified as large, medium, and small, provided by the Canadian company Expretio. Among the main findings, it stands out that all models achieved optimal results in competitive times, measured in seconds, and in all instances, the optimal solution was obtained by exploring at most one node.

\vspace{0.5cm}

\noindent {\bf Keywords:} Behavioral Demand, Mixed Integer Programming, Passenger Rail Transportation, Mathematical Modeling.

\end{quotation}\thispagestyle{empty}


% %=============================== Lista de Tabelas e Figuras ==========================
% \listoffigures
% \addtocontents{lof}{\protect\thispagestyle{plain}}\thispagestyle{empty}

% \listoftables
% \addtocontents{lot}{\protect\thispagestyle{plain}}\thispagestyle{empty}
% \baselineskip 1.1 \baselineskip


% %=============================== Sumário =============================================
\tableofcontents 
\addtocontents{toc}{\protect\thispagestyle{empty}}\thispagestyle{empty} 


% %=============================== Capítulos ===========================================%
\doublespacing %Para um espaçamento duplo

\chapter{Introdução}
% \addcontentsline{toc}{chapter}{1\hspace{1em}Introdução}
% \setcounter{chapter}{1}

\section{Conceitos básicos}
Esta seção apresenta alguns dos termos comumente utilizados na gestão de receitas no transporte ferroviário de passageiros, além de aprofundar-se em conceitos específicos que frequentemente geram confusão entre si.

\begin{description}[style=unboxed, leftmargin=0cm]

	\item[Viagem:] Um trem programado para uma data específica é denominado "viagem". Uma viagem inclui uma estação de partida (estação de origem) e uma estação de chegada (estação de destino final). Exemplo: O trem \#03450-1 de 12 de janeiro de 2024 é uma viagem:

	      \begin{figure}[H]
		      \begin{center}
			      \includegraphics[scale=0.12]{img/viagem.png}
			      \caption{Representação de viagem}
			      \label{fig: viagem}
		      \end{center}
	      \end{figure}
	      \vspace{-1cm}

	\item[Trecho:] Um trecho é uma conexão direta entre duas estações,também pode ser denominado como origem-destino. Além, será dito que um trecho é adjacente se, e somente se, não houver estações intermediárias entre elas; caso contrário, serão não adjacentes. Por exemplo, o trecho AC é ñao adjcente e os trechos AB e BC são adjacentes. É importante aclarar que que os trechos não adjacentes podem conter outros trechos, tanto adjacentes quanto não adjacentes.

	      \begin{figure}[H]
		      \begin{center}
			      \includegraphics[scale=0.12]{img/trecho.png}
			      \caption{Representação de trechos}
			      \label{fig: trecho}
		      \end{center}
	      \end{figure}
	      \vspace{-1cm}

	\item[Itinerário:] Um itinerário é uma combinação única de origem, destino, horário de partida e trem. Um itinerário pode consistir em um ou mais trechos, e uma viagem pode englobar um ou mais itinerários.

	      \begin{figure}[H]
		      \begin{center}
			      \includegraphics[scale=0.12]{img/itinerario.png}
			      \caption{Representação de itinerários}
			      \label{fig: itinerario}
		      \end{center}
	      \end{figure}
	      \vspace{-1cm}

	\item[Classes de controle:] Uma classe de controle é um produto oferecido por um operador de transporte a passageiros potenciais por um preço específico. As classes de controle também são conhecidas como classes tarifárias, produtos tarifários ou classes de reservas. Uma classe de controle define os benefícios e/ou restrições que o passageiro terá como resultado do preço pago pelo bilhete.

	\item[Horizonte de reserva:] É o período de tempo entre o momento em que os bilhetes para o trem em questão são disponibilizados para venda pela primeira vez e a data de partida do trem. O horizonte de reserva geralmente é discretizado por dias, de forma que cada período representa um dia específico antes da partida (DBD). Frequentemente, realizamos uma agregação temporal para reduzir o tamanho do horizonte de reserva, selecionando alguns DBD específicos como pontos de controle (CP). Nesse contexto, o horizonte de contabilização é discretizado por períodos, onde o tamanho de cada período corresponde ao intervalo de tempo entre dois CP consecutivos.

	      \begin{figure}[H]
		      \begin{center}
			      \includegraphics[scale=0.53]{img/h_reserva.png}
			      \caption{Representação do horizonte de reserva}
			      \label{fig: h_reserva}
		      \end{center}
	      \end{figure}
	      \vspace{-1cm}

	\item[Reservas:] As reservas representam o número de assentos protegidos para atender à demanda potencial de uma classe de controle específica em um determinado itinerário, durante um período específico do horizonte de reserva. O tamanho da reserva é definido no processo de otimização, considerando a demanda potencial do itinerário e da classe de controle correspondente. As reservas também são conhecidas como níveis de proteção.

	\item[Autorizações:] As autorizações representam o número de assentos disponíveis para venda em um determinado itinerário e classe de controle. Elas podem ser entendidas como a quantidade de bilhetes apresentados aos passageiros como disponíveis para compra. As autorizações têm como objetivo controlar o volume de passageiros ao longo de um itinerário ou trecho.
	
	As Autorizações podem ser classificadas como estáticas ou dinâmicas. As Autorizações estáticas não estão indexadas no tempo, enquanto as dinâmicas estão. Para uma melhor compreensão, observe a Figura \ref{fig: classAuto}, onde, no caso de uma classe de controle $A_1$, ela assume um valor só para todo o horizonte de reserva no caso de autorização estática, e toma valores distintos em função de cada período para as autorizações dinâmicas.

	\begin{figure}[H]
		\begin{center}
			\includegraphics[scale=0.25]{img/classAuto.png}
			\caption{Classes de autorização}
			\label{fig: classAuto}
		\end{center}
	\end{figure}
	\vspace{-1cm}

	\item[Demanda comportamental:] Neste modelo, assume-se que o comportamento de compra dos passageiros é influenciado pelo conjunto de classes de controle disponíveis. Um exemplo desse conceito é ilustrado na Figura \ref{fig: dc1}. Considere um cenário em que 4 clientes desejam comprar a tarifa A3, sendo essa sua primeira opção. Caso a tarifa A3 não esteja disponível, um dos clientes desiste da compra, enquanto os outros 3 permanecem no sistema e estão dispostos a adquirir a tarifa A2. Se A2 também não estiver disponível, mais 2 clientes abandonam a compra, restando apenas um cliente que opta por A1 (geralmente, a opção de maior valor).

	      \begin{figure}[H]
		      \begin{center}
			      \includegraphics[scale=0.09]{img/dc1.png}
			      \caption{Representação de demanda comportamental}
			      \label{fig: dc1}
		      \end{center}
	      \end{figure}
	      \vspace{-1cm}

	\item[Demanda independente:] Os clientes compram um produto específico, independentemente da oferta disponível no momento. Em um modelo de demanda independente, assume-se que um cliente disposto a pagar R\$100 por um bilhete nunca optaria por um bilhete mais barato (por exemplo, R\$50), mesmo que este estivesse disponível.

	      \begin{figure}[H]
		      \begin{center}
			      \includegraphics[scale=0.7]{img/di1.png}
			      \caption{Representação de demanda independente}
			      \label{fig: di1}
		      \end{center}
	      \end{figure}
	      \vspace{-1cm}

	\item[Skip lagging:] No contexto do transporte ferroviário de passageiros, skip lagging refere-se a uma estratégia de otimização na programação das paradas dos trens, na qual certos serviços omitem paradas intermediárias em determinados trechos. O objetivo é reduzir os tempos de viagem e aumentar a eficiência operacional, garantindo que essa omissão não comprometa a acessibilidade e a qualidade do serviço oferecido aos passageiros. Na seção de modelagem matemática, será apresentado um conjunto de restrições associadas a esse conceito.

	\item[Fulfillments over periods:] No contexto do transporte ferroviário de passageiros, fulfillments over periods refere-se ao cumprimento de determinados níveis de serviço, demanda ou capacidade ao longo de um período de tempo definido. Esse conceito é especialmente relevante na gestão de receitas, planejamento operacional e alocação de recursos, onde é essencial garantir que a oferta de serviços ferroviários atenda a objetivos estratégicos ao longo do tempo, em vez de se concentrar apenas em momentos específicos. Na seção de modelagem matemática, será apresentado um conjunto de restrições associadas a esse conceito.

	\item[Listas de preferência:] Representam uma hierarquia de alternativas de compra que um passageiro está disposto a considerar no momento da aquisição de uma passagem ferroviária, baseando-se, principalmente, nos preços ofertados nesse momento. Essas listas são fundamentais para entender o comportamento do consumidor e otimizar as estratégias de preços e disponibilidade de assentos. Elas podem ser utilizadas para modelar a demanda de forma mais precisa, levando em conta as preferências dos passageiros e suas reações a diferentes ofertas tarifárias.
\end{description}

\section{Visão geral da pesquisa}

Os problemas relacionados à gestão de receitas (Revenue Management - RM) podem ser definidos como estratégias que visam maximizar as receitas ajustando-se preços e a disponibilidade de capacidade com base na demanda \citep{Gallego1994}. A área de RM foi inicialmente desenvolvida pela indústria aérea nas décadas de 70 e 80, quando as companhias aéreas começaram a adotar técnicas de otimização para definir os preços dos bilhetes com base na disponibilidade de assentos e na antecedência das reservas \citep{article_base}. O objetivo principal do RM consiste em otimizar a disponibilidade e o preço dos produtos para gerar a maior quantidade de receita possível. Entre suas principais características estão a segmentação de clientes, o controle de capacidade, o ajuste dinâmico de preços e o uso de previsões para estimar a demanda. O sucesso alcançado no setor aeronáutico foi tão expressivo que o RM foi expandido para outros setores com características semelhantes, como hotelaria, restaurantes, varejo, comércio eletrônico e transporte \citep{HEO2009446}.

A otimização de receitas no transporte ferroviário de passageiros tornou-se uma área de pesquisa essencial para aumentar a sustentabilidade e a competitividade do setor. Esse tipo de transporte enfrenta desafios específicos na gestão de sua capacidade devido à variabilidade da demanda, à rigidez tarifária e à necessidade crescente de adaptar suas estratégias às dinâmicas de mercado \citep{Guerriero2021}. Nesse contexto, o RM oferece uma estrutura robusta para abordar a alocação ideal de recursos, utilizando técnicas avançadas de modelagem matemática e análise de dados \citep{Ammirato2020}.

O principal objetivo do RM no transporte ferroviário de passageiros pode ser exemplificado da seguinte forma: um operador de transporte define o itinerário de um trem específico, detalhando a origem, o destino e o horário de partida. Os clientes, ou seja, os potenciais passageiros, podem adquirir bilhetes antecipadamente para viajar nesse trem. Chamamos de classes tarifárias ou produtos tarifários os bilhetes disponíveis para venda. O horizonte de reserva é o período de tempo entre a disponibilização inicial dos bilhetes para compra e a partida do trem. Esse horizonte geralmente é dividido em dias, de forma que cada período representa um dia (ou conjunto de dias) antes da partida.

A função do RM, nesse contexto, é controlar a disponibilidade dos produtos tarifários ao longo do horizonte de reserva para maximizar a receita total. Mais especificamente, o processo de otimização busca determinar a quantidade de bilhetes de cada produto que deve estar disponível para venda em cada período, com o objetivo de maximizar os lucros associados a cada partida de trem.

Um elemento fundamental para maximizar as receitas é um modelo preciso da demanda. Alinhar oferta e demanda para otimizar os lucros requer uma compreensão aprofundada do comportamento dos clientes e uma previsão confiável de suas decisões diante de diferentes ofertas de produtos \citep{ZHAO2019776}.

Uma simplificação comum na modelagem da demanda para gestão de receitas é assumir que os clientes têm um comportamento de compra independente. Isso significa que cada cliente compra um produto específico, desconsiderando a oferta disponível no momento. Na prática, por exemplo, um cliente que desejasse comprar um bilhete de trem por R\$10 não pagaria R\$10,50 se essa fosse a única oferta disponível. Em vez disso, ele optaria por não viajar. Da mesma forma, em um modelo de demanda independente, assumimos que um cliente disposto a pagar R\$100 por um bilhete nunca compraria um bilhete mais barato (por exemplo, R\$50), mesmo que estivesse disponível.

Por outro lado, uma abordagem mais robusta considera o comportamento de compra baseado em faixas ou listas de preferência. Esse modelo assume que os clientes escolhem entre um conjunto de ofertas com base em suas preferências. Se a opção mais desejada não estiver disponível, eles passam para a próxima opção viável, desde que esta seja mais atrativa do que não realizar a compra.

Para abordar essa problemática, foram desenvolvidos três modelos de Programação Inteira Mista (MIP): O primeiro baseado em demanda independente. O segundo em demandas comportamentais ajustadas por proporções. Y o terceiro em demandas comportamentais ajustadas por hierarquia.

Esses modelos respeitam as restrições operacionais do sistema ferroviário, como capacidade dos trens, estrutura hierárquica dos produtos tarifários, reservas de bilhetes, disponibilidade de vendas dentro do horizonte de reserva e coerência dos preços dos bilhetes ao longo do tempo, entre outras.

O uso de instâncias reais fornecidas por uma empresa especializada permitiu validar a aplicabilidade dos modelos desenvolvidos, destacando tanto sua capacidade de capturar a complexidade do mercado quanto sua eficiência computacional. Os resultados preliminares indicam que os modelos baseados em demandas comportamentais geraram soluções de melhor qualidade em comparação com o modelo de demanda independente. No entanto, em termos do valor da função objetivo, os modelos comportamentais apresentaram os mesmos resultados entre si, que foram ligeiramente diferentes dos obtidos pelo modelo independente.

Essa pesquisa contribui para o corpo de conhecimento em Revenue Management ao combinar técnicas de modelagem baseadas em demandas comportamentais, utilizando listas de preferência, e programação matemática para resolver problemas complexos de alocação de assentos no transporte ferroviário de passageiros.
% \end{quotation}
% \input{Fundamentacao-teorica}
\chapter{Revisão da literatura}

\section{Origem do Revenue Management (RM)}

Até o ano de 1978, a Junta de Aeronáutica Civil (CAB em inglês) limitava a concorrência entre as companhias aéreas, onde basicamente as companhias só podiam competir oferecendo serviços como refeições luxuosas e alta frequência nos horários de saída dos voos. Nesse ponto, a CAB não permitia que fosse oferecida uma tarifa menor para um voo, se esta fosse antieconômica para a indústria como um todo. Assim, mesmo que para uma companhia aérea fosse rentável colocar um valor baixo para uma passagem em comparação com outra, a CAB não permitiria, a menos que houvesse uma justificativa extremamente sólida. Quando esse tipo de situação ocorria, o restante das companhias aéreas justificava que o público seria prejudicado, pois elas teriam que aumentar o valor das passagens em outras rotas para compensar o baixo custo da nova proposta do concorrente.

Com a chegada da desregulamentação, as companhias aéreas se depararam com um mundo cheio de novas formas de concorrência, onde o preço das passagens se tornou prioritário. E foi nesse momento que iniciou a verdadeira concorrência entre as transportadoras. Aqui surgiu um novo problema em função da diversidade de preços com diferentes restrições que limitam a disponibilidade de assentos a tarifas mais baixas, a presença de múltiplos voos operados por diversas companhias aéreas em diferentes rotas, e a variabilidade na demanda por assentos em função de fatores como a temporada, o dia da semana, a hora do dia e a qualidade do serviço oferecido, o que influencia a escolha dos passageiros entre diferentes opções de voo.

Nesse momento, esse problema foi denominado como problema de preços e combinação de passageiros e foi modelado como: cada passageiro em um voo representa um custo de oportunidade, já que sua ocupação de um assento impede que outro passageiro com um itinerário mais rentável ou uma classe de tarifa mais alta o utilize. Isso se traduz na possibilidade de assentos vazios em diferentes segmentos de voo, o que afeta a eficiência da rede da companhia aérea ao considerar múltiplos passageiros com diversas origens, destinos e classes de tarifas.

Houve dois possíveis resultados: 1) a otimização da combinação de passageiros permite que as companhias aéreas estruturem de maneira mais eficaz seu sistema de reservas, estabelecendo limites e prioridades adequadas para o número de passageiros com diferentes classes de tarifas em distintos voos. 2) Além disso, possibilita a avaliação de diversos cenários de preço e rota, considerando o benefício gerado a partir da melhor combinação de passageiros em relação a um cenário específico.

Ao ajustar a estrutura das classes de tarifas, as companhias aéreas buscam gerenciar o deslocamento de passageiros por meio de estratégias de preços e a aplicação de restrições como horários, duração da estadia e tempo de antecedência à saída do voo. Além disso, buscam reduzir o deslocamento controlando a capacidade, determinando a quantidade de assentos atribuídos a cada classe de tarifa em cada segmento de voo.

Por outro lado, a otimização da combinação de passageiros é formulada como: "Dada a previsão diária da demanda de passageiros nas diferentes classes de tarifas, qual combinação de passageiros e classes de tarifas em cada segmento de voo maximizará as receitas do dia?" Essa resposta ajuda a companhia aérea a determinar a alocação ideal de reservas entre as diversas classes de tarifas em cada segmento de voo \cite{article_base}. 

Essas últimas duas definições foram conhecidas como Yield Management e, posteriormente, com a chegada de novos sistemas de informação, regras de controle e outras condições, foram generalizadas e aplicadas em outras indústrias de características semelhantes, que no futuro seriam chamadas de Revenue Management \cite{article_YM_to_RM}.

Según \cite{article_Ryzin2014}, a gestão de receitas (RM) abrange o conjunto de estratégias e táticas que as empresas utilizam para gerenciar de forma científica a demanda por seus produtos e serviços. Além disso, pode-se dizer que, seu objetivo é vender cada unidade de ações para o cliente certo, no momento e pelo preço corretos \cite{doi:10.1080/02642069.2010.491543}.

A princípio, os problemas de gestão de RM parecem ser simples; no entanto, nada poderia estar mais longe da realidade. Esses problemas têm uma complexidade esmagadora, e este documento não seria suficiente para detalhar cada um deles, apenas para mencionar alguns, temos modelagem, análise teórica, implementação, previsão, vendas excessivas, controle de estoque de assentos, preços, etc. Então, como sempre acontece, trabalha-se com simplificações de fatores muito complexos e com aproximações em outros casos \cite{doi:10.1287/trsc.33.2.233}.


\section{Modelagem do Transporte Ferroviário de Passageiros}
Nos últimos anos, a gestão de reservas de bilhetes no transporte ferroviário de passageiros evoluiu significativamente graças à aplicação de modelos matemáticos avançados. Esses modelos têm como objetivo otimizar a alocação de assentos, o planejamento de paradas e as estratégias de preços, buscando maximizar as receitas e melhorar a satisfação dos passageiros. A seguir, são descritos alguns dos enfoques mais relevantes utilizados nesse campo, com base em pesquisas recentes.

Um dos modelos destacados é o desenvolvido por \cite{zhou2023pricing}, que integra a teoria das perspectivas, o modelo logit e um modelo de transferência de fluxo de passageiros para alocar a demanda de maneira eficaz. Esse enfoque permite estabelecer preços diferenciados e distribuir os assentos de forma a maximizar as receitas, considerando as preferências e comportamentos de diferentes segmentos de passageiros.

Por outro lado, o planejamento das paradas dos trens e a estratégia de preços estão interligados e impactam tanto as receitas quanto a experiência dos passageiros. Em \cite{zhou2022nonlinear} propuseram um modelo de otimização não linear inteiro misto que aborda conjuntamente a estratégia de preços dos bilhetes e o planejamento das paradas. Esse modelo busca maximizar as receitas do transporte ferroviário e minimizar o tempo de viagem dos passageiros, alcançando um equilíbrio eficiente entre oferta e demanda.

Além disso, a demanda de passageiros está sujeita a incertezas que tornam a gestão operacional mais desafiadora. Han e Ren em 2020 desenvolveram um modelo que otimiza conjuntamente o planejamento de paradas e a alocação de bilhetes, utilizando a teoria da incerteza. Esse modelo busca maximizar a satisfação dos passageiros e a taxa média de ocupação dos assentos, oferecendo soluções robustas frente às flutuações na demanda \cite{han2020uncertainty}.

Posteriormente, em \cite{schoebel2021tariff}, investiga-se como a escolha da rota dos passageiros é influenciada pelas estruturas tarifárias e pelos preços dos bilhetes. Sua pesquisa abordou o problema de determinar a tarifa mais econômica em sistemas de transporte público, avaliando diferentes estruturas tarifárias, como aquelas baseadas em distância ou zonas, e propondo algoritmos altamente eficientes para resolver esses problemas.
\chapter{Metodologia}

\section{Instâncias}

Para verificar as características dos modelos propostos, foram utilizadas 10 instâncias reais fornecidas por uma empresa canadense parceira. As características de cada uma dessas instâncias estão apresentadas na Tabela \ref{tab: instancias}.

\begin{table}[H]
	\centering
	\begin{tabular}{ccccccc}
		\toprule
		\textbf{Instância}                                                  &
		\textbf{\begin{tabular}[c]{@{}c@{}}Nome \\ Trem\end{tabular}}       &
		\textbf{\begin{tabular}[c]{@{}c@{}}Capacidade \\ Trem\end{tabular}} &
		\textbf{Data Partida}                                               &
		\textbf{\# Trechos}                                                 &
		\textbf{\# Períodos}                                                &
		\textbf{\# Classes}                                                                                          \\
		\midrule
		Inst1                                                               & 68 & 561 & 2023-11-21 & 256 & 120 & 15 \\
		Inst2                                                               & 13 & 637 & 2023-11-26 & 252 & 146 & 15 \\
		Inst3                                                               & 71 & 563 & 2023-10-13 & 250 & 105 & 16 \\
		Inst4                                                               & 71 & 561 & 2023-11-21 & 241 & 124 & 15 \\\midrule
		Inst5                                                               & 8  & 561 & 2022-12-04 & 152 & 116 & 17 \\
		Inst6                                                               & 40 & 565 & 2023-11-21 & 149 & 68  & 16 \\
		Inst7                                                               & 74 & 491 & 2023-07-25 & 136 & 70  & 16 \\\midrule
		Inst8                                                               & 72 & 493 & 2023-10-25 & 100 & 85  & 16 \\
		Inst9                                                               & 45 & 563 & 2023-03-16 & 90  & 73  & 16 \\
		Inst10                                                              & 15 & 493 & 2023-08-07 & 50  & 60  & 12 \\
		\bottomrule
	\end{tabular}
	\caption{Resumo das instâncias utilizadas no experimento}
	\label{tab: instancias}
\end{table}

A coluna "\# Trechos" \, representa a quantidade de trechos que a viagem do trem possui em cada instância. A coluna "\# Períodos" \, indica a quantidade de períodos considerados na programação desse trajeto dentro do seu horizonte de reserva. Por fim, a coluna "\# Classes" \, mostra o número máximo de classes disponíveis para cada viagem de cada trem; Observe que, para cada trecho, podem estar disponíveis todas as classes indicadas na coluna correspondente ou uma quantidade menor.

Por outro lado, estabeleceu-se que as instâncias compreendidas entre a instância Inst1 e a instância Inst4 foram classificadas como grandes, as instâncias entre a instância Inst5 e a instância Inst7 como médias, e as demais foram classificadas como pequenas.

\section{Modelos Matemáticos}
Foram abordados dois enfoques distintos para o desenho dos modelos. O primeiro enfoque, baseado nos tipos de demanda, abrange a demanda independente e a demanda comportamental. O segundo enfoque trata da indexação das passagens autorizadas para venda, sendo classificado em estático e dinâmico. No enfoque estático, a quantidade de assentos a serem autorizados é calculada para todo o horizonte de reserva de uma vez, sem considerar variações temporais. Já no enfoque dinâmico, os assentos autorizados são calculados por períodos dentro do horizonte de reserva, levando em conta a evolução da demanda ao longo do tempo.

Os dois enfoques resultaram em quatro modelos distintos, conforme ilustrado na Figura X. No entanto, os autores desta pesquisa buscaram analisar como a solução de cada modelo variava ao adicionar ou remover os conjuntos de restrições específicos (fulfillment over periods e skip lagging). Para isso, cada modelo foi inicialmente dividido em uma versão básica, à qual foram incorporados, de forma separada, os conjuntos de restrições. Por fim, obteve-se o modelo completo, que inclui todos os conjuntos de restrições. Essa classificação pode ser visualizada na Tabela X.


\begin{table}[H]
\centering
\small
\renewcommand{\arraystretch}{1.2}
\begin{tabularx}{\textwidth}{
  @{} 
  >{\raggedright\arraybackslash}p{2.5cm} % Tipo de Autorização
  >{\raggedright\arraybackslash}p{2.5cm} % Tipo de Demanda
  >{\raggedright\arraybackslash}X        % Modelo
  >{\raggedright\arraybackslash}X        % Nome do Modelo
  @{}
}
\toprule
\textbf{Tipo de Autorização} & \textbf{Tipo de Demanda} & \textbf{Modelo}                         & \textbf{Nome do Modelo}                \\
\midrule
\multirow{8}{=}{Estática}
  & \multirow{4}{=}{Independente}
    & Modelo Básico                            & \texttt{Model\_Basic\_Independ\_Est}      \\
  & 
    & Modelo \emph{fulfillments over periods}  & \texttt{Model\_Fulfill\_Independ\_Est}    \\
  & 
    & Modelo \emph{skip lagging}               & \texttt{Model\_Skipla\_Independ\_Est}     \\
  & 
    & Modelo Completo                          & \texttt{Model\_Comp\_Independ\_Est}       \\
\cmidrule(lr){2-2}\cmidrule(lr){3-4}
  & \multirow{4}{=}{Comportamental}
    & Modelo Básico                            & \texttt{Model\_Basic\_Comporta\_Est}      \\
  & 
    & Modelo \emph{fulfillments over periods}  & \texttt{Model\_Fulfill\_Comporta\_Est}    \\
  & 
    & Modelo \emph{skip lagging}               & \texttt{Model\_Skipla\_Comporta\_Est}     \\
  & 
    & Modelo Completo                          & \texttt{Model\_Comp\_Comporta\_Est}       \\
\midrule
\multirow{8}{=}{Dinâmica}
  & \multirow{4}{=}{Independente}
    & Modelo Básico                            & \texttt{Model\_Basic\_Independ\_Din}      \\
  & 
    & Modelo \emph{fulfillments over periods}  & \texttt{Model\_Fulfill\_Independ\_Din}    \\
  & 
    & Modelo \emph{skip lagging}               & \texttt{Model\_Skipla\_Independ\_Din}     \\
  & 
    & Modelo Completo                          & \texttt{Model\_Comp\_Independ\_Din}       \\
\cmidrule(lr){2-2}\cmidrule(lr){3-4}
  & \multirow{4}{=}{Comportamental}
    & Modelo Básico                            & \texttt{Model\_Basic\_Comporta\_Din}      \\
  & 
    & Modelo \emph{fulfillments over periods}  & \texttt{Model\_Fulfill\_Comporta\_Din}    \\
  & 
    & Modelo \emph{skip lagging}               & \texttt{Model\_Skipla\_Comporta\_Din}     \\
  & 
    & Modelo Completo                          & \texttt{Model\_Comp\_Comporta\_Din}       \\
\bottomrule
\end{tabularx}
\end{table}







% \begin{table}[H]
% 	\begin{tabular}{@{}ccll@{}}
% 		\toprule
% 		\multicolumn{1}{l}{\textbf{Tipo de Demanda}} & \multicolumn{1}{l}{\textbf{Tipo de autorizacao}} & \textbf{Modelo}                  & \textbf{Nome do Modelo}       \\ \midrule
% 		\multirow{8}{*}{Independente}                & \multirow{4}{*}{Estática}                        & Modelo Básico                    & Model\_Basic\_Independ\_Est   \\
% 		                                             &                                                  & Modelo fulfillments over periods & Model\_Fulfill\_Independ\_Est \\
% 		                                             &                                                  & Modelo skip lagging              & Model\_Skipla\_Independ\_Est  \\
% 		                                             &                                                  & Modelo Completo                  & Model\_Comp\_Independ\_Est    \\ \cmidrule(l){2-4}
% 		                                             & \multirow{4}{*}{Dinâmica}                        & Modelo Básico                    & Model\_Basic\_Independ\_Din   \\
% 		                                             &                                                  & Modeo fulfillments over periods  & Model\_Fulfill\_Independ\_Din \\
% 		                                             &                                                  & Modelo skip lagging              & Model\_Skipla\_Independ\_Din  \\
% 		                                             &                                                  & Modelo Completo                  & Model\_Comp\_Independ\_Din    \\ \midrule
% 		\multirow{8}{*}{Comportamental}              & \multirow{4}{*}{Estática}                        & Modelo Básico                    & Model\_Basic\_Comporta\_Est   \\
% 		                                             &                                                  & Modelo fulfillments over periods & Model\_Fulfill\_Comporta\_Est \\
% 		                                             &                                                  & Modelo skip lagging              & Model\_Skipla\_Comporta\_Est  \\
% 		                                             &                                                  & Modelo Completo                  & Model\_Comp\_Comporta\_Est    \\ \cmidrule(l){2-4}
% 		                                             & \multirow{4}{*}{Dinâmica}                        & Modelo Básico                    & Model\_Basic\_Comporta\_Din   \\
% 		                                             &                                                  & Modelo fulfillments over periods & Model\_Fulfill\_Comporta\_Din \\
% 		                                             &                                                  & Modelo skip lagging              & Model\_Skipla\_Comporta\_Din  \\
% 		                                             &                                                  & Modelo Completo                  & Model\_Comp\_Comporta\_Din    \\ \bottomrule
% 	\end{tabular}
% \end{table}



% Embora três modelos matemáticos tenham sido propostos — a formulação independente e as formulações comportamentais —, ao resolver as instâncias decidiu-se inicialmente testar os modelos sem as restrições de fulfillment nem as restrições de skiplagging. Em seguida, cada modelo foi testado separadamente com cada grupo de restrições. Por fim, ambos os modelos foram avaliados considerando os dois conjuntos de restrições simultaneamente. Essa abordagem foi adotada para observar como a solução evoluía ou se comportava ao incluir cada tipo de restrição no modelo. Dito isso, a seguir são apresentados os modelos com suas respectivas descrições.
\vspace{0.5cm}

% \begin{small}
% 	\begin{longtable}{p{5.4cm} p{10.4cm}}
% 		\hline
% 		\textbf{Tipo de Modelo}            & \textbf{Descrição}                                                                                                                       \\ \hline
% 		BaseModel                          & Modelo base independente sem as restrições de Fulfillments nem as restrições de Skiplagging.                                             \\ \hline
% 		BaseModelFulfillments              & Modelo base independente com as restrições de Fulfillments.                                                                              \\ \hline
% 		BaseModelSkiplagging               & Modelo base independente com as restrições de Skiplagging.                                                                               \\ \hline
% 		BaseModelFull                      & Modelo independente completo com os dois conjuntos de restrições.                                                                        \\ \hline
% 		HierarBehavioralModel              & Modelo base comportamental com ajuste de demanda do tipo hierarquia, sem as restrições de Fulfillments nem as restrições de Skiplagging. \\ \hline
% 		HierarBehavioralModelFulfillments  & Modelo base comportamental com ajuste de demanda do tipo hierarquia, com as restrições de Fulfillments.                                  \\ \hline
% 		HierarBehavioralModelSkiplagging   & Modelo base comportamental com ajuste de demanda do tipo hierarquia, com as restrições de Skiplagging.                                   \\ \hline
% 		HierarBehavioralModelFull          & Modelo comportamental completo com ajuste de demanda do tipo hierarquia, com os dois conjuntos de restrições.                            \\ \hline
% 		PercentBehavioralModel             & Modelo base comportamental com ajuste de demanda do tipo proporção, sem as restrições de Fulfillments nem as restrições de Skiplagging.  \\ \hline
% 		PercentBehavioralModelFulfillments & Modelo base comportamental com ajuste de demanda do tipo proporção, com as restrições de Fulfillments.                                   \\ \hline
% 		PercentBehavioralModelSkiplagging  & Modelo base comportamental com ajuste de demanda do tipo proporção, com as restrições de Skiplagging.                                    \\ \hline
% 		PercentBehavioralModelFull         & Modelo comportamental completo com ajuste de demanda do tipo proporção, com os dois conjuntos de restrições.                             \\ \hline
% 		\caption{Descrição dos modelos matemáticos propostos}
% 		\label{tab:modelos}
% 	\end{longtable}
% \end{small}


\section{Ferramentas e ambiente computacional}
O experimento foi conduzido em um computador ASUS, com sistema operacional Windows 11 Pro de 64 bits. O equipamento conta com um processador Intel(R) Core(TM) i9-13900KF (32 CPUs) de 13ª geração, com frequência de ~3.0 GHz e arquitetura x64, 128 GB de memória RAM e tarjeta grafica  NVIDIA GeForce RTX 4070 Ti de 16GB de ram.

Para o desenvolvimento e execução do experimento, foi utilizada a linguagem de programação Python na versão 3.12.7, juntamente com o solver Gurobi, versão 12.0.2.



\chapter{Modelagem matemática}

\section{Modelagem matemática}

Dentro da pesquisa, foram realizadas duas modelagens distintas, ambas, diferentemente da literatura clássica, foram elaboradas com base nas estações de origem e destino, e não nos legs do percurso.

Para compreender essas propostas, consideremos uma versão simplificada do problema como se mostra na figura \ref{fig: fig1}, onde temos:

\begin{itemize}
	\item 4 estações pelas quais o trem deve passar em um único sentido, ou seja, o trem não tem retorno.
	\item O trem tem uma capacidade máxima de assentos.
	\item Há apenas um tipo de classe comercial.
	\item Existe apenas um período no horizonte de reserva.
	\item A variável de decisão é a quantidade de assentos que pode ser disponibilizada para venda em um trecho com origem e destino específicos.
	\item Todos os assentos disponibilizados para venda serão vendidos.
\end{itemize}

\begin{figure}[th]
	\begin{center}
		\includegraphics[scale=0.18]{img/repre_ini1.png}
		\caption{Versão gráfica simples}
		% Fonte:~\cite{khaksar2013genetic}}
		\label{fig: fig1}
	\end{center}
\end{figure}


\section{Primeira modelagem matemática}\label{sec:modelo1}

Agora, para a primeira proposta de modelagem, temos o seguinte

\noindent $x_{ij}$: Quantidade de assentos que serão seguradas no trecho com origem em $i$ e destino em $j$, onde $j>i$ (variavel de decisão). \\
\noindent $A_i$: Quantidade de assentos vagos na estação $i$. \\
\noindent $P_{ij}$: Preço da passagem no trajeto com origem em $i$ e destino em $j$. \\
\noindent $Q$: Capacidade física do trem.

Dado o exposto, a função objetivo será maximizar o lucro para cada possível venda em cada trajeto $i,j$, matematicamente seria:

$FO: max \quad x_{12}P_{12} + x_{13}P_{13} + x_{14}P_{14} + x_{23}P_{23} + x_{24}P_{24} + x_{34}P_{34}$

s.a.

Estação 1: $x_{12} + x_{13} + x_{14} \leq A_1 \quad onde \quad A_1 = Q $ \\
\indent Estação 2: $x_{23} + x_{24}  \leq  A_2 \quad onde \quad A_2 = A_1 - (x_{12} + x_{13} + x_{14}) + x_{12} $ \\
\indent Estação 3: $x_{34} \leq A_3 \quad onde \quad A_3 = A_2 - (x_{23} + x_{24}) + x_{13} + x_{23} $

Note que as restrições são aplicadas para cada uma das três primeiras estações, E1, E2 e E3, já que são as estações que têm pelo menos um destino, e a última estação, E4, é excluída, pois não possui nenhum destino.

Cada uma das restrições leva em consideração o fluxo de pessoas que sairão e entrarão no trem. Levando isso em conta, é necessário calcular a disponibilidade do trem para cada estação. Considere uma solução viável para o modelo, conforme mostrado na figura \ref{fig: fig2}, com uma capacidade total de 100 assentos para um trem.

\begin{figure}[!ht]
	\begin{center}
		\includegraphics[scale=0.4]{img/fig2.png}
		\caption{Solução factível para o problema simplificado}
		% Fonte:~\cite{khaksar2013genetic}}
		\label{fig: fig2}
	\end{center}
\end{figure}

Note que, para a restrição da estação 1, o trem está com todos os assentos vazios, ou seja, \(A_1=100\), e que a soma das variáveis seria \(x_{12} + x_{13} + x_{14} = 10 + 5 + 15 = 30\). Portanto, teríamos \(30 \leq 100\), ou seja, foram disponibilizados para venda 30 assentos dos 100 que o trem possui. Nesse sentido, no momento da partida do trem da estação 1, haveria 70 assentos vazios ou disponíveis para venda em estações posteriores.

Agora, para a estação 2, teríamos \(A_2 = 100 - 30 + 10 = 70 + 10 = 80\). Já era conhecido que havia 70 assentos disponíveis vindos da estação 1, mas também é preciso levar em conta que os assentos com destino à estação 2 também ficarão disponíveis da estação 2 em diante, para este caso \(x_{12} = 10\). Portanto, para a estação 2, teríamos 80 assentos vazios para disponibilizar, ou seja, \(60 \leq 80\). Analogamente, o mesmo raciocínio seria aplicado para a estação 3, ou seja, teríamos a soma de todos os assentos que chegaram à estação 3, \(x_{13} = 5\) e \(x_{23} = 40\), assim teríamos \(A_3 = 80 - 60 + 5 + 40 = 20 + 5 + 40 = 65\), e no final teríamos \(60 \leq 65\).

Além da lógica anterior, assume-se que há um trem, com vários vagões ou cabines, que viajará de uma estação $E_1$ até uma estação $E_n$ (onde $n$ é a última estação onde o trem chegará). Esse trem terá um itinerário que conterá o nome do trem, a estação de origem, a estação de destino, a data e hora de partida e de chegada. Além disso, haverá uma lista de preços (para o mesmo tipo de assento) a ser disponibilizada para venda. Cada um dos preços da lista será chamado de classe de controle ou control class. Os bilhetes serão disponibilizados para venda antes da partida do trem, e o tempo entre a disponibilização e a referida partida será chamado de horizonte de reserva. Esse horizonte será dividido em vários períodos, que podem ter diferentes temporalidades. Por exemplo, pode haver períodos em dias, semanas, meses, etc., e combinações entre eles. Cada um desses períodos será chamado de check point.

Cada relação possível entre uma estação e outra será denominada origem-destino ou trecho. Será dito que uma origem-destino é adjacente se, e somente se, não houver estações intermediárias entre elas; caso contrário, serão não adjacentes. Por exemplo, na figura \ref{fig: fig1}, os trechos adjacentes seriam: E1-E2, E2-E3, E3-E4, e os trechos não adjacentes seriam: E1-E3, E1-E4 e E2-E4. Além disso, observe que os trechos não adjacentes podem conter outros trechos, tanto adjacentes quanto não adjacentes. Por exemplo, o trecho E1-E4 da figura \ref{fig: fig1} contém os trechos adjacentes E1-E2, E2-E3, e contém os trechos não adjacentes E1-E3 e E2-E4.

Vejamos agora o modelo proposto completo:

\begin{table}[H]
	\centering
	\small
	\begin{tabular}{p{2cm} p{9.5cm} p{3.2cm}}
		\toprule
		\textbf{Definição} & \textbf{Descrição}                                                                                                                                            & \textbf{Domínio}                             \\ \midrule
		\multicolumn{3}{l}{\textbf{Conjuntos}}                                                                                                                                                                                          \\ \midrule
		$O$                & Conjunto de Estações de origem                                                                                                                          &                                              \\
		$D$                & Conjunto de Estações de Destino                                                                                                                          &                                              \\
		$OD$               & Conjunto de Trechos com itinerario                                                                                                                          &                                              \\
		$NAD$              & Conjunto de Trechos que NÃO são Adjacentes e que tem itinerario                                                                                             &                                              \\
		$BRI_{(o,d)}$      & Conjunto de Trechos contidos dentro de cada trecho $(o,d)$ NÃO Adjacente                                                                                    &                                              \\
		$V$                & Conjunto de Cabines do trem                                                                                                                                 &                                              \\
		$T$                & Conjunto de Check-Points (Períodos)                                                                                                                         &                                              \\ 
		$K_v$              & É o conjunto de classes de controle para cada vagão $v$. Por exemplo, suponha que há duas cabines $z$ e $p$, e cada cabine contém três classes de controle $c_1, c_2, c_3$, então a representação seria $K_z:\{c_1,c_2,c_3\}$ e $K_p:\{c_1,c_2,c_3\}$. Além disso, considere que os elementos de cada $K_v$ são ordenados, onde sempre se cumpre que a classe de menor indice é a classe mais costosa, ou seja $c_1>c_2>c_3$.                                                                                                        &                                              \\ \midrule
		\multicolumn{3}{l}{\textbf{Parâmetros}}                                                                                                                                                                                         \\ \midrule
		$n$                & Quantidade de Estações                                                                                                                                                 &                                              \\
		$Q$                & Capacidade física do trem                                                                                                                                          &                                              \\
		$P_{ijvk}$         & Preços  das passagem no Trecho $(i,j)$, Cabine $v$ e Classe de Control $k$                                                                                  & $(i,j) \in OD,v \in V, k \in K_v$            \\
		$d_{ijvkt}$        & Demanda no Trecho $(i,j)$, Cabine $v$ e Classe de Control $k$                                                                                 & $(i,j) \in OD,v \in V, k \in K_v, t \in T$   \\ \midrule
		\multicolumn{3}{l}{\textbf{Variáveis de decisão}}                                                                                                                                                                               \\ \midrule
		$X_{ijvkt}$        & Quantidade de passagem atribuídos no trecho $(i,j)$, cabine $v$ e com classe de control $k$ no período $t$                                                  & $(i,j) \in OD, v \in V, k \in K_v, t \in T$  \\
		$Y_{ijvkt}$        & Quantidade de passagem autorizados no trecho $(i,j)$, cabine $v$ e com classe de control $k$ no período $t$                                                 & $(i,j) \in OD, v \in V, k \in K_v, t \in T$  \\
		$BNA_{ijvkt}$      & É uma variavel binaria que toma o valor de 1 quando $Y_{ijvkt} \neq 0$ e toma  valor de 0 caso contrario, aplica-se apenas a trechos que não são adjacentes & $(i,j) \in NAD, v \in V, k \in K_v, t \in T$ \\\midrule
		\multicolumn{3}{l}{\textbf{Variável auxiliar}}                                                                                                                                                                               \\ \midrule
		$A_{i}$            & Armazena a quantidade de assentos vazios disponíveis para venda em cada estação de origem durante todo o horizonte de reserva. Cabe esclarecer que esta não é uma variável de decisão, pois esta variável apenas armazena um cálculo com base na capacidade física do trem e nas variáveis de decisão de atribuições                                                                                                     & $i \in O$                                    \\
		\bottomrule
	\end{tabular}
	\caption{Notação matemática}
	\label{tab: m1_definicao}
\end{table}

\begin{align}
	& Max \quad Z = \sum_{(i,j)\in OD} \sum_{v\in V} \sum_{k\in K_v} \sum_{t\in T} P_{ijvk} X_{ijvkt}                                                                                                                \label{eq: m1_fo}                          \\
	& \text{s.a.}  \notag                                                                                                                                                                                                                                       \\
	& A_{i} = A_{i-1} - \sum_{(i,j) \in OD/j \geq i} \sum_{v\in V} \sum_{k\in K_v}\sum_{t\in T}X_{i-1,j,v,k,t} + \sum_{(i,j) \in OD/j<i}\sum_{v\in V} \sum_{k\in K_v}\sum_{t\in T}X_{jivkt}, \quad \forall i \in O  \label{eq: m1_disponi}                     \\
	& \sum_{(i,j) \in OD}\sum_{v\in V} \sum_{k\in K_v}\sum_{t\in T} X_{ijvkt} \leq A_{i} , \quad \forall i \in O /i<j, i < n                                                                                        \label{eq: m1_cap_assig}                   \\
	& Y_{ijvkt} \geq Y_{i,j,v,k+1,t},  \quad \forall (i,j) \in OD / i < j, v \in V, k \in K_v / k < \lVert K_v \rVert , t \in T                                                                                                                     \label{eq: m1_jerar_class}                 \\   % P_{ijvk} \geq P_{i,j,v,k+1}
	& X_{ijvkt} \leq d_{ijvkt},  \quad \forall (i,j) \in OD / i < j  ,v \in V, k \in K_v, t\in T                                                                                                                                                  \label{eq: m1_assig_menor_dem}             \\
	& \sum_{(i,j) \in OD}\sum_{v\in V}\sum_{t\in T} Y_{i,j,v,k,t} \leq Q, \quad  k = min\{K_v\}, \forall i \in O                                                                                                  \label{eq: m1_cap_autho_1er_class}         \\
	& Y_{i,j,v,k,t} \geq  X_{i,j,v,k,t},  \quad k = max\{K_v\}, \forall(i,j) \in OD ,v \in V, t \in T                                                                                                                                     \label{eq: m1_autho_mayor_assig_1er_class} \\
	& Y_{i,j,v,k,t} \geq  X_{i,j,v,k,t} + Y_{i,j,v,k + 1,t} , \forall(i,j) \in OD, v \in V, k \in K_v / k < \lVert K_v \rVert , t \in T                                                                                                            \label{eq: m1_autho_mayor_assig_mas_autho} \\
	& BNA_{o,d,v,k,t} \leq Y_{o,d,v,k,t} \leq BNA_{o,d,v,k,t} Q, \quad  \forall (o,d)\in NAD, v \in V, k \in K_v, t \in T                                                                                                                \label{eq: m1_activ_bin_autho}            \\
	& BNA_{o,d,v,k,t} \leq Y_{i,j,v,k,t} \leq BNA_{o,d,v,k,t} Q, \quad  \forall (o,d)\in NAD, (i,j) \in BRI_{(o,d)}, v \in V, k \in K_v, t \in T                                                                                            \label{eq: m1_autho_igualar_trecho_maior}  \\
	& X_{0,j,v,k,t} = 0,     \quad \forall j \in D, k \in K_v, t \in T                                                                                                                                                                     \label{eq: m1_ini_assig}                   \\
	& A_{0} = Q                                                                                                                                                                                                      \label{eq: m1_ini_disponi}                 \\
	& X_{ijvkt} \in \mathbb{Z}^+                                                                                                                                                                                     \label{eq: m1_dom_assig}                   \\
	& Y_{ijvkt} \in \mathbb{Z}^+                                                                                                                                                                                     \label{eq: m1_dom_autho}                   \\
	& A_{j} \in \mathbb{Z}^+                                                                                                                                                                                         \label{eq: m1_dom_disponi}                 \\
	& BNA_{ijvkt} \in \{0,1\}                                                                                                                                                                                        \label{eq: m1_dom_bin_nadja}
\end{align}
% \begin{align}      
% 	& Y_{i,j,v,k,t} \geq  X_{i,j,v,k,t},  \quad k = max\{K_v\}, \forall(i,j),v,t                                                                                                                                     \label{eq: m1_autho_mayor_assig_1er_class} \\
% 	& Y_{i,j,v,k,t} \geq  X_{i,j,v,k,t} + Y_{i,j,v,k + 1,t} , \forall(i,j),v, k, t / k < \lVert K_v\rVert                                                                                                            \label{eq: m1_autho_mayor_assig_mas_autho} \\
% 	& BNA_{o,d,v,k,t} \leq Y_{o,d,v,k,t} \leq BNA_{o,d,v,k,t} Q, \quad  \forall (o,d)\in NAD, v, k, t                                                                                                                \label{eq: m1_activ_bin_autho}             \\
% 	& BNA_{o,d,v,k,t} \leq Y_{i,j,v,k,t} \leq BNA_{o,d,v,k,t} Q, \quad  \forall (o,d)\in NAD, (i,j) \in BRI_{(o,d)}, v,k,t                                                                                           \label{eq: m1_autho_igualar_trecho_maior}  \\
% 	& X_{0,j,v,k,t} = 0,     \quad \forall j,k,t                                                                                                                                                                     \label{eq: m1_ini_assig}                   \\
% 	& A_{0} = Q                                                                                                                                                                                                      \label{eq: m1_ini_disponi}                 \\
% 	& X_{ijvkt} \in \mathbb{Z}^+                                                                                                                                                                                     \label{eq: m1_dom_assig}                   \\
% 	& Y_{ijvkt} \in \mathbb{Z}^+                                                                                                                                                                                     \label{eq: m1_dom_autho}                   \\
% 	& A_{j} \in \mathbb{Z}^+                                                                                                                                                                                         \label{eq: m1_dom_disponi}                 \\
% 	& BNA_{ijvkt} \in \{0,1\}                                                                                                                                                                                        \label{eq: m1_dom_bin_nadja}
% \end{align}


Na equação \ref{eq: m1_fo}, a qual representa a função objetivo, temos a soma do produto entre a quantidade de assentos atribuídos a cada trajeto de origem e destino para a classe comercial em cada período e cada vagão, multiplicada pelo preço correspondente para cada trajeto e classe. Observe que queremos maximizar os ingressos em função dos assentos que estão atribuídos, que é o mais próximo que se tem da realidade em função da demanda conhecida.

A restrição \ref{eq: m1_disponi} é utilizada para calcular a disponibilidade de assentos de cada estação de origem, em cada período de tempo para cada classe em cada vagão e é a generalização do exemplo simplificado para calcular a variável auxiliar $A_i$.

A restrição \ref{eq: m1_cap_assig} garante que todas as autorizações habilitadas a partir de cada estação de origem para cada período e cada classe de cada vagão não excedam a disponibilidade da sua estação de origem correspondente (a disponibilidades é calculada na restrição \ref{eq: m1_disponi}).

A restrição \ref{eq: m1_jerar_class} é uma restrição de hierarquia e garante que as quantidades de autorizações para as classes de maior preço sejam sempre maiores do que as quantidades de autorizações de menor preço em cada vagão, em cada trecho, e em  cada período do horizonte de reserva.

A restrição \ref{eq: m1_assig_menor_dem} garante que a quantidade de atribuições não ultrapasse a demanda para cada trecho de cada classe em cada vagoen e em cada período no horizonte de reserva.

A restrição \ref{eq: m1_cap_autho_1er_class} garanta que a soma de autorizações da classe mais costosa de cada vagão, de cada estação de origem, de tudos os periodos, não ultrapase a capacidade do trem, note que apenas estamos considerando a classe mais cara devido à natureza cumulativa das variáveis de autorização é por isso que o valor de k é o mínimo das classes de cada vagão, pois a ordem do nome das classes é crescente mas o seu valor é decrescente. Para melhor compreensão, suponhamos uma solução para um problema de dois vagões $V_1$ e $V_2$, 3 classes para $V_1$ e 3 classes para $V_2$, 5 estações,  10 trechos, um período e uma capacidade física do trem de 700 cadeiras, conforme mostra a figura \ref{fig: autorization}.

\begin{figure}[!ht]
	\begin{center}
		\includegraphics[scale=0.40]{img/exemplo1.png}
		\caption{Solução factível para a variavel de desição Autorização}
		% Fonte:~\cite{khaksar2013genetic}}
		\label{fig: autorization}
	\end{center}
\end{figure}

Observe que os nomes das classes são números ordenados de forma crescente [1, 2, 3] também o valor da classe 1 é mais caro que o valor da classe 2 e este é maior que o valor da classe 3. Além disso, a soma que não ultrapassará a capacidade do trem é a soma das classes 1 de cada vagão de cada estação de origem. Por exemplo, para a estação 3 seria $43 + 85$ para $V_1$ trecho 3-4 e 3-5, mais, $105 + 88$ para $V_2$ nos mesmos trechos, ou seja $43 + 85 + 105 + 88 = 321 \leq 700$  


Até ao momento foi referido que a variável $Y$ tem um carácter cumulativo e são as restrições \ref{eq: m1_autho_mayor_assig_1er_class} e \ref{eq: m1_autho_mayor_assig_mas_autho} que controlam este comportamento. A restrição \ref{eq: m1_autho_mayor_assig_1er_class} é um caso particular da restrição \ref{eq: m1_autho_mayor_assig_mas_autho}, aplicada apenas à última classe, ou classe mais barata atribuída para cada vagão ($k=max\{K_v\}$), e garante que a soma de todos os períodos, de cada estação de origem da classe mais barata da variável "autorização" é maior ou igual à variável de decisão "segurada" nas mesmas condições. Por outro lado, a restrição \ref{eq: m1_autho_mayor_assig_mas_autho} garante que cada classe autorizada seja sempre maior ou igual à classe autorizada imediatamente menor, mais a quantidade segurada da mesma classe, isto para cada período, cada trecho e cada classe diferente da classe mais barata. Para melhor compreensão, assuma as mesmas suposições que foram feitas na restrição \ref{eq: m1_cap_autho_1er_class} %com a diferença que agora as tabelas representam uma solução factivel para a soma de n períodos e não um único período.


\begin{figure}[h!]
	\centering
	\begin{subfigure}[b]{0.35\linewidth}
		\includegraphics[width=\linewidth]{img/exemplo1.png}
		\caption{Autorização [Variavel $Y$]}
		\label{fig:auto_assig_a}
	\end{subfigure}\hspace{5mm}
	\begin{subfigure}[b]{0.35\linewidth}
		\includegraphics[width=\linewidth]{img/exemplo2.png}
		\caption{Atribução [Variavel $X$]}
		\label{fig:auto_assig_b}
	\end{subfigure}
	\caption{Solução factível para as variaveis de desição Autorização e Atribução}
	\label{fig:auto_assig}
\end{figure}

Observe a linha correspondente ao trecho 1-2 do vagão $V_2$ na tabela \ref{fig:auto_assig_b}, veja que a classe segurada mais barata foi a classe 2 com valor de 52, por este motivo na tabela \ref{fig:auto_assig_a} na mesma posição o valor deverá ser igual ou maior que 52, que neste caso é o mesmo valor; Agora observe para o mesmo trecho para o vagão $V_1$ classe 3 em ambas as tabelas acontece a mesma coisa, esse comportamento é garantido pela restrição \ref{eq: m1_autho_mayor_assig_1er_class}. Agora não vamos olhar para a classe mais barata, vamos olhar para qualquer outra, por exemplo, para o mesmo trecho veja a classe 1 do vagão $V_1$ da tabela \ref{fig:auto_assig_b} com valor 29, se quiséssemos saber o valor correspondente na tabela \ref{fig:auto_assig_a} deveríamos adicionar a classe imediata menor (à direita) da classe 1 na tabela \ref{fig:auto_assig_a}, neste caso seria a classe 2 com valor 116, e some o valor da classe 1 da tabela \ref{fig:auto_assig_b}, que já sabemos que é 29, assim, o valor buscado será maior ou igual a $116+29 = 145$, como visto em tabela \ref{fig:auto_assig_a}, lembre-se que nessa posição o valor mínimo será o calculado, mas poderá assumir um valor superior. Esta última situação é controlada pela restrição \ref{eq: m1_autho_mayor_assig_mas_autho}.

Suponha que você tem um trecho não adjacente E1-E3, e que esse trecho contém outros trechos E1-E2 e E2-E3. Para esta situação, a classe mais barata ativada no trecho E1-E3 deverá ser a classe mais barata ativada nos trechos E1-E2 e E2-E3. Isso é feito com o objetivo de que as combinações dos preços dos bilhetes por trechos não sejam mais econômicos do que o preço de um bilhete direto. Para alcançar isso, é criada uma variável binária para cada trecho não adjacente ($BNA$), que será ativada, ou tomará o valor de 1, quando as atribuições $"Y"$ (ou assentos a serem disponibilizados para venda) de uma classe desse trecho, de um vagão e de um período, forem diferentes de zero e tomará o valor de zero caso contrário. Esse comportamento será controlado pela restrição \ref{eq: m1_activ_bin_autho}.

Uma vez calculados os valores para $BNA$ dos trechos não adjacentes, a restrição \ref{eq: m1_autho_igualar_trecho_maior} fará com que as classes de controle de todos os trechos contidos em cada trecho não adjacente sejam $0$ ou, no mínimo $1$. Assim, quando a classe de controle do trecho não adjacente assumir um valor de $BNA = 0$, essa classe será $0$ para os trechos contidos. Por outro lado, se a classe de controle do trecho não adjacente assumir o valor de $BNA = 1$, então essa classe tomará, no mínimo, o valor de $1$ para todos os trechos contidos.

Para um melhor entendimento, suponha um trem com 1 vagão que passa por 3 estações, $E_1$, $E_2$ e $E_3$, e tem um horizonte de reserva de um único período. Sob esta situação, o trecho não adjacente seria $"E_1-E_3"$ e os trechos contidos seriam $"E_1-E_2"$ e $"E_2-E_3"$. Agora imagine que cada trecho tem 6 classes diferentes ($c_1, c_2, c_3, c_4, c_5, c_6$), onde o preço  de  $c_1 \geq c_2 \geq c_3 \geq c_4 \geq c_5 \geq c_6$, e que as autorizações para o trecho não adjacente (variavel $Y$) tomam os valores mostrados na tabela \ref{fig: exemplo_sip}:

\begin{figure}[!ht]
	\begin{center}
		\includegraphics[scale=0.15]{img/tab_trecho_grande.png}
		\caption{Exemplo simplificado do funcionamento da restrição \ref{eq: m1_autho_igualar_trecho_maior}}
		% Fonte:~\cite{khaksar2013genetic}}
		\label{fig: exemplo_sip}
	\end{center}
\end{figure}

Para o trecho $"E_1-E_3"$, note que quando $Y \neq 0$, $BNA = 1$ e quando $Y = 0$, $BNA = 0$ (controlado pela restrição \ref{eq: m1_activ_bin_autho}). Agora observe que, quando $ BNA = 1$ para uma certa classe, os trechos $E_1-E_2$ e $E_2-E_3$ assumem valores entre 1 e um número suficientemente grande ($1 \le Y \leq Q$) neste caso a capacidade do trem, o que indica que os assentos a serem disponibilizados para essa classe nesses trechos não podem ser zero. Por outro lado, quando $BNA = 0$, os trechos menores devem zerar a classe correspondente com ($0 \leq Y \leq 0$), ou seja, não se deve disponibilizar assentos com essa classe (controlado pela restrição \ref{eq: m1_autho_igualar_trecho_maior}). Deve-se esclarecer que as classe de controle dos trechos contidos, apenas imitam o comportamento da classe não adjacente correspondente, e não os valores que esta assume.

As restrições de \ref{eq: m1_ini_assig} e \ref{eq: m1_ini_disponi} são usadas para inicializar a restrição \ref{eq: m1_disponi} quando \(i = 1\). E as restrições de \ref{eq: m1_dom_assig} a \ref{eq: m1_dom_bin_nadja} representam o domínio das variáveis.

\section{Segunda modelagem matemática}\label{sec:modelo2}

Vamos considerar novamente uma versão simplificada do problema. Na verdade, para esta modelagem, serão levadas em conta as mesmas variáveis do primeiro modelo, exceto a variável de disponibilidade \(A\), conforme mostrado a seguir:

\noindent $x_{ij}$: Quantidade de assentos que será disponibilizada para venda no trecho com origem em $i$ e destino em $j$, onde $j>i$ \\
\noindent $P_{ij}$: Preço da passagem no trajeto com origem em $i$ e destino em $j$ \\
\noindent $Q$: Capacidade do trem

\noindent Assim, a função objetivo e as restrições são como segue:

$FO: max \quad x_{12}P_{12} + x_{13}P_{13} + x_{14}P_{14} + x_{23}P_{23} + x_{24}P_{24} + x_{34}P_{34}$

s.a.

\noindent{\it Restrições para estações de origem}

Estação 1: $x_{12} + x_{13} + x_{14} \leq Q $ \\
\indent Estação 2: $x_{23} + x_{24}  \leq  Q $ \\
\indent Estação 3: $x_{34} \leq Q $

\noindent{\it Restrições para estações de destino}

Estação 2: $x_{12} \leq Q $ \\
\indent Estação 3: $x_{13} + x_{23}  \leq  Q $ \\
\indent Estação 4: $x_{14} + x_{24} + x_{34} \leq Q $

Observe que esta formulação é baseada nos modelos de transporte, onde as estações de origem seriam os depósitos e estão restritas por sua capacidade (capacidade do trem), e as estações de destino seriam os destinos e estão restritas, neste caso, pela mesma capacidade do trem e não pela demanda de cada destino.

Esta formulação garante que sempre será disponibilizada, no máximo, a capacidade do trem tanto para cada saída quanto para cada chegada do trem. Imagine uma solução viável como a mostrada na figura \ref{fig: fig2}.

\begin{figure}[h]
	\begin{center}
		\includegraphics[scale=0.4]{img/fig3.png}
		\caption{Solução factível para o problema simplificado}
		% Fonte:~\cite{khaksar2013genetic}}
		\label{fig: fig3}
	\end{center}
\end{figure}

Observe que os valores das variáveis são os mesmos que foram mostrados na figura \ref{fig: fig3}. E ainda todas as restrições, tanto por linha quanto por coluna (por origens e por destinos), continuam sendo atendidas.

\begin{table}[h]
	\centering
	\small
	\begin{tabular}{p{2cm} p{9.5cm} p{3.2cm}}
		\toprule
		\textbf{Definição} & \textbf{Notação}                                                                                                                                            & \textbf{Domínio}                             \\ \midrule
		\multicolumn{3}{l}{\textbf{Conjuntos}}                                                                                                                                                                                          \\ \midrule
		$OD$               & Conjunto de Trechos com itinerario                                                                                                                          &                                              \\
		$NAD$              & Conjunto de Trechos que NÃO são Adjacentes e que tem itinerario                                                                                             &                                              \\
		$BRI_{(o,d)}$      & Conjunto de Trechos contidos dentro de cada trecho $(o,d)$ NÃO Adjacente                                                                                    &                                              \\
		$V$                & Conjunto de Cabines do trem                                                                                                                                 &                                              \\
		$K_v$              & Conjunto de Classes de Control de cada cabine em $V$                                                                                                        &                                              \\
		$T$                & Conjunto de Check-Points (Períodos)                                                                                                                         &                                              \\ \midrule
		\multicolumn{3}{l}{\textbf{Parâmetros}}                                                                                                                                                                                         \\ \midrule
		$Q$                & Capacidade do trem                                                                                                                                          &                                              \\
		$P_{ijvk}$         & Preços  das passagem no Trecho $(i,j)$, Cabine $v$ e Classe de Control $k$                                                                                  & $(i,j) \in OD,v \in V, k \in K_v$            \\
		$D_{ijvkt}$        & Demanda  das passagem no Trecho $(i,j)$, Cabine $v$ e Classe de Control $k$                                                                                 & $(i,j) \in OD,v \in V, k \in K_v, t \in T$   \\ \midrule
		\multicolumn{3}{l}{\textbf{Variáveis de decisão}}                                                                                                                                                                               \\ \midrule
		$X_{ijvkt}$        & Quantidade de passagem atribuídos no trecho $(i,j)$, cabine $v$ e com classe de control $k$ no período $t$                                                  & $(i,j) \in OD, v \in V, k \in K_v, t \in T$  \\
		$Y_{ijvkt}$        & Quantidade de passagem autorizados no trecho $(i,j)$, cabine $v$ e com classe de control $k$ no período $t$                                                 & $(i,j) \in OD, v \in V, k \in K_v, t \in T$  \\
		$BNA_{ijvkt}$      & É uma variavel binaria que toma o valor de 1 quando $Y_{ijvkt} \neq 0$ e toma  valor de 0 caso contrario, aplica-se apenas a trechos que não são adjacentes & $(i,j) \in NAD, v \in V, k \in K_v, t \in T$ \\
		\bottomrule
	\end{tabular}
	\caption{Notação matemática}
	\label{tab: m2_definicao}
\end{table}
\begin{align}
	 & Max \quad Z = \sum_{(i,j)\in OD} \sum_{v\in V} \sum_{k\in K_v} \sum_{t\in T} P_{ijvk} X_{ijvkt}                                 \label{eq: m2_fo}                          \\
	 & \text{s.a.}  \notag                                                                                                                                                        \\
	 & \sum_{(i,j)\in OD}\sum_{v\in V}\sum_{k\in K_v}\sum_{t\in T}X_{ijvkt} \leq Q , \quad \forall j / j>1, i<j                        \label{eq: m2_cap_assig_destino}           \\
	 & \sum_{(i,j)\in OD}\sum_{v\in V}\sum_{k\in K_v}\sum_{t\in T}X_{ijvkt} \leq Q , \quad \forall i / i<n, j>i                        \label{eq: m2_cap_assig_origem}            \\
	 & Y_{ijvkt} \geq Y_{i,j,v,k+1,t},  \quad \forall (i,j),v,k,t / i < j, k < \lVert K_v \rVert,  P_{ijvk} \geq P_{i,j,v,k+1}         \label{eq: m2_jerar_class}                 \\
	 & X_{ijvkt} \leq D_{ijvkt},  \quad \forall (i,j),v,k,t/ i < j                                                                     \label{eq: m2_assig_menor_dem}             \\
	 & \sum_{(i,j) \in OD}\sum_{v\in V}\sum_{t\in T} Y_{i,j,v,k,t} \leq Q, \quad  k = min\{K_v\}, \forall i \in OD                     \label{eq: m2_cap_autho_1er_class}         \\
	 & Y_{i,j,v,k,t} \geq  X_{i,j,v,k,t},  \quad k = max\{K_v\}, \forall(i,j),v,t                                                      \label{eq: m2_autho_mayor_assig_1er_class} \\
	 & Y_{i,j,v,k,t} \geq  X_{i,j,v,k,t} + Y_{i,j,v,k + 1,t} , \forall(i,j),v, k, t / k < \lVert K_v\rVert                             \label{eq: m2_autho_mayor_assig_mas_autho} %\\
	%  & BNA_{o,d,v,k,t} \leq Y_{o,d,v,k,t} \leq BNA_{o,d,v,k,t} Q, \quad  \forall (o,d)\in NAD, v, k, t                                 \label{eq: m2_activ_bin_autho}             \\
	%  & BNA_{o,d,v,k,t} \leq Y_{i,j,v,k,t} \leq BNA_{o,d,v,k,t} Q, \quad  \forall (o,d)\in NAD, (i,j) \in BRI_{(o,d)}, v,k,t            \label{eq: m2_autho_igualar_trecho_maior}  \\
	%  & X_{ijvkt} \in \mathbb{Z}^+                                                                                                      \label{eq: m2_dom_assig}                  %\\
\end{align}
\begin{align}
	& BNA_{o,d,v,k,t} \leq Y_{o,d,v,k,t} \leq BNA_{o,d,v,k,t} Q, \quad  \forall (o,d)\in NAD, v, k, t                                 \label{eq: m2_activ_bin_autho}             \\
	& BNA_{o,d,v,k,t} \leq Y_{i,j,v,k,t} \leq BNA_{o,d,v,k,t} Q, \quad  \forall (o,d)\in NAD, (i,j) \in BRI_{(o,d)}, v,k,t            \label{eq: m2_autho_igualar_trecho_maior}  \\
	& X_{ijvkt} \in \mathbb{Z}^+                                                                                                      \label{eq: m2_dom_assig}                   \\
	& Y_{ijvkt} \in \mathbb{Z}^+                                                                                                      \label{eq: m2_dom_autho}                   \\
	& BNA_{ijvkt} \in \{0,1\}                                                                                                         \label{eq: m2_dom_bin_nadja}
\end{align}

Como já foi mencionado, nesta formulação modificamos as restrições que controlam as variáveis asseguradas X, ou seja, mudamos as restrições 1 e 2 do primeiro modelo e eliminamos a variável de decisão Ai.

% \end{adjustwidth}
%Note que, na definição, não temos mais a variável de decisão de disponibilidade \(A_i\). Neste caso, a equação \ref{eq: m2_fo} representa a função objetivo que esta tentando maximizar a soma do produto entre as quantidades seguradas e os preços das mesmas, ou seja, estamos maximizando a receita em função das quantidades dos assentos que estão assegurados.

%A restrição \ref{eq: m2_cap_assig_destino} garante que a quantidade total de assentos autorizados para cada destino seja a quantidade máxima de assentos do trem para todas as classes e todos os períodos.
%A restrição \ref{eq: m2_cap_assig_origem} garante que a quantidade de assentos autorizados para cada origem seja no máximo a capacidade do trem para todas as classes e todos os períodos.
%As restrições de \ref{eq: m2_cap_autho_1er_class} a \ref{eq: m2_dom_autho} representam o mesmo que o primeiro modelo já exposto.\\

Para este caso, as restrições \ref{eq: m2_cap_assig_destino} e \ref{eq: m2_cap_assig_origem} representam a generalização do problema simplificado, onde a primeira garante que a quantidade de assentos autorizados para venda não viole a capacidade do trem ao chegar a cada estação de destino; e a segunda garante que a quantidade de assentos autorizados respeite a capacidade do trem no momento de sair de cada estação de origem. O restante das restrições foi explicado na formulação 1.
% \input{Contribuicoes}
% %\input{Metodos-de-resolucao-propostos}
% type:ignore

\chapter{Resultados Computacionais}

\section{Modelos Matemáticos}
Embora dois modelos matemáticos tenham sido propostos — o modelo independente e o modelo comportamental —, ao resolver as instâncias decidiu-se inicialmente testar os modelos sem as restrições de fulfillment nem as restrições de skip. Em seguida, cada modelo foi testado separadamente com cada grupo de restrições. Por fim, ambos os modelos foram avaliados considerando os dois conjuntos de restrições simultaneamente. Essa abordagem foi adotada para observar como a solução evoluía ou se comportava ao incluir cada tipo de restrição no modelo. Dito isso, a seguir são apresentados os modelos com suas respectivas descrições.
\vspace{0.5cm}

\begin{small}
	\begin{longtable}{p{5.4cm} p{10.4cm}}
		\hline
		\textbf{Tipo de Modelo}            & \textbf{Descrição}                                                                                                                       \\ \hline
		BaseModel                          & Modelo base independente sem as restrições de Fulfillments nem as restrições de Skiplagging.                                             \\ \hline
		BaseModelFulfillments              & Modelo base independente com as restrições de Fulfillments.                                                                              \\ \hline
		BaseModelSkiplagging               & Modelo base independente com as restrições de Skiplagging.                                                                               \\ \hline
		BaseModelFull                      & Modelo independente completo com os dois conjuntos de restrições.                                                                        \\ \hline
		HierarBehavioralModel              & Modelo base comportamental com ajuste de demanda do tipo hierarquia, sem as restrições de Fulfillments nem as restrições de Skiplagging. \\ \hline
		HierarBehavioralModelFulfillments  & Modelo base comportamental com ajuste de demanda do tipo hierarquia, com as restrições de Fulfillments.                                  \\ \hline
		HierarBehavioralModelSkiplagging   & Modelo base comportamental com ajuste de demanda do tipo hierarquia, com as restrições de Skiplagging.                                   \\ \hline
		HierarBehavioralModelFull          & Modelo comportamental completo com ajuste de demanda do tipo hierarquia, com os dois conjuntos de restrições.                            \\ \hline
		PercentBehavioralModel             & Modelo base comportamental com ajuste de demanda do tipo proporção, sem as restrições de Fulfillments nem as restrições de Skiplagging.  \\ \hline
		PercentBehavioralModelFulfillments & Modelo base comportamental com ajuste de demanda do tipo proporção, com as restrições de Fulfillments.                                   \\ \hline
		PercentBehavioralModelSkiplagging  & Modelo base comportamental com ajuste de demanda do tipo proporção, com as restrições de Skiplagging.                                    \\ \hline
		PercentBehavioralModelFull         & Modelo comportamental completo com ajuste de demanda do tipo proporção, com os dois conjuntos de restrições.                             \\ \hline
		\caption{Descrição dos Modelos}
		\label{tab:modelos}
	\end{longtable}
\end{small}


\section{Instâncias}

Para verificar as características dos modelos propostos, foram utilizadas 10 instâncias reais fornecidas pela empresa canadense Expetrio. As características de cada uma dessas instâncias estão apresentadas na Tabela \ref{tab:instancias}.

\begin{table}[h!]
	\centering
	\caption{Resumo das instâncias utilizadas no experimento}
	\begin{tabular}{ccccccc}
		\toprule
		\textbf{Instância}                                                  &
		\textbf{\begin{tabular}[c]{@{}c@{}}Nome \\ Trem\end{tabular}}       &
		\textbf{\begin{tabular}[c]{@{}c@{}}Capacidade \\ Trem\end{tabular}} &
		\textbf{Data Partida}                                               &
		\textbf{\# Trechos}                                                 &
		\textbf{\# Períodos}                                                &
		\textbf{\# Classes}                                                                                          \\
		\midrule
		Instância1                                                          & 68 & 561 & 2023-11-21 & 256 & 120 & 15 \\
		Instância2                                                          & 13 & 637 & 2023-11-26 & 252 & 146 & 15 \\
		Instância3                                                          & 71 & 563 & 2023-10-13 & 250 & 105 & 16 \\
		Instância4                                                          & 71 & 561 & 2023-11-21 & 241 & 124 & 15 \\\midrule
		Instância5                                                          & 8  & 561 & 2022-12-04 & 152 & 116 & 17 \\
		Instância6                                                          & 40 & 565 & 2023-11-21 & 149 & 68  & 16 \\
		Instância7                                                          & 74 & 491 & 2023-07-25 & 136 & 70  & 16 \\\midrule
		Instância8                                                          & 72 & 493 & 2023-10-25 & 100 & 85  & 16 \\
		Instância9                                                          & 45 & 563 & 2023-03-16 & 90  & 73  & 16 \\
		Instância10                                                         & 15 & 493 & 2023-08-07 & 50  & 60  & 12 \\
		\bottomrule
	\end{tabular}
	\label{tab:instancias}
\end{table}

A coluna "\# Trechos" \, representa a quantidade de trechos que a viagem do trem possui em cada instância. A coluna "\# Períodos" \, indica a quantidade de períodos considerados na programação desse trajeto dentro do seu horizonte de reserva. Por fim, a coluna "\# Classes" \, mostra o número máximo de classes disponíveis para cada viagem de cada trem; Observe que, para cada trecho, podem estar disponíveis todas as classes indicadas na coluna correspondente ou uma quantidade menor.

Por outro lado, afirmamos que as instâncias compreendidas entre a Instância1 e a Instância4 são classificadas como grandes, as instâncias entre a Instância5 e a Instância7 como médias, e as demais como pequenas.

\section{Experimentos Computacionais}
Para resolver as instâncias utilizando os modelos matemáticos propostos, foi empregado um computador da marca Dell, modelo Precision 3660, equipado com o sistema operacional Windows 11 Pro de 64 bits, processador 13th Gen Intel(R) Core(TM) i7-13700 2.10 GHz baseado em arquitetura x64, e memória RAM de 16GB. Além disso, utilizou-se o Python como linguagem de programação na versão 3.10.14, e o solver Gurobi na versão 11.0.3.

Em primeiro lugar, apresentam-se os resultados de cada instância após serem resolvidas por meio de cada um dos modelos propostos. Essa organização busca evidenciar de forma clara e comparativa o desempenho dos modelos em diferentes cenários, destacando as principais diferenças e similaridades nas soluções obtidas para cada instância.

\begin{table}[h!]
	\centering
	\resizebox{\textwidth}{!}{%
		\begin{tabular}{lcccccccccc}
			\hline
			\textbf{Modelo}                                                             &
			\textbf{\begin{tabular}[c]{@{}c@{}}T. Criação \\ Model (seg.)\end{tabular}} &
			\textbf{\begin{tabular}[c]{@{}c@{}}T. Solução \\(seg.)\end{tabular}}        &
			\textbf{\begin{tabular}[c]{@{}c@{}}N° Nós \\Explorado\end{tabular}}         &
			\textbf{\begin{tabular}[c]{@{}c@{}}N° \\Iterações\end{tabular}}             &
			\textbf{\begin{tabular}[c]{@{}c@{}}N° \\Soluções\end{tabular}}              &
			\textbf{\begin{tabular}[c]{@{}c@{}}Z\\Relaxado\end{tabular}}                &
			\textbf{Z*}                                                                 &
			\textbf{\begin{tabular}[c]{@{}c@{}}$\Delta$ Z \\(\%)\end{tabular}}                                                                        \\ \hline
			BaseModel                                                                   & 1,97  & 0,07 & 1 & 0   & 1 & -          & 170.484,85 & 0,00 \\
			BaseModelFulfillments                                                       & 2,46  & 0,08 & 1 & 0   & 1 & -          & 170.484,85 & 0,00 \\
			BaseModelSkiplagging                                                        & 83,38 & 0,20 & 1 & 0   & 1 & -          & 4.875,58   & 0,00 \\
			BaseModelFull                                                               & 63,10 & 0,19 & 1 & 0   & 1 & -          & 4.406,58   & 0,00 \\ \hline
			HierarBehavioralModel                                                       & 2,82  & 0,07 & 1 & 53  & 2 & 170.467,27 & 170.454,60 & 0,01 \\
			HierarBehavioralModelFulfillments                                           & 3,51  & 0,10 & 1 & 190 & 2 & 170.467,27 & 170.454,60 & 0,01 \\
			HierarBehavioralModelSkiplagging                                            & 64,61 & 0,19 & 1 & 0   & 1 & -          & 4.875,58   & 0,00 \\
			HierarBehavioralModelFull                                                   & 63,95 & 0,21 & 1 & 0   & 1 & -          & 4.361,53   & 0,00 \\ \hline
			PercentBehavioralModel                                                      & 2,81  & 0,06 & 1 & 0   & 1 & -          & 166.345,19 & 0,00 \\
			PercentBehavioralModelFulfillments                                          & 3,32  & 0,09 & 1 & 190 & 2 & 170.467,27 & 170.454,60 & 0,01 \\
			PercentBehavioralModelSkiplagging                                           & 64,51 & 0,19 & 1 & 0   & 1 & -          & 4.875,58   & 0,00 \\
			PercentBehavioralModelFull                                                  & 63,89 & 0,21 & 1 & 0   & 1 & -          & 4.361,53   & 0,00 \\ \hline
		\end{tabular}%
	}
	\caption{Resultados para a Instância1}
	\label{tab:resul_instan1}
\end{table}


\begin{table}[h!]
	\centering
	\resizebox{\textwidth}{!}{%
		\begin{tabular}{lcccccccccc}
			\hline
			\textbf{Modelo}                                                             &
			\textbf{\begin{tabular}[c]{@{}c@{}}T. Criação \\ Model (seg.)\end{tabular}} &
			\textbf{\begin{tabular}[c]{@{}c@{}}T. Solução \\(seg.)\end{tabular}}        &
			\textbf{\begin{tabular}[c]{@{}c@{}}N° Nós \\Explorado\end{tabular}}         &
			\textbf{\begin{tabular}[c]{@{}c@{}}N° \\Iterações\end{tabular}}             &
			\textbf{\begin{tabular}[c]{@{}c@{}}N° \\Soluções\end{tabular}}              &
			\textbf{\begin{tabular}[c]{@{}c@{}}Z\\Relaxado\end{tabular}}                &
			\textbf{Z*}                                                                 &
			\textbf{\begin{tabular}[c]{@{}c@{}}$\Delta$ Z \\(\%)\end{tabular}}                                                                        \\ \hline
			BaseModel                                                                   & 2,57  & 0,08 & 1 & 0   & 1 & -          & 314.883,03 & 0,00 \\
			BaseModelFulfillments                                                       & 3,10  & 0,10 & 1 & 0   & 1 & -          & 314.883,03 & 0,00 \\
			BaseModelSkiplagging                                                        & 70,31 & 0,21 & 1 & 0   & 1 & -          & 9.094,40   & 0,00 \\
			BaseModelFull                                                               & 69,87 & 0,21 & 1 & 0   & 1 & -          & 8.015,40   & 0,00 \\ \hline
			HierarBehavioralModel                                                       & 3,63  & 0,08 & 1 & 54  & 2 & 314.613,53 & 314.599,53 & 0,00 \\
			HierarBehavioralModelFulfillments                                           & 4,27  & 0,11 & 1 & 345 & 5 & 314.613,53 & 314.599,53 & 0,00 \\
			HierarBehavioralModelSkiplagging                                            & 71,38 & 0,20 & 1 & 0   & 1 & -          & 9.094,40   & 0,00 \\
			HierarBehavioralModelFull                                                   & 70,85 & 0,20 & 1 & 0   & 1 & -          & 8.015,40   & 0,00 \\ \hline
			PercentBehavioralModel                                                      & 3,58  & 0,07 & 1 & 0   & 1 & -          & 313.547,03 & 0,00 \\
			PercentBehavioralModelFulfillments                                          & 4,25  & 0,10 & 1 & 345 & 5 & 314.613,53 & 314.599,53 & 0,00 \\
			PercentBehavioralModelSkiplagging                                           & 71,58 & 0,20 & 1 & 0   & 1 & -          & 9.094,40   & 0,00 \\
			PercentBehavioralModelFull                                                  & 71,06 & 0,21 & 1 & 0   & 1 & -          & 8.015,40   & 0,00 \\ \hline
		\end{tabular}%
	}
	\caption{Resultados para a Instância2}
	\label{tab:resul_instan2}
\end{table}


\begin{table}[h!]
	\centering
	\resizebox{\textwidth}{!}{%
		\begin{tabular}{lcccccccccc}
			\hline
			\textbf{Modelo}                                                             &
			\textbf{\begin{tabular}[c]{@{}c@{}}T. Criação \\ Model (seg.)\end{tabular}} &
			\textbf{\begin{tabular}[c]{@{}c@{}}T. Solução \\(seg.)\end{tabular}}        &
			\textbf{\begin{tabular}[c]{@{}c@{}}N° Nós \\Explorado\end{tabular}}         &
			\textbf{\begin{tabular}[c]{@{}c@{}}N° \\Iterações\end{tabular}}             &
			\textbf{\begin{tabular}[c]{@{}c@{}}N° \\Soluções\end{tabular}}              &
			\textbf{\begin{tabular}[c]{@{}c@{}}Z\\Relaxado\end{tabular}}                &
			\textbf{Z*}                                                                 &
			\textbf{\begin{tabular}[c]{@{}c@{}}$\Delta$ Z \\(\%)\end{tabular}}                                                                         \\ \hline
			BaseModel                                                                   & 1,99   & 0,07 & 1 & 0   & 1 & -          & 145.579,55 & 0,00 \\
			BaseModelFulfillments                                                       & 2,37   & 0,09 & 1 & 0   & 1 & -          & 145.579,55 & 0,00 \\
			BaseModelSkiplagging                                                        & 206,88 & 0,48 & 1 & 0   & 1 & -          & 1.955,91   & 0,00 \\
			BaseModelFull                                                               & 200,57 & 0,50 & 1 & 0   & 1 & -          & 1.940,91   & 0,00 \\ \hline
			HierarBehavioralModel                                                       & 2,83   & 0,07 & 1 & 85  & 4 & 144.986,40 & 144.978,90 & 0,01 \\
			HierarBehavioralModelFulfillments                                           & 3,18   & 0,10 & 1 & 271 & 3 & 144.986,40 & 144.978,90 & 0,01 \\
			HierarBehavioralModelSkiplagging                                            & 204,46 & 0,48 & 1 & 0   & 4 & -          & 2.063,91   & 0,00 \\
			HierarBehavioralModelFull                                                   & 197,51 & 0,47 & 1 & 0   & 4 & -          & 2.048,91   & 0,00 \\ \hline
			PercentBehavioralModel                                                      & 2,89   & 0,06 & 1 & 0   & 1 & -          & 139.907,48 & 0,00 \\
			PercentBehavioralModelFulfillments                                          & 3,00   & 0,09 & 1 & 271 & 3 & 144.986,40 & 144.978,90 & 0,01 \\
			PercentBehavioralModelSkiplagging                                           & 202,23 & 0,49 & 1 & 0   & 4 & -          & 2.063,91   & 0,00 \\
			PercentBehavioralModelFull                                                  & 197,98 & 0,47 & 1 & 0   & 4 & -          & 2.048,91   & 0,00 \\ \hline
		\end{tabular}%
	}
	\caption{Resultados para a Instância3}
	\label{tab:resul_instan3}
\end{table}


\begin{table}[h!]
	\centering
	\resizebox{\textwidth}{!}{%
		\begin{tabular}{lcccccccccc}
			\hline
			\textbf{Modelo}                                                             &
			\textbf{\begin{tabular}[c]{@{}c@{}}T. Criação \\ Model (seg.)\end{tabular}} &
			\textbf{\begin{tabular}[c]{@{}c@{}}T. Solução \\(seg.)\end{tabular}}        &
			\textbf{\begin{tabular}[c]{@{}c@{}}N° Nós \\Explorado\end{tabular}}         &
			\textbf{\begin{tabular}[c]{@{}c@{}}N° \\Iterações\end{tabular}}             &
			\textbf{\begin{tabular}[c]{@{}c@{}}N° \\Soluções\end{tabular}}              &
			\textbf{\begin{tabular}[c]{@{}c@{}}Z\\Relaxado\end{tabular}}                &
			\textbf{Z*}                                                                 &
			\textbf{\begin{tabular}[c]{@{}c@{}}$\Delta$ Z \\(\%)\end{tabular}}                                                                                                                                             \\ \hline
			BaseModel                          & 1,98   & 0,06 & 1 & 0   & 1 & -          & 166.118,29 & 0,00 \\ 
			BaseModelFulfillments              & 2,17   & 0,08 & 1 & 0   & 1 & -          & 166.118,29 & 0,00 \\ 
			BaseModelSkiplagging               & 22,44  & 0,11 & 1 & 0   & 1 & -          & 6.139,82   & 0,00 \\ 
			BaseModelFull                      & 22,09  & 0,11 & 1 & 0   & 1 & -          & 5.867,82   & 0,00 \\ \hline
			HierarBehavioralModel              & 2,58   & 0,06 & 1 & 28  & 2 & -          & 165.869,29 & 0,00 \\ 
			HierarBehavioralModelFulfillments  & 3,00   & 0,09 & 1 & 120 & 4 & -          & 165.869,29 & 0,00 \\ 
			HierarBehavioralModelSkiplagging   & 22,78  & 0,10 & 1 & 0   & 1 & -          & 6.237,82   & 0,00 \\ 
			HierarBehavioralModelFull          & 22,70  & 0,10 & 1 & 0   & 1 & -          & 5.965,82   & 0,00 \\ \hline
			PercentBehavioralModel             & 2,59   & 0,05 & 1 & 0   & 1 & -          & 162.681,09 & 0,00 \\ 
			PercentBehavioralModelFulfillments & 3,01   & 0,08 & 1 & 120 & 4 & -          & 165.869,29 & 0,00 \\ 
			PercentBehavioralModelSkiplagging  & 22,86  & 0,10 & 1 & 0   & 1 & -          & 6.237,82   & 0,00 \\ 
			PercentBehavioralModelFull         & 22,82  & 0,11 & 1 & 0   & 1 & -          & 5.965,82   & 0,00 \\ \hline
		\end{tabular}%
	}
	\caption{Resultados para a Instância4}
	\label{tab:resul_instan4}
\end{table}


\begin{table}[h!]
	\centering
	\resizebox{\textwidth}{!}{%
		\begin{tabular}{lcccccccccc}
			\hline
			\textbf{Modelo}                                                             &
			\textbf{\begin{tabular}[c]{@{}c@{}}T. Criação \\ Model (seg.)\end{tabular}} &
			\textbf{\begin{tabular}[c]{@{}c@{}}T. Solução \\(seg.)\end{tabular}}        &
			\textbf{\begin{tabular}[c]{@{}c@{}}N° Nós \\Explorado\end{tabular}}         &
			\textbf{\begin{tabular}[c]{@{}c@{}}N° \\Iterações\end{tabular}}             &
			\textbf{\begin{tabular}[c]{@{}c@{}}N° \\Soluções\end{tabular}}              &
			\textbf{\begin{tabular}[c]{@{}c@{}}Z\\Relaxado\end{tabular}}                &
			\textbf{Z*}                                                                 &
			\textbf{\begin{tabular}[c]{@{}c@{}}$\Delta$ Z \\(\%)\end{tabular}}                                                                                                                                             \\ \hline
			BaseModel                          & 1,71   & 0,06 & 1 & 0   & 1 & -          & 130.291,89 & 0,00 \\ 
			BaseModelFulfillments              & 1,97   & 0,06 & 1 & 0   & 1 & -          & 130.291,89 & 0,00 \\ 
			BaseModelSkiplagging               & 3,69   & 0,06 & 1 & 0   & 1 & -          & 4.230,91   & 0,00 \\ 
			BaseModelFull                      & 3,73   & 0,06 & 1 & 0   & 1 & -          & 3.738,91   & 0,00 \\ \hline
			HierarBehavioralModel              & 2,19   & 0,05 & 1 & 0   & 1 & -          & 129.930,89 & 0,00 \\ 
			HierarBehavioralModelFulfillments  & 2,46   & 0,07 & 1 & 1   & 3 & -          & 129.930,89 & 0,00 \\ 
			HierarBehavioralModelSkiplagging   & 4,21   & 0,06 & 1 & 0   & 1 & -          & 4.230,91   & 0,00 \\ 
			HierarBehavioralModelFull          & 4,23   & 0,06 & 1 & 0   & 1 & -          & 3.707,91   & 0,00 \\ \hline
			PercentBehavioralModel             & 2,16   & 0,05 & 1 & 0   & 1 & -          & 127.021,19 & 0,00 \\ 
			PercentBehavioralModelFulfillments & 2,46   & 0,07 & 1 & 1   & 3 & -          & 129.930,89 & 0,00 \\ 
			PercentBehavioralModelSkiplagging  & 4,18   & 0,06 & 1 & 0   & 1 & -          & 4.230,91   & 0,00 \\ 
			PercentBehavioralModelFull         & 4,23   & 0,06 & 1 & 0   & 1 & -          & 3.707,91   & 0,00 \\ \hline
		\end{tabular}%
	}
	\caption{Resultados para a Instância5}
	\label{tab:resul_instan5}
\end{table}




\begin{table}[h!]
	\centering
	\resizebox{\textwidth}{!}{%
		\begin{tabular}{lcccccccccc}
			\hline
			\textbf{Modelo}                                                             &
			\textbf{\begin{tabular}[c]{@{}c@{}}T. Criação \\ Model (seg.)\end{tabular}} &
			\textbf{\begin{tabular}[c]{@{}c@{}}T. Solução \\(seg.)\end{tabular}}        &
			\textbf{\begin{tabular}[c]{@{}c@{}}N° Nós \\Explorado\end{tabular}}         &
			\textbf{\begin{tabular}[c]{@{}c@{}}N° \\Iterações\end{tabular}}             &
			\textbf{\begin{tabular}[c]{@{}c@{}}N° \\Soluções\end{tabular}}              &
			\textbf{\begin{tabular}[c]{@{}c@{}}Z\\Relaxado\end{tabular}}                &
			\textbf{Z*}                                                                 &
			\textbf{\begin{tabular}[c]{@{}c@{}}$\Delta$ Z \\(\%)\end{tabular}}                                                                                                                                             \\ \hline
			BaseModel                          & 0,96   & 0,03 & 0 & 0   & 1 & -          & 67.715,07 & 0,00 \\ 
			BaseModelFulfillments              & 1,05   & 0,04 & 0 & 0   & 1 & -          & 67.715,07 & 0,00 \\ 
			BaseModelSkiplagging               & 1,67   & 0,03 & 0 & 0   & 1 & -          & 5.956,95  & 0,00 \\ 
			BaseModelFull                      & 1,68   & 0,03 & 0 & 0   & 1 & -          & 5.558,95  & 0,00 \\ \hline
			HierarBehavioralModel              & 1,24   & 0,03 & 0 & 0   & 1 & -          & 67.593,22 & 0,00 \\ 
			HierarBehavioralModelFulfillments  & 1,42   & 0,05 & 0 & 0   & 1 & -          & 67.587,22 & 0,00 \\ 
			HierarBehavioralModelSkiplagging   & 2,02   & 0,04 & 0 & 0   & 1 & -          & 5.956,95  & 0,00 \\ 
			HierarBehavioralModelFull          & 2,04   & 0,04 & 0 & 0   & 1 & -          & 5.558,95  & 0,00 \\ \hline
			PercentBehavioralModel             & 1,26   & 0,03 & 0 & 0   & 1 & -          & 65.256,64 & 0,00 \\ 
			PercentBehavioralModelFulfillments & 1,43   & 0,04 & 0 & 0   & 1 & -          & 67.587,22 & 0,00 \\ 
			PercentBehavioralModelSkiplagging  & 2,05   & 0,03 & 0 & 0   & 1 & -          & 5.956,95  & 0,00 \\ 
			PercentBehavioralModelFull         & 2,03   & 0,04 & 0 & 0   & 1 & -          & 5.558,95  & 0,00 \\ \hline
		\end{tabular}%
	}
	\caption{Resultados para a Instância6}
	\label{tab:resul_instan6}
\end{table}



\begin{table}[h!]
	\centering
	\resizebox{\textwidth}{!}{%
		\begin{tabular}{lcccccccccc}
			\hline
			\textbf{Modelo}                                                             &
			\textbf{\begin{tabular}[c]{@{}c@{}}T. Criação \\ Model (seg.)\end{tabular}} &
			\textbf{\begin{tabular}[c]{@{}c@{}}T. Solução \\(seg.)\end{tabular}}        &
			\textbf{\begin{tabular}[c]{@{}c@{}}N° Nós \\Explorado\end{tabular}}         &
			\textbf{\begin{tabular}[c]{@{}c@{}}N° \\Iterações\end{tabular}}             &
			\textbf{\begin{tabular}[c]{@{}c@{}}N° \\Soluções\end{tabular}}              &
			\textbf{\begin{tabular}[c]{@{}c@{}}Z\\Relaxado\end{tabular}}                &
			\textbf{Z*}                                                                 &
			\textbf{\begin{tabular}[c]{@{}c@{}}$\Delta$ Z \\(\%)\end{tabular}}                                                                                                                                             \\ \hline
			BaseModel                          & 0,87   & 0,04 & 0 & 0   & 1 & -          & 48.719,71 & 0,00 \\ 
			BaseModelFulfillments              & 1,10   & 0,05 & 0 & 0   & 1 & -          & 48.719,71 & 0,00 \\ 
			BaseModelSkiplagging               & 2,68   & 0,03 & 1 & 0   & 1 & -          & 1.028,50  & 0,00 \\ 
			BaseModelFull                      & 2,70   & 0,03 & 1 & 0   & 1 & -          & 959,50    & 0,00 \\ \hline
			HierarBehavioralModel              & 1,31   & 0,06 & 1 & 72  & 8 & -          & 47.990,71 & 0,00 \\ 
			HierarBehavioralModelFulfillments  & 1,51   & 0,08 & 1 & 238 & 6 & -          & 47.990,71 & 0,00 \\ 
			HierarBehavioralModelSkiplagging   & 3,11   & 0,03 & 1 & 0   & 1 & -          & 1.028,50  & 0,00 \\ 
			HierarBehavioralModelFull          & 3,12   & 0,04 & 1 & 0   & 1 & -          & 959,50    & 0,00 \\ \hline
			PercentBehavioralModel             & 1,27   & 0,02 & - & -   & - & -          & Infac     & - \\ 
			PercentBehavioralModelFulfillments & 1,51   & 0,08 & 1 & 238 & 6 & -          & 47.990,71 & 0,00 \\ 
			PercentBehavioralModelSkiplagging  & 3,11   & 0,04 & 1 & 0   & 1 & -          & 1.028,50  & 0,00 \\ 
			PercentBehavioralModelFull         & 3,09   & 0,03 & 1 & 0   & 1 & -          & 959,50    & 0,00 \\ \hline
		\end{tabular}%
	}
	\caption{Resultados para a Instância7}
	\label{tab:resul_instan7}
\end{table}


\begin{table}[h!]
	\centering
	\resizebox{\textwidth}{!}{%
		\begin{tabular}{lcccccccccc}
			\hline
			\textbf{Modelo}                                                             &
			\textbf{\begin{tabular}[c]{@{}c@{}}T. Criação \\ Model (seg.)\end{tabular}} &
			\textbf{\begin{tabular}[c]{@{}c@{}}T. Solução \\(seg.)\end{tabular}}        &
			\textbf{\begin{tabular}[c]{@{}c@{}}N° Nós \\Explorado\end{tabular}}         &
			\textbf{\begin{tabular}[c]{@{}c@{}}N° \\Iterações\end{tabular}}             &
			\textbf{\begin{tabular}[c]{@{}c@{}}N° \\Soluções\end{tabular}}              &
			\textbf{\begin{tabular}[c]{@{}c@{}}Z\\Relaxado\end{tabular}}                &
			\textbf{Z*}                                                                 &
			\textbf{\begin{tabular}[c]{@{}c@{}}$\Delta$ Z \\(\%)\end{tabular}}                                                                                                                                             \\ \hline
			BaseModel                          & 0,81   & 0,03 & 0 & 0   & 1 & -          & 53.844,31 & 0,00 \\ 
			BaseModelFulfillments              & 1,03   & 0,05 & 0 & 0   & 1 & -          & 53.844,31 & 0,00 \\ 
			BaseModelSkiplagging               & 1,66   & 0,04 & 0 & 0   & 1 & -          & 1.274,66  & 0,00 \\ 
			BaseModelFull                      & 1,69   & 0,04 & 0 & 0   & 1 & -          & 1.244,66  & 0,00 \\ \hline
			HierarBehavioralModel              & 1,23   & 0,06 & 1 & 64  & 5 & -          & 52.091,71 & 0,00 \\ 
			HierarBehavioralModelFulfillments  & 1,42   & 0,07 & 1 & 215 & 8 & -          & 52.091,71 & 0,00 \\ 
			HierarBehavioralModelSkiplagging   & 1,94   & 0,03 & 0 & 0   & 1 & -          & 1.274,66  & 0,00 \\ 
			HierarBehavioralModelFull          & 1,95   & 0,04 & 0 & 0   & 1 & -          & 1.244,66  & 0,00 \\ \hline
			PercentBehavioralModel             & 1,17   & 0,02 & - & -   & - & -          & Infac     & -\\ 
			PercentBehavioralModelFulfillments & 1,40   & 0,07 & 1 & 215 & 8 & -          & 52.091,71 & 0,00 \\ 
			PercentBehavioralModelSkiplagging  & 1,94   & 0,04 & 0 & 0   & 1 & -          & 1.274,66  & 0,00 \\ 
			PercentBehavioralModelFull         & 1,98   & 0,04 & 0 & 0   & 1 & -          & 1.244,66  & 0,00 \\ \hline
		\end{tabular}%
	}
	\caption{Resultados para a Instância8}
	\label{tab:resul_instan8}
\end{table}


\begin{table}[h!]
	\centering
	\resizebox{\textwidth}{!}{%
		\begin{tabular}{lcccccccccc}
			\hline
			\textbf{Modelo}                                                             &
			\textbf{\begin{tabular}[c]{@{}c@{}}T. Criação \\ Model (seg.)\end{tabular}} &
			\textbf{\begin{tabular}[c]{@{}c@{}}T. Solução \\(seg.)\end{tabular}}        &
			\textbf{\begin{tabular}[c]{@{}c@{}}N° Nós \\Explorado\end{tabular}}         &
			\textbf{\begin{tabular}[c]{@{}c@{}}N° \\Iterações\end{tabular}}             &
			\textbf{\begin{tabular}[c]{@{}c@{}}N° \\Soluções\end{tabular}}              &
			\textbf{\begin{tabular}[c]{@{}c@{}}Z\\Relaxado\end{tabular}}                &
			\textbf{Z*}                                                                 &
			\textbf{\begin{tabular}[c]{@{}c@{}}$\Delta$ Z \\(\%)\end{tabular}}                                                                                                                                             \\ \hline
			BaseModel                          & 0,73   & 0,03 & 0 & 0   & 1 & -          & 55.282,08 & 0,00 \\ 
			BaseModelFulfillments              & 0,88   & 0,04 & 0 & 0   & 1 & -          & 55.282,08 & 0,00 \\ 
			BaseModelSkiplagging               & 1,29   & 0,03 & 0 & 0   & 1 & -          & 2.589,25  & 0,00 \\ 
			BaseModelFull                      & 1,31   & 0,03 & 0 & 0   & 1 & -          & 2.589,25  & 0,00 \\ \hline
			HierarBehavioralModel              & 1,04   & 0,03 & 0 & 0   & 1 & -          & 55.230,08 & 0,00 \\ 
			HierarBehavioralModelFulfillments  & 1,21   & 0,06 & 1 & 13  & 5 & -          & 55.230,08 & 0,00 \\ 
			HierarBehavioralModelSkiplagging   & 1,65   & 0,03 & 0 & 0   & 1 & -          & 2.589,25  & 0,00 \\ 
			HierarBehavioralModelFull          & 1,66   & 0,04 & 0 & 0   & 1 & -          & 2.589,25  & 0,00 \\ \hline
			PercentBehavioralModel             & 1,05   & 0,03 & 0 & 0   & 1 & -          & 50.834,74 & 0,00 \\ 
			PercentBehavioralModelFulfillments & 1,21   & 0,05 & 1 & 13  & 5 & -          & 55.230,08 & 0,00 \\ 
			PercentBehavioralModelSkiplagging  & 1,66   & 0,03 & 0 & 0   & 1 & -          & 2.589,25  & 0,00 \\ 
			PercentBehavioralModelFull         & 1,66   & 0,04 & 0 & 0   & 1 & -          & 2.589,25  & 0,00 \\ \hline
		\end{tabular}%
	}
	\caption{Resultados para a Instância9}
	\label{tab:resul_instan9}
\end{table}


\begin{table}[h!]
	\centering
	\resizebox{\textwidth}{!}{%
		\begin{tabular}{lcccccccccc}
			\hline
			\textbf{Modelo}                                                             &
			\textbf{\begin{tabular}[c]{@{}c@{}}T. Criação \\ Model (seg.)\end{tabular}} &
			\textbf{\begin{tabular}[c]{@{}c@{}}T. Solução \\(seg.)\end{tabular}}        &
			\textbf{\begin{tabular}[c]{@{}c@{}}N° Nós \\Explorado\end{tabular}}         &
			\textbf{\begin{tabular}[c]{@{}c@{}}N° \\Iterações\end{tabular}}             &
			\textbf{\begin{tabular}[c]{@{}c@{}}N° \\Soluções\end{tabular}}              &
			\textbf{\begin{tabular}[c]{@{}c@{}}Z\\Relaxado\end{tabular}}                &
			\textbf{Z*}                                                                 &
			\textbf{\begin{tabular}[c]{@{}c@{}}$\Delta$ Z \\(\%)\end{tabular}}                                                                                                                                             \\ \hline
			BaseModel                          & 0,38   & 0,01 & 0 & 0   & 1 & -          & 21.548,62 & 0,00 \\ 
			BaseModelFulfillments              & 0,43   & 0,01 & 0 & 0   & 1 & -          & 21.548,62 & 0,00 \\ 
			BaseModelSkiplagging               & 0,66   & 0,01 & 0 & 0   & 1 & -          & 979,86    & 0,00 \\ 
			BaseModelFull                      & 0,66   & 0,01 & 0 & 0   & 1 & -          & 903,86    & 0,00 \\ \hline
			HierarBehavioralModel              & 0,53   & 0,01 & 1 & 5   & 3 & -          & 22.112,12 & 0,00 \\ 
			HierarBehavioralModelFulfillments  & 0,64   & 0,02 & 1 & 0   & 1 & -          & 22.112,12 & 0,00 \\ 
			HierarBehavioralModelSkiplagging   & 0,87   & 0,02 & 1 & 14  & 4 & -          & 890,36    & 0,00 \\ 
			HierarBehavioralModelFull          & 0,88   & 0,02 & 1 & 10  & 2 & -          & 682,86    & 0,00 \\ \hline
			PercentBehavioralModel             & 0,53   & 0,01 & - & -   & - & -          & Infac     & - \\ 
			PercentBehavioralModelFulfillments & 0,67   & 0,02 & 1 & 0   & 1 & -          & 22.112,12 & 0,00 \\ 
			PercentBehavioralModelSkiplagging  & 0,88   & 0,01 & 1 & 14  & 4 & -          & 890,36    & 0,00 \\ 
			PercentBehavioralModelFull         & 0,84   & 0,02 & 1 & 10  & 2 & -          & 682,86    & 0,00 \\ \hline
		\end{tabular}%
	}
	\caption{Resultados para a Instância10}
	\label{tab:resul_instan10}
\end{table}



\chapter{Conclusões e trabalhos futuro}

O presente estudo abordou a problemática de otimizar o Transporte Ferroviário de Passageiros por meio do desenvolvimento de modelos matemáticos de programação inteira mista, explorando dois enfoques principais: demanda independente e demanda comportamental. A aplicação desses modelos em cenários reais possibilitou uma avaliação comparativa detalhada, destacando as diferenças e os benefícios inerentes a cada abordagem.

Em primeiro lugar, do ponto de vista computacional, todos os modelos demonstraram ser robustos e capazes de alcançar soluções ótimas dentro de tempos razoáveis, validando assim sua viabilidade para aplicações práticas na indústria ferroviária.

Em segundo lugar, os modelos mostraram alta eficiência em encontrar a solução ou soluções ótimas dentro do espaço de busca, explorando no máximo um único nó. Até o momento, não foi possível explicar com precisão esse comportamento. No entanto, pode-se afirmar que isso não está relacionado à simplicidade das instâncias utilizadas, pois foram testadas instâncias consideradas de grande porte na indústria, representando até 30 estações e envolvendo até 74.788 variáveis de decisão inteiras.

Em terceiro lugar, observou-se que, embora os modelos comportamentais apresentem um valor da função objetivo ligeiramente inferior ao do modelo independente, a qualidade da solução é superior. Isso reforça que, nesse tipo de problema, além de buscar a maximização do lucro, também se valoriza que a solução seja baseada em características mais próximas da realidade. Destaca-se como um resultado inesperado a semelhança dos valores da função objetivo entre os dois enfoques.

Em quarto lugar, a infactibilidade do modelo PercentBehavioralModel ocorreu devido às características específicas das instâncias instância7, instância8 e instância10, que durante o ajuste da demanda comportamental em função da demanda independente, criou-se uma situação inviável. Isso demonstra que é necessário cuidado ao aplicar esse enfoque, para evitar situações semelhantes.

Trabalhos futuros a serem incluídos na versão final desta proposta incluem:

\begin{itemize}
    \item Aprofundar a revisão bibliográfica: Fortalecer as bases teóricas da pesquisa, explorando ainda mais a literatura relacionada;
    \item Explicar com maior clareza o comportamento dos modelos: Investigar detalhadamente o motivo pelo qual os modelos encontram a solução ótima explorando apenas um nó, além de explicar com mais precisão as causas da inviabilidade do modelo PercentBehavioralModel em certas instâncias;
    \item Explorar outros enfoques para ajustar a demanda: neste caso, optou-se por adicionar restrições que melhorassem seu comportamento. No entanto, também seria possível abordar essa questão alterando a formulação da função objetivo.
\end{itemize}






%=============================== Referências Bibliográficas===========================
\cleardoublepage
\phantomsection

% Aumenta el contador de capítulo y muestra el número manualmente
\refstepcounter{chapter}
% \chapter*{\thechapter\  Referências}

% Añade a la tabla de contenido como capítulo numerado
\addcontentsline{toc}{chapter}{\thechapter\ \hspace{0.234cm} Referências}

\renewcommand{\bibname}{Referências}
\bibliographystyle{plainnat}

\bibliography{sections/referencias}


% %=============================== Anexos ==========================
% \appendix
% \renewcommand{\appendixname}{Appendix} %trocar Apendice por Anexo
% \include{G-Appendix}

\end{document}